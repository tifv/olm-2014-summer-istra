% Auto-generated by jeolm

\documentclass[a4paper,10pt]{article}
\usepackage[T2A]{fontenc}
\usepackage[utf8]{inputenc}
\usepackage[russian]{babel}
\pagestyle{empty}
\usepackage{jeolm}% /_style/jeolm
\usepackage{geometry}
\geometry{margin=2em,lmargin=2em,rmargin=2em}
\geometry{a5paper}
\usepackage{parskip}
\makeatletter
\newcommand\jeolminstitutionname
    {{\upshape[}\/Московская ЛШ по математике{\upshape\/]}}
\newcommand\jeolmdaterange
    {{\upshape[}\/15--28 июня 2014{\upshape\/]}}
\let\jeolmheadertemplate\undefined
\newcommand\jeolmheader{%
{\Large\vspace{4ex}}\par%
\begingroup\small\sffamily%
\strut\ifx\jeolmauthors\relax%
  \hfill{\bfseries\jeolminstitutionname}\hfill
\else
  {\bfseries\jeolminstitutionname}%
    \hfill
  {\mdseries\jeolmauthors}%
\fi\strut\nopagebreak\\%
\strut{\itshape\jeolmdaterange}\hfill
\ifx\jeolmgroupname\relax\else
    \begingroup\edef\x{\endgroup\noexpand\in@{,}{\jeolmgroupname}}\x
    \ifin@ группы: \else группа:\fi
    \enspace{\large\strut\jeolmgroupname}%
\fi\ifx\jeolmdate\relax\else
    \qquad{\itshape\jeolmdate}%
\fi\strut\nopagebreak\\%
\rule[1ex]{\textwidth}{0.5pt} %
\endgroup% \small\sffamily
{\vspace{-1ex}\Large\vspace{-4ex}\vspace{-\parskip}}}
\let\jeolmauthors\relax
\let\jeolmgroupname\relax
\let\jeolmdate\relax
\makeatother

\begin{document}

\input{integral.in.tex}% integral.tex
\tableofcontents

\clearpage

\begingroup \def\jeolmgroupname{Кенгуру, Пингвины}
\begingroup
\def\jeolmauthors{Фёдор Бахарев, Глеб Погудин}
\let\jeolmdate\relax
\jeolmheader
\resetproblem
\endgroup
\addcontentsline{toc}{section}{Отборочная устная олимпиада, 7--8 классы}
\input{olympiad-g78-n1.in.tex}% olympiad/g78-n1.tex
\endgroup% \def\jeolmgroupname

\clearpage

\begingroup
\def\jeolmauthors{Михаил Харитонов}
\def\jeolmgroupname{Носороги}
\let\jeolmdate\relax
\jeolmheader
\resetproblem
\endgroup
\addcontentsline{toc}{section}{Письменная олимпиада, 9 класс}
\input{olympiad-g9.in.tex}% olympiad/g9.tex

\clearpage

\begingroup
\def\jeolmauthors{С.\,Беляков, А.\,Гусев, А.\,Кушнир, О.\,Орлов}
\def\jeolmgroupname{Гризли}
\let\jeolmdate\relax
\jeolmheader
\resetproblem
\endgroup
\addcontentsline{toc}{section}{Письменная олимпиада, 10--11 классы}
\input{olympiad-g1011.in.tex}% olympiad/g1011.tex

\clearpage

\begingroup
\def\jeolmauthors{Алексей Доледенок}
\def\jeolmgroupname{Тигры}
\let\jeolmdate\relax
\jeolmheader
\resetproblem
\endgroup
\addcontentsline{toc}{section}{Письменная олимпиада, 10 класс}
\input{olympiad-g10.in.tex}% olympiad/g10.tex

\clearpage

\begingroup
\def\jeolmauthors{С.\,Беляков, А.\,Гусев, А.\,Кушнир, О.\,Орлов}
\def\jeolmgroupname{Зубры}
\let\jeolmdate\relax
\jeolmheader
\resetproblem
\endgroup
\addcontentsline{toc}{section}{Письменная олимпиада, 11 класс}
\input{olympiad-g11.in.tex}% olympiad/g11.tex

\clearpage

\begingroup \def\jeolmgroupname{Кенгуру, Пингвины}
\begingroup
\def\jeolmauthors{Фёдор Ивлев, Лев Шабанов}
\let\jeolmdate\relax
\jeolmheader
\resetproblem
\endgroup
\addcontentsline{toc}{section}{Заключительная устная олимпиада, 7--8 классы}
\input{olympiad-g78-n2.in.tex}% olympiad/g78-n2.tex
\endgroup% \def\jeolmgroupname

\clearpage

\begingroup
\def\jeolmauthors{Алексей Пономарёв}
\def\jeolmgroupname{Кенгуру}
\let\jeolmdate\relax
\jeolmheader
\resetproblem
\endgroup
\addcontentsline{toc}{section}{Прогрессии и другая борьба с многоточием}
\input{algebra-progression-g78r2.in.tex}% algebra/progression-g78r2.tex

\clearpage

\begingroup
\def\jeolmauthors{Алексей Устинов}
\def\jeolmgroupname{Пингвины}
\let\jeolmdate\relax
\jeolmheader
\resetproblem
\endgroup
\addcontentsline{toc}{section}{Числа Фиббоначи}
\input{algebra-fibbonaci.in.tex}% algebra/fibbonaci.tex

\clearpage

\begingroup
\def\jeolmauthors{Юлий Тихонов}
\def\jeolmgroupname{Носороги}
\let\jeolmdate\relax
\jeolmheader
\resetproblem
\endgroup
\addcontentsline{toc}{section}{Квадратный трёхчлен}
\input{algebra-quadratic-trinomial-g9.in.tex}% algebra/quadratic-trinomial-g9.tex

\clearpage

\begingroup
\def\jeolmauthors{Юлий Тихонов}
\def\jeolmgroupname{Носороги}
\let\jeolmdate\relax
\jeolmheader
\resetproblem
\endgroup
\addcontentsline{toc}{section}{Ещё квадратный трёхчлен}
\input{algebra-quadratic-trinomial-g9-more.in.tex}% algebra/quadratic-trinomial-g9-more.tex

\clearpage

\begingroup
\def\jeolmauthors{Олег Орлов}
\def\jeolmgroupname{Гризли}
\let\jeolmdate\relax
\jeolmheader
\resetproblem
\endgroup
\addcontentsline{toc}{section}{Квадратный трёхчлен}
\input{algebra-quadratic-trinomial-g1011.in.tex}% algebra/quadratic-trinomial-g1011.tex

\clearpage

\begingroup
\def\jeolmauthors{Олег Орлов}
\def\jeolmgroupname{Гризли}
\let\jeolmdate\relax
\jeolmheader
\resetproblem
\endgroup
\addcontentsline{toc}{section}{Добавка по алгебре и теории чисел}
\input{algebra-mixture-g1011.in.tex}% algebra/mixture-g1011.tex

\clearpage

\begingroup
\def\jeolmauthors{Виктор Трещёв}
\def\jeolmgroupname{Гризли}
\let\jeolmdate\relax
\jeolmheader
\resetproblem
\endgroup
\addcontentsline{toc}{section}{Многочлены с целыми коэффициентами}
\input{algebra-polynomial-w-integer-coeff-g1011.in.tex}% algebra/polynomial-w-integer-coeff-g1011.tex

\clearpage

\begingroup
\def\jeolmauthors{Алексей Устинов}
\def\jeolmgroupname{Тигры}
\let\jeolmdate\relax
\jeolmheader
\resetproblem
\endgroup
\addcontentsline{toc}{section}{Многочлены: китайская теорема об остатках, лемма Гаусса, \ldots}
\input{algebra-polynomial-chinese-remainder-theorem.in.tex}% algebra/polynomial-chinese-remainder-theorem.tex

\clearpage

\begingroup
\def\jeolmauthors{Алексей Устинов}
\def\jeolmgroupname{Тигры}
\let\jeolmdate\relax
\jeolmheader
\resetproblem
\endgroup
\addcontentsline{toc}{section}{Интерполяционный многочлен Лагранжа}
\input{algebra-polynomial-interpolation.in.tex}% algebra/polynomial-interpolation.tex

\clearpage

\begingroup
\def\jeolmauthors{Алексей Устинов}
\def\jeolmgroupname{Тигры}
\let\jeolmdate\relax
\jeolmheader
\resetproblem
\endgroup
\addcontentsline{toc}{section}{Производящие функции}
\input{algebra-generating-function.in.tex}% algebra/generating-function.tex

\clearpage

\begingroup
\def\jeolmauthors{Олег Орлов}
\def\jeolmgroupname{Зубры}
\let\jeolmdate\relax
\jeolmheader
\resetproblem
\endgroup
\addcontentsline{toc}{section}{Основная теорема алгебры}
\input{algebra-fundamental-theorem.in.tex}% algebra/fundamental-theorem.tex

\clearpage

\begingroup
\def\jeolmauthors{Олег Орлов}
\def\jeolmgroupname{Зубры}
\let\jeolmdate\relax
\jeolmheader
\resetproblem
\endgroup
\addcontentsline{toc}{section}{Добавка по алгебре и теории чисел}
\input{algebra-mixture-g11.in.tex}% algebra/mixture-g11.tex

\clearpage

\begingroup
\def\jeolmauthors{Алексей Пономарёв}
\def\jeolmgroupname{Кенгуру}
\let\jeolmdate\relax
\jeolmheader
\resetproblem
\endgroup
\addcontentsline{toc}{section}{Неравенства о средних}
\input{algebra-inequality-mean-g78r2.in.tex}% algebra/inequality/mean-g78r2.tex

\clearpage

\begingroup
\def\jeolmauthors{Юлий Тихонов}
\def\jeolmgroupname{Носороги}
\let\jeolmdate\relax
\jeolmheader
\resetproblem
\endgroup
\addcontentsline{toc}{section}{Неравенство о средних}
\input{algebra-inequality-mean-g9.in.tex}% algebra/inequality/mean-g9.tex

\clearpage

\begingroup
\def\jeolmauthors{Олег Орлов}
\def\jeolmgroupname{Гризли}
\let\jeolmdate\relax
\jeolmheader
\resetproblem
\endgroup
\addcontentsline{toc}{section}{Метод Штурма}
\input{algebra-inequality-smoothing.in.tex}% algebra/inequality/smoothing.tex

\clearpage

\begingroup
\def\jeolmauthors{Олег Орлов}
\def\jeolmgroupname{Гризли}
\let\jeolmdate\relax
\jeolmheader
\resetproblem
\endgroup
\addcontentsline{toc}{section}{Неравенства}
\input{algebra-inequality-mixture-g1011.in.tex}% algebra/inequality/mixture-g1011.tex

\clearpage

\begingroup
\def\jeolmauthors{Олег Орлов}
\def\jeolmgroupname{Зубры}
\let\jeolmdate\relax
\jeolmheader
\resetproblem
\endgroup
\addcontentsline{toc}{section}{Неравенство Йенсена}
\input{algebra-inequality-jensen.in.tex}% algebra/inequality/jensen.tex

\clearpage

\begingroup
\def\jeolmauthors{Олег Орлов}
\def\jeolmgroupname{Зубры}
\let\jeolmdate\relax
\jeolmheader
\resetproblem
\endgroup
\addcontentsline{toc}{section}{Неравенства}
\input{algebra-inequality-mixture-g11.in.tex}% algebra/inequality/mixture-g11.tex

\clearpage

\begingroup
\def\jeolmauthors{Фёдор Бахарев}
\def\jeolmgroupname{Кенгуру}
\let\jeolmdate\relax
\jeolmheader
\resetproblem
\endgroup
\addcontentsline{toc}{section}{Серия 1, сравнительно-простая}
\input{algebra-number-theory-modular-arithmetic-g78-r2-1.in.tex}% algebra/number-theory/modular-arithmetic-g78/r2-1.tex
\resetproblem
\addcontentsline{toc}{section}{Серия 2, сравнения и~остатки}
\input{algebra-number-theory-modular-arithmetic-g78-r2-2.in.tex}% algebra/number-theory/modular-arithmetic-g78/r2-2.tex

\clearpage

\begingroup
\def\jeolmauthors{Фёдор Бахарев}
\def\jeolmgroupname{Пингвины}
\let\jeolmdate\relax
\jeolmheader
\resetproblem
\endgroup
\addcontentsline{toc}{section}{Серия 1, сравнительно-простая}
\input{algebra-number-theory-modular-arithmetic-g78-r1-1.in.tex}% algebra/number-theory/modular-arithmetic-g78/r1-1.tex
\resetproblem
\addcontentsline{toc}{section}{Серия 2, всё ещё сравнения, но посложнее}
\input{algebra-number-theory-modular-arithmetic-g78-r1-2.in.tex}% algebra/number-theory/modular-arithmetic-g78/r1-2.tex

\clearpage

\begingroup
\def\jeolmauthors{Фёдор Бахарев}
\def\jeolmgroupname{Кенгуру}
\let\jeolmdate\relax
\jeolmheader
\resetproblem
\endgroup
\addcontentsline{toc}{section}{Серия 3, алгебраический разнобой}
\input{algebra-number-theory-mixture-g78r2.in.tex}% algebra/number-theory/mixture-g78r2.tex

\clearpage

\begingroup
\def\jeolmauthors{Фёдор Бахарев}
\def\jeolmgroupname{Пингвины}
\let\jeolmdate\relax
\jeolmheader
\resetproblem
\endgroup
\addcontentsline{toc}{section}{Серия 3, про показатели}
\input{algebra-number-theory-exponent-g78r1.in.tex}% algebra/number-theory/exponent-g78r1.tex

\clearpage

\begingroup
\def\jeolmauthors{Юлий Тихонов, Михаил Ягудин}
\def\jeolmgroupname{Пингвины}
\let\jeolmdate\relax
\jeolmheader
\resetproblem
\endgroup
\addcontentsline{toc}{section}{Диофантовы уравнения}
\input{algebra-number-theory-diophantine-equations-g78r1.in.tex}% algebra/number-theory/diophantine-equations-g78r1.tex

\clearpage

\begingroup
\def\jeolmauthors{Алексей Устинов}
\def\jeolmgroupname{Пингвины}
\let\jeolmdate\relax
\jeolmheader
\resetproblem
\endgroup
\addcontentsline{toc}{section}{Функция Эйлера}
\input{algebra-number-theory-euler-function-g78r1.in.tex}% algebra/number-theory/euler-function-g78r1.tex

\clearpage

\begingroup
\def\jeolmauthors{Олег Герман}
\def\jeolmgroupname{Носороги}
\let\jeolmdate\relax
\jeolmheader
\resetproblem
\endgroup
\addcontentsline{toc}{section}{Теория чисел}
\input{algebra-number-theory-divisibility-g9.in.tex}% algebra/number-theory/divisibility-g9.tex

\clearpage

\begingroup
\def\jeolmauthors{Михаил Харитонов}
\def\jeolmgroupname{Носороги}
\let\jeolmdate\relax
\jeolmheader
\resetproblem
\endgroup
\addcontentsline{toc}{section}{Теория чисел}
\input{algebra-number-theory-mixture-g9.in.tex}% algebra/number-theory/mixture-g9.tex

\clearpage

\begingroup
\def\jeolmauthors{Олег Орлов}
\def\jeolmgroupname{Гризли}
\let\jeolmdate\relax
\jeolmheader
\resetproblem
\endgroup
\addcontentsline{toc}{section}{Функция Эйлера}
\input{algebra-number-theory-euler-function-g1011.in.tex}% algebra/number-theory/euler-function-g1011.tex

\clearpage

\begingroup
\def\jeolmauthors{Олег Орлов}
\def\jeolmgroupname{Гризли}
\let\jeolmdate\relax
\jeolmheader
\resetproblem
\endgroup
\addcontentsline{toc}{section}{Теория чисел}
\input{algebra-number-theory-mixture-g1011.in.tex}% algebra/number-theory/mixture-g1011.tex

\clearpage

\begingroup
\def\jeolmauthors{Алексей Устинов}
\def\jeolmgroupname{Тигры}
\let\jeolmdate\relax
\jeolmheader
\resetproblem
\endgroup
\addcontentsline{toc}{section}{Китайская теорема об остатках}
\input{algebra-number-theory-chinese-remainder-theorem.in.tex}% algebra/number-theory/chinese-remainder-theorem.tex

\clearpage

\begingroup
\def\jeolmauthors{Андрей Меньщиков}
\def\jeolmgroupname{Тигры}
\let\jeolmdate\relax
\jeolmheader
\resetproblem
\endgroup
\addcontentsline{toc}{section}{Добавка по теории чисел}
\input{algebra-number-theory-mixture-g10.in.tex}% algebra/number-theory/mixture-g10.tex

\clearpage

\begingroup
\def\jeolmauthors{Олег Орлов}
\def\jeolmgroupname{Зубры}
\let\jeolmdate\relax
\jeolmheader
\resetproblem
\endgroup
\addcontentsline{toc}{section}{Теория чисел}
\input{algebra-number-theory-mixture-g11.in.tex}% algebra/number-theory/mixture-g11.tex

\clearpage

\begingroup
\def\jeolmauthors{Андрей Кушнир}
\def\jeolmgroupname{Кенгуру}
\let\jeolmdate\relax
\jeolmheader
\resetproblem
\endgroup
\addcontentsline{toc}{section}{Неравенство треугольника}
\input{geometry-triangle-inequality.in.tex}% geometry/triangle-inequality.tex

\clearpage

\begingroup
\def\jeolmauthors{Андрей Кушнир}
\def\jeolmgroupname{Кенгуру}
\let\jeolmdate\relax
\jeolmheader
\resetproblem
\endgroup
\addcontentsline{toc}{section}{Равенство треугольников}
\input{geometry-equal-triangles-g78r2.in.tex}% geometry/equal-triangles-g78r2.tex

\clearpage

\begingroup
\def\jeolmauthors{Андрей Кушнир}
\def\jeolmgroupname{Кенгуру}
\let\jeolmdate\relax
\jeolmheader
\resetproblem
\endgroup
\addcontentsline{toc}{section}{Геометрические неравенства}
\input{geometry-inequality-g78r2.in.tex}% geometry/inequality-g78r2.tex

\clearpage

\begingroup
\def\jeolmauthors{Андрей Кушнир}
\def\jeolmgroupname{Кенгуру}
\let\jeolmdate\relax
\jeolmheader
\resetproblem
\endgroup
\addcontentsline{toc}{section}{Комбинаторная геометрия}
\input{geometry-combinatorial-g78r2.in.tex}% geometry/combinatorial-g78r2.tex

\clearpage

\begingroup
\def\jeolmauthors{Фёдор Ивлев}
\def\jeolmgroupname{Пингвины}
\let\jeolmdate\relax
\jeolmheader
\resetproblem
\endgroup
\addcontentsline{toc}{section}{Входной разнобой}
\input{geometry-mixture-g78r1-intro.in.tex}% geometry/mixture-g78r1-intro.tex

\clearpage

\begingroup
\def\jeolmauthors{Фёдор Ивлев}
\def\jeolmgroupname{Пингвины}
\let\jeolmdate\relax
\jeolmheader
\resetproblem
\endgroup
\addcontentsline{toc}{section}{Вписанные углы, начало}
\input{geometry-inscribed-angle-g78r1-1.in.tex}% geometry/inscribed-angle-g78r1/1.tex
\resetproblem
\addcontentsline{toc}{section}{Вписанные углы, продолжение}
\input{geometry-inscribed-angle-g78r1-2.in.tex}% geometry/inscribed-angle-g78r1/2.tex

\clearpage

\begingroup
\def\jeolmauthors{Михаил Ягудин}
\def\jeolmgroupname{Пингвины}
\let\jeolmdate\relax
\jeolmheader
\resetproblem
\endgroup
\addcontentsline{toc}{section}{Геометрический разнобой}
\input{geometry-mixture-g78r1.in.tex}% geometry/mixture-g78r1.tex

\clearpage

\begingroup
\def\jeolmauthors{Фёдор Ивлев}
\def\jeolmgroupname{Пингвины}
\let\jeolmdate\relax
\jeolmheader
\resetproblem
\endgroup
\addcontentsline{toc}{section}{Комбинаторная геометрия}
\input{geometry-combinatorial-g78r1.in.tex}% geometry/combinatorial-g78r1.tex

\clearpage

\begingroup
\def\jeolmauthors{Фёдор Бахарев}
\def\jeolmgroupname{Носороги}
\let\jeolmdate\relax
\jeolmheader
\resetproblem
\endgroup
\addcontentsline{toc}{section}{Серия 1, о прямоугольном треугольнике}
\input{geometry-right-triangle-g9.in.tex}% geometry/right-triangle-g9.tex

\clearpage

\begingroup
\def\jeolmauthors{Фёдор Бахарев}
\def\jeolmgroupname{Носороги}
\let\jeolmdate\relax
\jeolmheader
\resetproblem
\endgroup
\addcontentsline{toc}{section}{Серия 2, про геометрические места точек}
\input{geometry-locus-g9.in.tex}% geometry/locus-g9.tex

\clearpage

\begingroup
\def\jeolmauthors{Фёдор Бахарев}
\def\jeolmgroupname{Носороги}
\let\jeolmdate\relax
\jeolmheader
\resetproblem
\endgroup
\addcontentsline{toc}{section}{Серия 3, про радикальную ось и не только}
\input{geometry-radical-axis.in.tex}% geometry/radical-axis.tex

\clearpage

\begingroup
\def\jeolmauthors{Фёдор Ивлев}
\def\jeolmgroupname{Носороги}
\let\jeolmdate\relax
\jeolmheader
\resetproblem
\endgroup
\addcontentsline{toc}{section}{Комбинаторная геометрия}
\input{geometry-combinatorial-g9.in.tex}% geometry/combinatorial-g9.tex

\clearpage

\begingroup
\def\jeolmauthors{Андрей Кушнир}
\def\jeolmgroupname{Гризли}
\let\jeolmdate\relax
\jeolmheader
\resetproblem
\endgroup
\addcontentsline{toc}{section}{Геометрия. Листик первый}
\input{geometry-mixture-g1011-1.in.tex}% geometry/mixture-g1011-1.tex

\clearpage

\begingroup
\def\jeolmauthors{Андрей Кушнир}
\def\jeolmgroupname{Гризли}
\let\jeolmdate\relax
\jeolmheader
\resetproblem
\endgroup
\addcontentsline{toc}{section}{Геометрия. Листик второй}
\input{geometry-mixture-g1011-2.in.tex}% geometry/mixture-g1011-2.tex

\clearpage

\begingroup
\def\jeolmauthors{Андрей Кушнир}
\def\jeolmgroupname{Гризли}
\let\jeolmdate\relax
\jeolmheader
\resetproblem
\endgroup
\addcontentsline{toc}{section}{Добавка по геометрии}
\input{geometry-mixture-g1011-appendage.in.tex}% geometry/mixture-g1011-appendage.tex

\clearpage

\begingroup
\def\jeolmauthors{Алексей Доледенок}
\def\jeolmgroupname{Гризли}
\let\jeolmdate\relax
\jeolmheader
\resetproblem
\endgroup
\addcontentsline{toc}{section}{Они пересекаются!}
\input{geometry-orthocenter-g1011.in.tex}% geometry/orthocenter-g1011.tex

\clearpage

\begingroup
\def\jeolmauthors{Андрей Меньщиков}
\def\jeolmgroupname{Тигры}
\let\jeolmdate\relax
\jeolmheader
\resetproblem
\endgroup
\addcontentsline{toc}{section}{Центр масс, часть 1}
\input{geometry-mass-point-1.in.tex}% geometry/mass-point/1.tex
\resetproblem
\addcontentsline{toc}{section}{Центр масс, часть 2}
\input{geometry-mass-point-2.in.tex}% geometry/mass-point/2.tex
\resetproblem
\addcontentsline{toc}{section}{Центр масс, часть 3}
\input{geometry-mass-point-3.in.tex}% geometry/mass-point/3.tex

\clearpage

\begingroup
\def\jeolmauthors{Алексей Доледенок}
\def\jeolmgroupname{Тигры}
\let\jeolmdate\relax
\jeolmheader
\resetproblem
\endgroup
\addcontentsline{toc}{section}{Прямая Симсона}
\input{geometry-simson-line-g10.in.tex}% geometry/simson-line-g10.tex

\clearpage

\begingroup
\def\jeolmauthors{Алексей Доледенок}
\def\jeolmgroupname{Гризли}
\let\jeolmdate\relax
\jeolmheader
\resetproblem
\endgroup
\addcontentsline{toc}{section}{Немного об Эйлере}
\input{geometry-euler-line-g1011.in.tex}% geometry/euler-line/g1011.tex

\clearpage

\begingroup
\def\jeolmauthors{Алексей Доледенок}
\def\jeolmgroupname{Тигры}
\let\jeolmdate\relax
\jeolmheader
\resetproblem
\endgroup
\addcontentsline{toc}{section}{Немного об Эйлере}
\input{geometry-euler-line-g10.in.tex}% geometry/euler-line/g10.tex

\clearpage

\begingroup
\def\jeolmauthors{Андрей Кушнир}
\def\jeolmgroupname{Зубры}
\let\jeolmdate\relax
\jeolmheader
\resetproblem
\endgroup
\addcontentsline{toc}{section}{Проективная геометрия}
\input{geometry-projective-g11-basics.in.tex}% geometry/projective-g11/basics.tex
\resetproblem
\addcontentsline{toc}{section}{Проективные преобразования. Окружность}
\input{geometry-projective-g11-circle.in.tex}% geometry/projective-g11/circle.tex

\clearpage

\begingroup
\def\jeolmauthors{Андрей Кушнир}
\def\jeolmgroupname{Зубры}
\let\jeolmdate\relax
\jeolmheader
\resetproblem
\endgroup
\addcontentsline{toc}{section}{Поляры}
\input{geometry-polar-line-g11.in.tex}% geometry/polar-line-g11.tex

\clearpage

\begingroup \def\jeolmgroupname{Тигры, Зубры}
\begingroup
\def\jeolmauthors{Александр Полянский}
\let\jeolmdate\relax
\jeolmheader
\resetproblem
\endgroup
\addcontentsline{toc}{section}{Воробьями по пушкам и окрестности}
\input{geometry-sparrows-1.in.tex}% geometry/sparrows/1.tex
\endgroup% \def\jeolmgroupname
\begingroup \def\jeolmgroupname{Тигры, Зубры}
\addcontentsline{toc}{section}{Воробьями по пушкам, продолжение}
\input{geometry-sparrows-2.in.tex}% geometry/sparrows/2.tex
\endgroup% \def\jeolmgroupname

\clearpage

\begingroup
\def\jeolmauthors{Глеб Погудин}
\def\jeolmgroupname{Кенгуру}
\let\jeolmdate\relax
\jeolmheader
\resetproblem
\endgroup
\addcontentsline{toc}{section}{Алгоритмические задачи. Поиск}
\input{combinatorics-search-g78r2.in.tex}% combinatorics/search-g78r2.tex

\clearpage

\begingroup
\def\jeolmauthors{Глеб Погудин}
\def\jeolmgroupname{Кенгуру}
\let\jeolmdate\relax
\jeolmheader
\resetproblem
\endgroup
\addcontentsline{toc}{section}{Алгоритмические задачи. Взвешивания}
\input{combinatorics-weighing-g78r2.in.tex}% combinatorics/weighing-g78r2.tex

\clearpage

\begingroup
\def\jeolmauthors{Глеб Погудин}
\def\jeolmgroupname{Кенгуру}
\let\jeolmdate\relax
\jeolmheader
\resetproblem
\endgroup
\addcontentsline{toc}{section}{Алгоритмические задачи. Разное}
\input{combinatorics-algorithm-g78r2.in.tex}% combinatorics/algorithm-g78r2.tex

\clearpage

\begingroup
\def\jeolmauthors{Глеб Погудин}
\def\jeolmgroupname{Кенгуру}
\let\jeolmdate\relax
\jeolmheader
\resetproblem
\endgroup
\addcontentsline{toc}{section}{Перечислительные задачи}
\input{combinatorics-counting-g78r2-1.in.tex}% combinatorics/counting-g78r2/1.tex
\resetproblem
\addcontentsline{toc}{section}{Все ещё перечислительные задачи}
\input{combinatorics-counting-g78r2-2.in.tex}% combinatorics/counting-g78r2/2.tex

\clearpage

\begingroup
\def\jeolmauthors{Глеб Погудин}
\def\jeolmgroupname{Кенгуру}
\let\jeolmdate\relax
\jeolmheader
\resetproblem
\endgroup
\addcontentsline{toc}{section}{Включения-исключения}
\input{combinatorics-inclusion-exclusion-g78r2.in.tex}% combinatorics/inclusion-exclusion-g78r2.tex

\clearpage

\begingroup
\def\jeolmauthors{Глеб Погудин}
\def\jeolmgroupname{Кенгуру}
\let\jeolmdate\relax
\jeolmheader
\resetproblem
\endgroup
\addcontentsline{toc}{section}{Логические задачи}
\input{combinatorics-logic-g78r2.in.tex}% combinatorics/logic-g78r2.tex

\clearpage

\begingroup
\def\jeolmauthors{Глеб Погудин}
\def\jeolmgroupname{Кенгуру}
\let\jeolmdate\relax
\jeolmheader
\resetproblem
\endgroup
\addcontentsline{toc}{section}{Комбинаторный разнобой}
\input{combinatorics-mixture-g78r2.in.tex}% combinatorics/mixture-g78r2.tex

\clearpage

\begingroup
\def\jeolmauthors{Лев Шабанов}
\def\jeolmgroupname{Кенгуру}
\let\jeolmdate\relax
\jeolmheader
\resetproblem
\endgroup
\addcontentsline{toc}{section}{Принцип крайнего}
\input{combinatorics-extreme-principle-g78r2.in.tex}% combinatorics/extreme-principle-g78r2.tex

\clearpage

\begingroup
\def\jeolmauthors{Владимир Шарич}
\def\jeolmgroupname{Пингвины}
\let\jeolmdate\relax
\jeolmheader
\resetproblem
\endgroup
\addcontentsline{toc}{section}{Математические игры, часть 0 (Шутки)}
\input{combinatorics-game-g78r1-joke.in.tex}% combinatorics/game-g78r1/joke.tex
\resetproblem
\addcontentsline{toc}{section}{Математические игры, часть 1 (Симметрия)}
\input{combinatorics-game-g78r1-symmetry.in.tex}% combinatorics/game-g78r1/symmetry.tex
\resetproblem
\addcontentsline{toc}{section}{Математические игры, часть 2 (Сумма)}
\input{combinatorics-game-g78r1-sum.in.tex}% combinatorics/game-g78r1/sum.tex
\resetproblem
\addcontentsline{toc}{section}{Математические игры, часть 3 (Преследования)}
\input{combinatorics-game-g78r1-chase.in.tex}% combinatorics/game-g78r1/chase.tex
\resetproblem
\addcontentsline{toc}{section}{Математические игры, часть 4 (Позиции)}
\input{combinatorics-game-g78r1-win-position.in.tex}% combinatorics/game-g78r1/win-position.tex
\resetproblem
\addcontentsline{toc}{section}{Математические игры, часть 5 (Разнобой)}
\input{combinatorics-game-g78r1-mixture.in.tex}% combinatorics/game-g78r1/mixture.tex

\clearpage

\begingroup
\def\jeolmauthors{Лев Шабанов}
\def\jeolmgroupname{Пингвины}
\let\jeolmdate\relax
\jeolmheader
\resetproblem
\endgroup
\addcontentsline{toc}{section}{Взвешивания}
\input{combinatorics-weighing-g78r1.in.tex}% combinatorics/weighing-g78r1.tex

\clearpage

\begingroup
\def\jeolmauthors{Лев Шабанов}
\def\jeolmgroupname{Пингвины}
\let\jeolmdate\relax
\jeolmheader
\resetproblem
\endgroup
\addcontentsline{toc}{section}{Алгоритмы и информационные оценки}
\input{combinatorics-information-g78r1.in.tex}% combinatorics/information-g78r1.tex

\clearpage

\begingroup
\def\jeolmauthors{Иван Митрофанов}
\def\jeolmgroupname{Носороги}
\let\jeolmdate\relax
\jeolmheader
\resetproblem
\endgroup
\addcontentsline{toc}{section}{Раскраски}
\jeolmfiguremap{t-tetramino}{combinatorics-painting-g9-t-tetramino}
\input{combinatorics-painting-g9.in.tex}% combinatorics/painting-g9.tex

\clearpage

\begingroup
\def\jeolmauthors{Иван Митрофанов}
\def\jeolmgroupname{Носороги}
\let\jeolmdate\relax
\jeolmheader
\resetproblem
\endgroup
\addcontentsline{toc}{section}{Инварианты}
\jeolmfiguremap{signes}{combinatorics-invariant-g9-signes}
\jeolmfiguremap{rooks}{combinatorics-invariant-g9-rooks}
\input{combinatorics-invariant-g9.in.tex}% combinatorics/invariant-g9.tex

\clearpage

\begingroup
\def\jeolmauthors{Иван Митрофанов}
\def\jeolmgroupname{Носороги}
\let\jeolmdate\relax
\jeolmheader
\resetproblem
\endgroup
\addcontentsline{toc}{section}{Полуинварианты и процессы}
\jeolmfiguremap{signes-1}{combinatorics-semi-invariant-g9-signes-1}
\jeolmfiguremap{signes-2}{combinatorics-semi-invariant-g9-signes-2}
\input{combinatorics-semi-invariant-g9.in.tex}% combinatorics/semi-invariant-g9.tex

\clearpage

\begingroup
\def\jeolmauthors{Иван Митрофанов}
\def\jeolmgroupname{Носороги}
\let\jeolmdate\relax
\jeolmheader
\resetproblem
\endgroup
\addcontentsline{toc}{section}{Примеры}
\input{combinatorics-constructive-g9.in.tex}% combinatorics/constructive-g9.tex

\clearpage

\begingroup
\def\jeolmauthors{Владимир Брагин}
\def\jeolmgroupname{Носороги}
\let\jeolmdate\relax
\jeolmheader
\resetproblem
\endgroup
\addcontentsline{toc}{section}{Взгляд в бесконечность, часть 1}
\input{combinatorics-infinity-g9-1.in.tex}% combinatorics/infinity-g9/1.tex

\clearpage

\begingroup
\def\jeolmauthors{Владимир Брагин}
\def\jeolmgroupname{Носороги}
\let\jeolmdate\relax
\jeolmheader
\resetproblem
\endgroup
\addcontentsline{toc}{section}{Взгляд в бесконечность, часть 2}
\input{combinatorics-infinity-g9-2.in.tex}% combinatorics/infinity-g9/2.tex

\clearpage

\begingroup
\def\jeolmauthors{Михаил Харитонов}
\def\jeolmgroupname{Носороги}
\let\jeolmdate\relax
\jeolmheader
\resetproblem
\endgroup
\addcontentsline{toc}{section}{Игры}
\input{combinatorics-game-g9.in.tex}% combinatorics/game-g9.tex

\clearpage

\begingroup
\def\jeolmauthors{Михаил Харитонов}
\def\jeolmgroupname{Носороги}
\let\jeolmdate\relax
\jeolmheader
\resetproblem
\endgroup
\addcontentsline{toc}{section}{Негеометрическая комбинаторная геометрия}
\input{combinatorics-geometric-g9.in.tex}% combinatorics/geometric-g9.tex

\clearpage

\begingroup
\def\jeolmauthors{Владимир Шарич}
\def\jeolmgroupname{Тигры}
\let\jeolmdate\relax
\jeolmheader
\resetproblem
\endgroup
\addcontentsline{toc}{section}{Рекурренты в комбинаторике, часть 1 (Зацикливания)}
\input{combinatorics-recurrent-g10-loop.in.tex}% combinatorics/recurrent-g10/loop.tex
\resetproblem
\addcontentsline{toc}{section}{Рекурренты в комбинаторике, часть 2 (Одномерные задачи)}
\input{combinatorics-recurrent-g10-one-dimension.in.tex}% combinatorics/recurrent-g10/one-dimension.tex
\resetproblem
\addcontentsline{toc}{section}{Рекурренты в комбинаторике, часть 3 (Фиббоначи)}
\input{combinatorics-recurrent-g10-fibbonaci.in.tex}% combinatorics/recurrent-g10/fibbonaci.tex
\resetproblem
\addcontentsline{toc}{section}{Рекурренты в комбинаторике, часть 4 (Комбинации)}
\input{combinatorics-recurrent-g10-binomial.in.tex}% combinatorics/recurrent-g10/binomial.tex
\resetproblem
\addcontentsline{toc}{section}{Рекурренты в комбинаторике, часть 5 (Каталан)}
\input{combinatorics-recurrent-g10-catalan.in.tex}% combinatorics/recurrent-g10/catalan.tex

\clearpage

\begingroup
\def\jeolmauthors{Сергей Беляков}
\def\jeolmgroupname{Гризли}
\let\jeolmdate\relax
\jeolmheader
\resetproblem
\endgroup
\addcontentsline{toc}{section}{Немного об играх}
\input{combinatorics-game-g1011.in.tex}% combinatorics/game-g1011.tex

\clearpage

\begingroup
\def\jeolmauthors{Сергей Беляков}
\def\jeolmgroupname{Гризли}
\let\jeolmdate\relax
\jeolmheader
\resetproblem
\endgroup
\addcontentsline{toc}{section}{Инварианты и полуинварианты}
\input{combinatorics-semi-invariant-g1011.in.tex}% combinatorics/semi-invariant-g1011.tex

\clearpage

\begingroup
\def\jeolmauthors{Сергей Беляков}
\def\jeolmgroupname{Гризли}
\let\jeolmdate\relax
\jeolmheader
\resetproblem
\endgroup
\addcontentsline{toc}{section}{Примитивная классика}
\input{combinatorics-counting-g1011.in.tex}% combinatorics/counting-g1011.tex

\clearpage

\begingroup
\def\jeolmauthors{Антон Гусев}
\def\jeolmgroupname{Зубры}
\let\jeolmdate\relax
\jeolmheader
\resetproblem
\endgroup
\addcontentsline{toc}{section}{Игры}
\input{combinatorics-game-g11.in.tex}% combinatorics/game-g11.tex

\clearpage

\begingroup
\def\jeolmauthors{Антон Гусев}
\def\jeolmgroupname{Зубры}
\let\jeolmdate\relax
\jeolmheader
\resetproblem
\endgroup
\addcontentsline{toc}{section}{Инварианты}
\input{combinatorics-invariant-g11.in.tex}% combinatorics/invariant-g11.tex

\clearpage

\begingroup
\def\jeolmauthors{Алексей Канель}
\def\jeolmgroupname{Зубры}
\let\jeolmdate\relax
\jeolmheader
\resetproblem
\endgroup
\addcontentsline{toc}{section}{Полуинварианты}
\input{combinatorics-semi-invariant-g11.in.tex}% combinatorics/semi-invariant-g11.tex

\clearpage

\begingroup
\def\jeolmauthors{Лев Шабанов}
\def\jeolmgroupname{Кенгуру}
\let\jeolmdate\relax
\jeolmheader
\resetproblem
\endgroup
\addcontentsline{toc}{section}{Графы, часть 1. Общие соображения}
\input{combinatorics-graph-g78r2-basics.in.tex}% combinatorics/graph/g78r2/basics.tex
\resetproblem
\addcontentsline{toc}{section}{Графы, часть 2. Двудольные}
\input{combinatorics-graph-g78r2-bipartite.in.tex}% combinatorics/graph/g78r2/bipartite.tex
\resetproblem
\addcontentsline{toc}{section}{Графы, часть 3. Ориентированные}
\input{combinatorics-graph-g78r2-oriented.in.tex}% combinatorics/graph/g78r2/oriented.tex

\clearpage

\begingroup
\def\jeolmauthors{Антон Гусев}
\def\jeolmgroupname{Гризли}
\let\jeolmdate\relax
\jeolmheader
\resetproblem
\endgroup
\addcontentsline{toc}{section}{Графы. Введение}
\input{combinatorics-graph-g1011-mixture.in.tex}% combinatorics/graph/g1011/mixture.tex
\resetproblem
\addcontentsline{toc}{section}{Ориентированные графы}
\input{combinatorics-graph-g1011-oriented.in.tex}% combinatorics/graph/g1011/oriented.tex

\clearpage

\begingroup
\def\jeolmauthors{Антон Гусев}
\def\jeolmgroupname{Гризли}
\let\jeolmdate\relax
\jeolmheader
\resetproblem
\endgroup
\addcontentsline{toc}{section}{Планарные графы}
\jeolmfiguremap{petersen}{combinatorics-graph-planar-petersen}
\input{combinatorics-graph-planar.in.tex}% combinatorics/graph/planar.tex

\clearpage

\begingroup
\def\jeolmauthors{Антон Гусев}
\def\jeolmgroupname{Тигры}
\let\jeolmdate\relax
\jeolmheader
\resetproblem
\endgroup
\addcontentsline{toc}{section}{Раскраски графов, часть 1}
\input{combinatorics-graph-painting-1.in.tex}% combinatorics/graph/painting/1.tex
\resetproblem
\addcontentsline{toc}{section}{Раскраски графов, часть 2}
\input{combinatorics-graph-painting-2.in.tex}% combinatorics/graph/painting/2.tex

\clearpage

\begingroup
\def\jeolmauthors{Антон Гусев}
\def\jeolmgroupname{Зубры}
\let\jeolmdate\relax
\jeolmheader
\resetproblem
\endgroup
\addcontentsline{toc}{section}{Графы}
\input{combinatorics-graph-mixture-g11.in.tex}% combinatorics/graph/mixture-g11.tex

\clearpage


\end{document}
