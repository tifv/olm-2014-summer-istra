% $date: 2014-06-21

% $timetable:
%   g11: 
%     2014-06-21:
%     - 2

\section*{Полуинварианты}

% $authors:
% - Алексей Канель

\begin{problems}

\item
В~парламенте у~каждого парламентария есть не~более трех врагов.
Разделите парламент на~две палаты так, чтобы у~каждого парламентария было
не~более одного врага в~его палате.

\item
Дано $n$ точек, никакие три из~которых не~лежат на~одной прямой.
Постройте несамопересекающуюся ломаную с~узлами в~этих точках.

\item
В~конкурсе <<Международный фестиваль патриотической песни>> участвуют несколько
стран.
Известно, что каждая песня оскорбительна не~более чем для трех других стран.
Каждая страна, исполнив песню, тут~же уезжает и~не~слышит песни, прозвучавшие
после неё.
Расположите страны в~таком порядке, чтобы каждая страна выслушала не~более трех
оскорблений.

\item
На~плоскости даны $n$ красных и~$n$ синих точек.
Пронумеруйте точки каждого цвета числами от~1 до~$n$ так, чтобы отрезки,
соединяющие точки с~одинаковыми номерами, не~пересекались.

\item
На~плоскости даны $n$~точек и~$n$~попарно непараллельных прямых.
Пронумеруйте точки и~прямые числами от~1 до~$n$ так, чтобы отрезки
перпендикуляров, опущенных из~соответствующих точек на~соответствующие прямые,
соединяющие точки с~одинаковыми номерами, не~пересекались.

\item
Дан граф, каждая вершина которого степени не~больше 5 (не~более 5~ребер
в~каждой вершине).
Докажите, что можно <<почти правильно>> раскрасить его в~три цвета, а~именно,
так, чтобы не~более $[n / 2]$ ребер соединяли вершины одного цвета.

\item
Дан граф~--- несколько городов, соединенных дорогами так, что из~каждого города
выходит нечетное число дорог.
Некоторые из~городов раскрашены в~красный цвет, а~некоторые~--- в~белый.
В~городе может произойти революция, если большинство его соседей раскрашено
не~в~тот~же цвет, что он~сам.
Каждый день ровно в~одном из~городов происходит революция, и~он~меняет цвет
на~тот, в~который раскрашено большинство его соседей.
Докажите, что в~конце концов революции прекратятся.

\item
Дан граф~--- несколько городов, соединенных дорогами так, что из~каждого города
выходит нечетное число дорог.
Некоторые из~городов раскрашены в~красный цвет, а~некоторые~--- в~белый.
В~городе может произойти революция, если большинство его соседей раскрашено
не~в~тот~же цвет, что он~сам.
Каждый ход революция одновременно происходит во всех городах, в~которых она
может произойти, и~они меняют цвет на~тот, в~который было раскрашено
большинство их~соседей.
Докажите, что начиная с~определенного момента любой город либо остановится
на~некотором цвете, либо будет менять цвет каждый ход.

\item
На~книжной полке каким-то образом расставлены тома полного собрания сочинений
Васи Пупкина.
Пьяный библиотекарь пытается расставить их~по~порядку.
Для этого он~берет два тома (не~обязательно соседних), которые стоят
относительно друг друга не~по~порядку (то есть больший номер раньше меньшего),
и~переставляет их~местами.
Докажите, что в~конце концов он~расставит книги по~порядку.

\item
На~книжной полке каким-то образом расставлены тома полного собрания сочинений
Васи Пупкина.
Пьяный библиотекарь пытается расставить их~по~порядку.
Для этого он~берет какой-то том, стоящий не~на~своем месте, сдвигает несколько
промежуточных томов, и~ставит этот том на~место.
Докажите, что в~конце концов он~расставит книги по~порядку.

\item
В~некотором порядке расставлены $m n + 1$ различных чисел.
Докажите, что можно указать либо $m + 1$ чисел, расположенных в~порядке
убывания, либо $n + 1$ чисел, расположенных в~порядке возрастания.

\item
\emph{Звёздчатый многоугольник}~--- это многоугольник, из~некоторой точки
внутри которого можно увидеть все вершины и~стороны.
Пусть дан звёздчатый многоугольник.
С~ним производится следующая операция:
берутся два соседних ребра, образующих невыпуклый угол, на~них строится
параллелограмм вовне многоугольника.
Исходные ребра стираются, вместо них вставляются противоположные ребра
параллелограмма.
\\
Докажите, что такой процесс в~конце концов приведет к~выпуклому многоугольнику,
и~остановится.

\item
\sbp
В~парламенте некоторые депутаты залепили некоторым другим пощечины, причем
каждый депутат залепил не~более одной пощечины
(пощечины не~рефлексивны: если $A$ ударил $B$, то, возможно, не~наоборот).
Докажите, что можно разбить парламент на~три палаты так, чтобы в~каждой палате
не~было дерущихся пар депутатов.
\\
\sbp
Решите задачу в~случае не~более двух пощечин и~пяти палат.

\end{problems}

