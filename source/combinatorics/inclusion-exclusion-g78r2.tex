% $date: 2014-06-20

% $timetable:
%   g78r2:
%     2014-06-20:
%     - 1

\section*{Включения-исключения}

% $authors:
% - Глеб Погудин

\subsection*{Вдогонку ко вчерашнему}

\emph{Для тех, кто не решил вчерашнюю задачу №11}

\begin{problems}

\item
Сколькими способами можно написать в строчку $20$ слов так, чтобы сначала было
написано несколько (возможно, ноль) слов \texttt{МАМА}, потом несколько
(возможно, ноль) слов \texttt{МЫЛА}, потом несколько (возможно, ноль) слов
\texttt{РАМУ}?
\emph{(Указание: попробуйте свести эту задачу к вчерашней задаче №7.)}

\item
А если ещё в конце несколько (возможно, ноль) слов \texttt{ДОЛГО}?

\item
Сколько решений в натуральных числах имеет уравнение $x + y + z = 15$?

\end{problems}


\subsection*{Включения-исключения}

\begin{problems}

\item
Сколько из чисел от $1$ до $2000$ делятся $2$, $5$ и $7$, но не делятся на $4$?

\item
Среди животных в саванне некоторые умные, а некоторые красивые.
Известно, что $20\%$ умных ещё и красивы, а $25\%$ красивых ещё и умны.
Только заяц и ишак не являются ни умными, ни красивыми.
Сколько всего животных, если их от $20$ до $30$?

\item
Сколькими способами можно рассадить десять сторожей по четырем сторожкам, чтобы
ни одна сторожка не оставалась без присмотра?

\item
В каждой из комнат замка или пол, или потолок, или стены покрашен в зеленый.
В $20$ комнатах зеленый потолок, в $30$ комнатах зеленый пол и в $10$ комнатах
зеленые стены.
Кроме того, в двух комнатах зелеными являются и стены, и потолок, в трех
комнатах зеленые и пол и потолок, и в четырех комнатах~--- стены и пол.
Могло ли в замке быть $55$ комнат?

% spell "\text{кв.\,м.}" -> "\text{квадратных метров}"

\item
Площадь комнаты~--- $6\,\text{кв.\,м.}$, на нем лежат три ковра, каждый площадью три
квадратных метра.
Докажите, что какие-то два из них перекрываются по площади хотя бы в один
квадратный метр.

\item
На столе лежит несколько карточек.
Известно, что для каждого натурального $n < 1000$ ровно на $n$ карточках
написаны какие-то делители числа $n$.
Докажите, что для любого $n < 1000$ на одной из карточек написано число $n$.

\end{problems}

