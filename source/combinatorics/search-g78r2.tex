% $date: 2014-06-16

% $timetable:
%   g78r2:
%     2014-06-16:
%     - 1

\section*{Алгоритмические задачи. Поиск}

% $authors:
% - Глеб Погудин

\subsection*{На занятии}

\begin{problems}

\item
Иван загадал число от $1$ до $4$.
Как за два вопроса узнать, что это за число?

\item
Иван загадал число от $1$ до $8$.
Как за $3$ вопроса узнать, что это было за число?

\item
Можно ли это узнать за два вопроса?

\item
Вы все еще можете задать три вопроса, но должны сразу предъявить список из этих трех
вопросов.
Как узнать число?

\item
Какое наименьшее число вопросов потребуется с заранее предъявленным планом для числа от
$1$ до $100$?

\item
Иван загадал число от $1$ до $n$.
Разрешено задавать ему вопросы, у которых не более $10$ вариантов ответа.
Какое наименьшее число заранее предъявленных вопросов потребуется, чтобы определить
число?

\item
То же самое, но задавать можно только обычные да/нет вопросы.

\end{problems}


\subsection*{На дом}

\begin{problems}

\item
Иван загадал число от $1$ до $8$, но разрешил задавать вопросы:
<<Является ли твое число делителем числа $X$?>>.
Как за три вопроса узнать число?

\item
Иван загадал клетку квадрата $5$ на $5$.
За один вопрос разрешается выбрать в квадрате клетчатый прямоугольник и спросить, лежит
ли в нем загаданная клетка.
За какое минимальное число вопросов можно угадать клетку?

\item
Иван загадал число от $1$ до $9$.
Разрешается задавать вопросы, на которые он может ответить <<да>>, <<нет>> или
<<не знаю>> (отвечает, как обычно, честно).
Как за два вопроса узнать число?

\item
Неторопливый мальчик Григорий загадал натуральное число от $1$ до $21$.
Ему можно задавать вопросы вида <<Твое число больше $X$?>>.
Проблема в том, что Григорий отвечает на очередной вопрос только после того, как ему
задали следующий (в частности, ответ на последний вопрос вы не получите никогда).
Как за семь вопросов узнать число?

\end{problems}

