% $date: 2014-06-23

% $timetable:
%   g78r1:
%     2014-06-23:
%     - 1

\section*{Взвешивания}

% $authors:
% - Лев Шабанов

\begin{problems}

\item
Лиса Алиса и Кот Базилио~--- фальшивомонетчики.
Базилио делает монеты тяжелее настоящих, а Алиса~--- легче.
У Буратино есть 15 одинаковых по внешнему виду монет, но какая-то одна~---
фальшивая.
Как двумя взвешиваниями на чашечных весах без гирь Буратино может определить,
кто сделал фальшивую монету~--- Кот Базилио или Лиса Алиса?

\item
Среди $201$ монеты $50$ фальшивых.
Каждая фальшивая отличается от настоящей по весу на 1 грамм
(в ту или в другую сторону).
Имеются чашечные весы со стрелкой, показывающей разность масс одной и другой
чашки.
Как за одно взвешивание узнать, является ли заранее выбранная монета настоящей?

\item
Есть $100$ кучек по $100$ монет в каждой, ровно в одной кучке все монеты
фальшивые (на грамм легче настоящих), а в остальных все настоящие.
За какое наименьшее количество взвешиваний на чашечных весах со стрелкой,
показывающей разницу между массами монет на чашках, определить кучку
с~фальшивыми монетами?

% spell "\text{г}" -> "\text{грамм}"

\item
Есть 6 гирек массами
$1\,\text{г}$, $2\,\text{г}$, $\ldots$, $6\,\text{г}$
и~с~такими же наклейками.
Как за 2 взвешивания на чашечных весах убедиться, что наклейки наклеены верно?

\item
Есть $2008$ монет, среди которых не более $2$ фальшивых
(отличных по весу от настоящих и равных между собой).
Как за $3$ взвешивания определить, есть ли фальшивые монеты, и если есть, то
легче они или тяжелее настоящих (количество определять не нужно)?

\item
\sbp
Есть 27 внешне неотличимых монет.
Известно, что одна из них фальшивая (легче настоящей).
За какое наименьшее количество взвешиваний на чашечных весах без гирь можно
найти фальшивую монету?
\\
\sbp
Есть золотые и серебряные слитки, всего $81$ штука, среди которых есть один
фальшивый, причем золото подделывают более легкой латунью, а серебро более
тяжелым свинцом.
За какое наименьшее число взвешиваний на чашечных весах без гирь можно найти
фальшивый слиток?
\\
\sbp
Есть 39 внешне неотличимых монет, одна из которых фальшивая
(отличается по весу от настоящей).
За какое наименьшее количество взвешиваний на чашечных весах без гирь можно
найти фальшивую монету и определить, легче она или тяжелее настоящей?
\\
\sbp
Есть 40 внешне неотличимых монет, одна из которых фальшивая
(отличается по весу от настоящей).
За какое наименьшее количество взвешиваний на чашечных весах без гирь можно
найти фальшивую монету, не обязательно определяя легче она или тяжелее
настоящей?
\\
\sbp Обобщите предыдущие пункты на случай $n$ монет/слитков.

\item
На суд в качестве вещественного доказательства было представлено $14$ монет.
Суду известно, что фальшивые монеты легче настоящих.
Эксперт обнаружил, что монеты с $1$-й по $7$-ю фальшивые, а с $8$-й по $14$-ю
настоящие.
Как за $3$ взвешивания эксперт сможет убедить суд в своей правоте?

\item
Есть $27$ монет достоинством $1$, $2$, $5$ рублей, по $9$ штук каждого вида.
Одна из них фальшивая (легче настоящей).
Настоящие монеты весят пропорционально своему достоинству
($1$, $2$, $5$ грамм).
За какое наименьшее число взвешиваний можно наверняка определить фальшивую
монету, если нельзя класть на весы заведомо настоящие монеты?

\item
Есть 9 фальшивых монет, среди которых есть 1 фальшивая монета (легче настоящей)
и трое чашечных весов, одни из которых неисправны и показывают случайный
результат.
Как за 4 взвешивания найти фальшивую монету?

\end{problems}

