% $date: 2014-06-18

% $timetable:
%   g78r2:
%     2014-06-18:
%     - 2

\section*{Перечислительные задачи}

% $authors:
% - Глеб Погудин

\subsection*{Цепочки}

\begin{problems}

\item
В классе $20$ человек.
Сколькими способами можно выставить им оценки (от 2 до 5) за контрольную?

\item
Сколько существует шестизначных чисел с цифрами одной четности?

\item
Сколькими способами можно расставить числа от $1$ до $100$ по кругу?

\item
Сколько существует десятизначных чисел, у которых среди любых трех цифр подряд
все различны?

\item
Есть четыре мальчика и четыре девочки.
Сколькими способами можно посадить их на лавочку, чтобы пол детей чередовался?

\item
Сколько существует цепочек длины десять из различных букв русского алфавита?

\item
Сколькими способами можно переставить буквы в слове АНАГРАММА?

\item
У сороконожки есть $40$ носков и $40$ ботинок.
Она может надевать и в любом порядке с одним лишь условием: на каждую ногу надо
сначала надеть носок, а только потом ботинок.
Сколькими способами она может обуться? 

\end{problems}


\subsection*{Кучки}

\begin{problems}

\item
Сколько различных делителей у числа $2^{10} 3^{7} 5^{3}$?

\item
Сколькими способами можно разбить класс из двадцати детей на три группы,
причем так, чтобы в каждой группе было ненулевое количество человек?

\item
Сколькими способами можно из этого же класса выбрать две волейбольные команды
по шесть человек?

\item
В классе 10 человек.
Сколькими способами можно из них выбрать команду из пяти человек и назначить в
ней капитана?

\item
Сколькими способами можно выбрать из десяти человек больше половины людей? 

\item
Сколькими способами можно выбрать в множестве из десяти элементов два
подмножества $A$ и $B$, которые бы в объединении давали бы все множество?

\end{problems}

