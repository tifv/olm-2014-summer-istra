% $date: 2014-06-19

% $timetable:
%   g78r2:
%     2014-06-19:
%     - 1

\section*{Все ещё перечислительные задачи}

% $authors:
% - Глеб Погудин


\subsection*{Тренировка с цешечками}

Через $C_n^k$ обозначается количество способов выбрать набор из $k$ предметов из
множества из $n$ различных предметов.
Для этого числа имеется формула $C_n^k = \frac{n!}{k!(n - k)!}$.

В нижеследующих задачах нужно использовать определение, а не формулу!

\begin{problems}

\item
Докажите, что $C_n^{k} = C_n^{n - k}$;

\item
Докажите, что $C_n^{k} = C_{n - 1}^k + C_{n - 1}^{k - 1}$;

\item
Докажите, что $C_n^0 + C_n^1 + \ldots + C_n^n = 2^n$;

\item
Чему равно $C_n^1 + 2 C_n^2 + 3 C_n^3 + \ldots + n C_n^n$?

\end{problems}

В следующих задачах пользуйтесь, чем хотите:

\begin{problems}

\item
Чему равно $C_{100}^0 + C_{100}^1 + \ldots + C_{100}^{49}$?

\item
Пусть $p$~--- простое число, и $1 \leq k < p$.
Докажите, что $C_p^k$ делится на $p$.

\end{problems}


\subsection*{Кучки, делающие вид, что они цепочки и цепочки, делающие вид, что они
кучки}

\begin{problems}

\item
Есть шесть одинаковых яблок и десять одинаковых груш.
Сколькими способами можно выложить их в ряд?

\item
А если там есть ещё четыре апельсина?

\item
Сколько существует шестизначных чисел, у которых каждая следующая цифра меньше
предыдущей?

\item
Сколькими способами можно выбрать четыре числа от $1$ до $20$, чтобы любые два
выбранных числа отличались хотя бы на два?

\item
Мальчик Петя пошел в магазин за пирожными.
Пирожных в магазине пять типов, денег у Пети хватает на десять пирожных.
Сколькими способами может Петя сделать покупки?

\end{problems}

