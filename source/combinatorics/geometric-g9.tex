% $date: 2014-06-27

% $timetable:
%   g9:
%     2014-06-27:
%     - 1

\section*{Негеометрическая комбинаторная геометрия}

% $authors:
% - Михаил Харитонов


\subsection*{Комбинаторная комбинаторная геометрия}

\begin{problems}

\item
На~шахматной доске расставляют королей так, чтобы они били все клетки.
Каково наименьшее число королей для этого нужно?

\item
Из~листа клетчатой бумаги $29 \times 29$ клеток вырезали 99~квадратов
$2 \times 2$.
Докажите, что из~остатка можно вырезать ещё один такой квадрат.

\end{problems}


\subsection*{Просто комбинаторная геометрия}

\begin{problems}

\item
Прожектор освещает прямой угол.
Четыре прожектора поместили в~произвольных точках плоскости.
Докажите, что прожекторы можно повернуть так, что они осветят всю плоскость.

\item
Квадратный каток надо осветить четырьмя прожекторами, висящими на~одной высоте.
Каков наименьший радиус освещённых кругов?

\item
Коридор полностью покрыт несколькими ковровыми дорожками.
Докажите, что можно убрать несколько дорожек так, чтобы
\\
\sbp
коридор был полностью покрыт, а~общая длина оставшихся дорожек была не~больше
удвоенной длины коридора;
\\
\sbp
оставшиеся дорожки не~перекрывались и~их~суммарная длина была не~меньше
половины длины коридора.

\item
На~столе лежат 15~журналов, полностью покрывая его.
Докажите, что можно убрать 7~журналов так, чтобы оставшиеся покрывали не~менее
$8 / 15$ площади стола.

\item
Пусть $A$~--- наибольшее число попарно непересекающихся кругов диаметра~1,
центры которых лежат внутри многоугольника $M$,
$B$~--- наименьшее число кругов диаметра~2, которыми можно покрыть
многоугольник.
Что больше: $A$ или $B$\,?
(Перечислите все возможности.)

\end{problems}

