% $date: 2014-06-19

% $timetable:
%   g1011:
%     2014-06-19:
%     - 2

\section*{Немного об играх}

% $authors:
% - Сергей Беляков

\begin{problems}

\item
Петя и Вася играют в следующую игру.
Они по очереди ставят крестики в свободные клетки доски.
Петя ходит первый.
Петя всегда ходит первый.
Проигрывает тот, кто не может сделать ход.
Кто выиграет при правильной игре, если игра происходит на доске
\quad
\sbp $3 \times 5$;
\quad
\sbp $N \times M$?

\item
Петя и Вася решили поиграть в другую игру.
Теперь они по очереди кладут одинаковые круглые монеты на круглый стол на
свободное место.
Проигрывает тот, кто не может сделать ход.
Кто выигрывает при правильной игре?

\item
Очередная игра.
На этот раз Петя и Вася положили на стол две стопки монет:
в одной $m$, в другой $n$ штук.
В свой ход они либо берут произвольное количество монет из любой стопки, либо
одинаковое количество монет из обеих стопок.
Забравший последнюю монету побеждает.
Кто выигрывает в зависимости от размеров стопок?

\item
Петя и Вася по очереди ставят цифры от $1$ до $9$ в квадрат $9 \times 9$.
В клетку разрешается ставить цифру, если она еще не встречается ни в клетках
того же столбца, ни в клетках той же строки.
Проигрывает тот, кто не может сделать ход.
Кто выигрывает при правильной игре?

\item
Петя и Вася по очереди закрашивают по две клетки на полоске $1 \times 2010$.
Петя хочет, чтобы расстояния между двумя отмеченными им за один ход клетками не
повторялись.
Сможет ли Вася ему помешать?
(Первым ходит Петя; игра заканчивается, когда все клетки полоски закрашены).

\item
Два игрока, Петя и Вася, играют в следующую игру.
Перед игроками лежит куча камней. 
Игроки ходят по очереди, первый ход делает Петя.
За один ход игрок может добавить в кучу один камень или увеличить количество
камней в куче в два раза.
Например, имея кучу из $15$ камней, за один ход можно получить кучу из $16$ или
$30$ камней.
У каждого игрока, чтобы делать ходы, есть неограниченное количество камней.
Игра завершается в тот момент, когда количество камней в куче становится не
менее $22$.
Победителем считается игрок, сделавший последний ход, то есть первым получивший
кучу, в которой будет 22 или больше камней.
Кто выигрывает при правильной игре?
(Вычислите ответ при каждом начальном количестве камней.)

\item
Дана куча из $n$ спичек.
Два игрока играют в следующую игру: каждый из них своим ходом может взять из
кучи $4^k$ спичек, где $k \geq 0$.
Проигрывает не имеющий хода.
При каких $n$ игрок, делающий первый ход, может обеспечить себе победу
независимо от игры второго?

\end{problems}

