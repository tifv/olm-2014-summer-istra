% $date: 2014-06-24

% $timetable:
%   g78r1:
%     2014-06-24:
%     - 3
%     2014-06-25:
%     - 1

\section*{Алгоритмы и информационные оценки}

% $authors:
% - Лев Шабанов

\begin{problems}

\item
\sbp
Дана колода из $52$ карт.
Ассистент загадал карту.
Фокусник может разложить все карты на $4$ стопки и спросить, в какой стопке
находится карта.
Сколько вопросов нужно задать фокуснику, чтобы наверняка угадать загаданную
карту?
\\
\sbp
А если загадано две карты и ассистент отвечает на вопрос, в каких кучках лежат
загаданные карты?

\item
На плоскости расположен квадрат, и невидимыми чернилами нанесена точка $P$.
Человек в специальных очках видит точку.
Если провести прямую, то~он отвечает на вопрос, по какую сторону от неё лежит
$P$ (если $P$ лежит на прямой, то он говорит, что $P$ лежит на прямой). 
Какое наименьшее число таких вопросов необходимо задать, чтобы узнать, лежит ли
точка $P$ внутри квадрата?

\item
\sbp
Петя задумал три натуральных числа $a$, $b$, $c$, не~превосходящих $100$.
За~один вопрос Вася может узнать у~Пети значение выражения $a x + b y + c z$
для любых натуральных чисел $x$, $y$, $z$.
За~какое наименьшее число вопросов Вася может узнать задуманные числа?
\\
\sbp
А~если Петя задумал произвольные натуральные числа
$a_1$, $a_2$,~$\ldots$, $a_n$, а~Вася узнает значение
$a_1 b_1 + a_2 b_2 + \ldots + a_n b_n$ для любых натуральных
$b_1$, $b_2$,~$\ldots$, $b_n$?

\item
\sbp
В гостиницу приехал путешественник.
У него вместо денег нашлась лишь серебряная цепочка из 7 звеньев.
Хозяин требует платить по одному звену в~день без задержек, готов давать сдачу
ранее полученными кусками цепочки, но плату вперед брать отказывается.
Какое наименьшее число звеньев придется распилить, чтобы можно было
расплачиваться все 7 дней?
\\
\sbp\label{problem:combi/information:23}%
А если цепочка из 23 звеньев и 23 дня?
\\
\sbp\label{problem:combi/information:79}%
А если цепочка замкнутая, состоит из 79 звеньев и нужно расплачиваться 79 дней?
\\
\sbp
Решите пункты
\ref{problem:combi/information:23}, \ref{problem:combi/information:79}
для случая $n$ звеньев и $n$ дней.

\item
\sbp
Есть $4$ золотые и $2$ серебряные монеты, среди которых по одной фальшивой
монете, которые легче настоящей
(фальшивые монеты весят одинаково, настоящие тоже).
За сколько взвешиваний  на чашечных весах можно наверняка найти обе фальшивые
монеты?
\\
\sbp
А если золотых монет $13$ (а серебряных $2$)?
\\
\sbp
А если золотых монет $n$?

\end{problems}

