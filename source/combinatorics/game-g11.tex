% $date: 2014-06-15

% $timetable:
%   g11: 
%     2014-06-15:
%     - 2
%     2014-06-16:
%     - 1

\section*{Игры}

% $authors:
% - Антон Гусев

\begin{problems}

\item
На~столе лежат 2014 монет.
За~ход разрешается объединить две одинаковые кучки в~одну.
Проигрывает тот, кто не~может сделать ход.
Кто выиграет при правильной игре?

\item
Каждой вершине куба поставлено в~соответствие некое неотрицательное
действительное число, причем сумма всех этих чисел равна~1.
Двое играют в~следующую игру.
Первый выбирает любую грань куба, второй выбирает другую грань, и, наконец,
первый выбирает третью грань куба.
При этом выбирать грани, параллельные уже выбранным, нельзя.
Докажите, что первый игрок может играть так, чтобы число, соответствующее общей
вершине трех выбранных граней, не~превосходило $1 / 6$.

\item
Есть 9~коробок соответственно с~1, 2, \ldots, 9 фишками.
Двое по~очереди берут по~одной фишке из~любой коробки, при необходимости
открывая~её.
Проигрывает тот, кто последним распечатает коробку.
Кто из~них выиграет независимо от~игры соперника?

\item
Аня и~Боря по~очереди (начинает Аня) расставляют в~клетках таблицы $6 \times 6$
вещественные числа.
Ставить число, которое уже стоит в~какой-либо клетке, нельзя.
После того, как вся таблица заполнена, в~каждой строке закрашивают черным
клетку с~наибольшим числом.
Аня выигрывает, если можно провести ломаную, соединяющую верхнюю сторону
таблицы с~нижней и~лежащую целиком в~черных клетках.
В~противном случае выигрывает Боря.
Кто выиграет при правильной игре?

\item\emph{Король-самоубийца.}
На~шахматной доске размером $1000 \times 1000$ стоит черный король
и~$499$ белых ладей.
Докажите, что при произвольном начальном расположении фигур король может встать
под удар белой ладьи, как~бы не~играли белые.
(Ходы делаются так~же, как и~в~обычных шахматах).

\item
Два пирата делят добычу, состоящую из~двух мешков монет и~алмаза, действуя
по~следующим правилам.
Вначале первый пират забирает себе из~любого мешка несколько монет
и~перекладывает из~этого мешка в~другой такое~же количество монет.
Затем также поступает второй (выбирая мешок, из~которого он~берет монеты,
по~своему усмотрению), и~т.~д. до~тех пор, пока можно брать монеты по~такому
правилу.
Пирату, взявшему монеты последним, достается алмаз.
Кому достанется алмаз, если каждый из~пиратов стремится получить его?
Дайте ответ в~зависимости от~первоначального количества монет в~мешках.

\item
Написано 20 чисел: 1, 2, \ldots, 20.
Двое играющих по~очереди ставят перед этими числами знаки <<$+$>> и~<<$-$>>
(знак можно поставить перед любым свободным числом).
Первый стремится к~тому, что~бы полученная после расстановки всех 20 знаков
сумма была как можно меньше по~модулю.
Какую наибольшую по~модулю сумму может себе обеспечить второй?

\item
Написан многочлен $x^{10} + * x^9 + * x^8 + \ldots + * x + 1$.
Двое играют в~такую игру.
Сначала первый заменяет любую из~звездочек некоторым числом, затем второй
заменяет одну из~оставшихся звездочек, затем снова первый и~т.~д.
(всего 9 ходов).
Если у~полученного многочлена не~будет действительных корней, то~выигрывает
первый игрок, а~если хотя~бы 1 корень будет~--- второй.
Кто выиграет при правильной игре?

\item
В~микросхеме 2000 контактов.
Изначально любые два контакта соединены отдельным проводом.
Хулиганы Вася и~Петя по~очереди перерезают провода.
Причем Петя (он~начинает) за~ход режет один провод, а~Вася~--- либо один, либо
три провода.
Хулиган, отрезающий последний провод от~какого-либо контакта, проигрывает.
Кто из~них выиграет при правильной игре?

\end{problems}

