% $date: 2014-06-17

% $timetable:
%   g78r2:
%     2014-06-17:
%     - 1

\section*{Алгоритмические задачи. Разное}

% $authors:
% - Глеб Погудин

\begin{problems}

\item
Обезьяна Анфиса живет в десятиэтажном доме и у неё есть два кокосовых ореха.
Она хочет узнать, какой самый большой номер этажа, с которого можно скинуть
орех, но он бы не разбился.
Она может кидать орехи с разных этажей.
Если орех разбился, его больше нельзя использовать.
Как ей узнать ответ на свой вопрос, сделав четыре броска?

\item
Дома у мальчика Пети в двадцати коробках лежат гвозди, причем во всех коробках
число гвоздей разное.
Пете можно задавать вопросы вида
<<В какой коробке больше гвоздей: в коробке №$i$ или в коробке №$j$?>>.
За какое наименьшее число вопросов можно гарантированно узнать, в какой коробке
гвоздей больше всего?

\item
Злобная Белоснежка захватила в плен трех гномов.
Каждое утро происходит следующее: Белоснежка надевает на них колпаки красного
или синего цвета (каждый гном видит чужие колпаки, но не видит своего).
Потом они одновременно называют предполагаемый цвет своего колпака.
Если хотя бы один ошибается, то все идут работать на рудники.
Как им договориться заранее так, чтобы не идти работать в половине случаев?

\end{problems}

