% $date: 2014-06-19

% $timetable:
%   g11: 
%     2014-06-19:
%     - 1

\section*{Инварианты}

% $authors:
% - Антон Гусев

\begin{problems}

\item 
\sbp
В вершине $A_1$ правильного $12$-угольника $A_1 A_2 \ldots A_{12}$ стоит знак
минус, а в остальных~--- плюсы.
Разрешается одновременно изменять знак в любых шести последовательных вершинах
многоугольника.
Докажите, что за несколько таких операций нельзя добиться того, чтобы в~вершине
$A_2$ оказался знак минус, а в остальных вершинах~--- плюсы.
\\
\sbp
Докажите то же утверждение, если разрешается менять знак одновременно в трех
последовательных вершинах многоугольника.

\item
С многоугольником разрешается проделывать следующие операции.
Можно отрезать от него прямой, которая пересекает его стороны в точках $A$ и
$B$ соответственно, некий кусок, перевернуть его и приложить обратно тем же
отрезком $AB$.
Можно ли такими операциями получить 
\\
\sbp
из правильного шестиугольника со стороной 1 правильный треугольник со стороной
$\sqrt{6}$;
\\
\sbp
из квадрата хоть какой-нибудь треугольник.

\item
В вершинах правильного $n$-угольника с центром в точке $O$ расставлены числа
$+1$ и $-1$.
За один шаг разрешается изменить знак у всех чисел, стоящих в вершинах
какого-либо правильного $k$-угольника с центром в точке $O$.
При этом мы допускаем и $2$-угольники.
Докажите, что существует такое первоначальное расположение $+1$ и $-1$, что из
него ни за какое число шагов нельзя получить набор из одних $+1$.
\\
\sbp $n = 15$;
\qquad
\sbp $n = 30$;
\qquad
\sbp $n$~--- любое число, большее 2.

\item
По кругу центрально-симметрично сидят 10 кузнечиков.
За ход какой-то кузнечик перепрыгивает через своего соседа так, что бы этот
сосед был серединой между начальной и конечной точкой кузнечика.
Причем кузнечику разрешается так прыгнуть, только если за свой прыжок он
пролетит только над одним кузнечиком (над тем соседом, через которого прыгает).
Может ли так получиться, что через некоторое количество ходов 9 кузнечиков
оказались на своих местах, а десятый в точке на той же дуге (между теми же
кузнечиками, что изначально), но не на своем месте?

\end{problems}

