% $date: 2014-06-20

% $timetable:
%   g78r2:
%     2014-06-20:
%     - 2

\section*{Логические задачи}

% $authors:
% - Глеб Погудин

\begin{problems}

% spell "правдивец" -> "рыцарь"

\item
$A$ говорит <<Я лжец или $B$ правдивец>>.
Кто из $A$ и $B$ лжец, а кто правдивец?

\item
Есть три человека $A$, $B$ и $C$.
$A$ говорит <<Мы все лжецы>>, $B$ говорит <<Ровно один из нас правдивец>>.
Кто из них кто?

\item
$A$: <<Мы все лжецы>>.
$B$: <<Ровно один из нас лжец>>.
Может ли $B$ или $C$ быть определен однозначно?

\item
$A$: <<Я лжец, а $B$~--- нет>>.
Кем являются $A$ и $B$?

\item
Одному из двух аборигенов задали вопрос <<Есть ли среди вас правдивец?>>.
Его ответа оказалось достаточно, чтобы узнать правдивый ответ на этот вопрос.
Какой же это ответ и кем был ответивший?

\item
Пусть $X$ и $Y$~--- какие-то утверждения.
Обозначим через $X \oplus Y$ утверждение <<Из $X$ и $Y$ ровно одно утверждение верно>>.
Докажите:
\begin{enumerate}
\item $X \oplus \text{<<$2 + 2 = 5$>>}$ верно только если верно $X$;
\item $X \oplus \text{<<$2 + 2 = 4$>>}$ верно только если неверно $X$;
\item Пусть среди $X$, $Y$ и $Z$ верны какие-то два. Верно ли $X \oplus Y \oplus Z$?
\end{enumerate}

\item
Вы встретили аборигена, но не знаете лжец он или правдивец.
Как за один вопрос выяснить у него, есть ли у него дома ручной крокодил?

\item
Все так же, как и в предыдущей задаче, но абориген понимает русский, а отвечает на
своем языке.
В нем есть слова <<ага>> и <<угу>>, которые означают да и нет, но вы не знаете, какое
что означает.
Как за один вопрос выяснить наличие крокодила?

\item
Все так же, как и в прошлой задаче, но аборигена три.
Один лжец, один правдивец и один нормальный.
Как за три вопроса узнать, кто какой?
(Нормальный отвечает как ему заблагорассудится.)

\end{problems}

