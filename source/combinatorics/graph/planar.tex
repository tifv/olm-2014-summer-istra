% $date: 2014-06-20

% $timetable:
%   g1011:
%     2014-06-20:
%     - 2
%     2014-06-21:
%     - 1

\section*{Планарные графы}

% $authors:
% - Антон Гусев

\begin{problems}

% spell "\text{В}" -> "\text{Вершины}"
% spell "\text{Р}" -> "\text{Ребра}"
% spell "\text{Г}" -> "\text{Грани}"

\item
На плоскости изображен связный граф с $\text{В}$ вершинами и
$\text{Р}$ ребрами, разбивающий плоскость на $\text{Г}$ граней.
Докажите формулу Эйлера $\text{В} - \text{Р} + \text{Г} = 2$.

\item
\sbp
От каждого из трех домов ведет тропинка к каждому из трех колодцев.
Докажите, что какие-то две тропинки пересекаются.
\\
\sbp
В лагере <<Команда>> школа, столовая, третий корпус, пятый корпус и
футбольное поле попарно соединены тропинками.
Докажите, что какие-то две тропинки пересекаются
\\
\sbp\label{problem:draw-peterson}%
Можно ли на плоскости изобразить граф Петерсена
%(рис.\,\ref{fig:petersen-graph})
без самопересечений?
%
%\begin{figure}[h]
\begin{center}
\jeolmfigure[width=0.3\textwidth]{petersen}
%\caption{Граф Петерсена. К задаче \ref{problem:draw-peterson}.}
%\label{fig:petersen-graph}%
\end{center}
%\end{figure}

\item
\sbp
На контурном глобусе нарисовано несколько стран.
Вася хочет раскрасить их в три цвета так, чтобы страны одного цвета не
граничили.
Докажите, что трех цветов может не хватить.
\\
\sbp
Докажите, что хотя бы одна из граней многогранника имеет не более 5 сторон.
\\
\sbp
Докажите, что 5 цветов Васе точно хватит.

\item
В квадрате отметили $100$ точек и соединили их между собой и с вершинами
квадрата так, что он оказался разбит на треугольнички.
\\
\sbp
Сколько было треугольничков?
\\
\sbp
Докажите, что в одном из них все углы не превосходят $120^\circ$.

\item
Докажите, что грани плоского графа можно раскрасить в два цвета так, что бы
никакие две грани одного цвета не граничили в том и только том случае, когда
степень каждой вершины этого графа четна.

\end{problems}

