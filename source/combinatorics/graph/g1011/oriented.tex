% $date: 2014-06-17

% $timetable:
%   g1011:
%     2014-06-17:
%     - 2

\section*{Ориентированные графы}

% $authors:
% - Антон Гусев

\begin{problems}

\item
В~1996 году каждый из президентов 15 бывших республик Советского Союза послал
в~подарок на~день рождения каждому из~остальных торт с~таким количеством
свечек, сколько лет исполнилось поздравляемому.
Могло~ли так случиться, что всего было послано 1997 свечек?

\item
В~Однобоком государстве между некоторыми (но, к~сожалению, пока не~между всеми)
усадьбами проложены дороги с~односторонним движением.
При~этом при появлении любой новой дороги (также с~односторонним движением)
между усадьбами, не~соединенными дорогой до~этого, появится возможность
добраться от~любой усадьбы до~любой другой, не~нарушая правил.
Докажите, что такая возможность имеется уже сейчас.

\item
В~стране каждые два города соединены дорогой с~односторонним движением.
Докажите, что есть город,
\\
\sbp
из~которого можно доехать до~любого другого;
\\
\sbp
из~которого можно доехать до~любого другого, заезжая не~более чем в~один
промежуточный город.

\item
В~стране любые два города соединены дорогой с~односторонним движением.
\\
\sbp
Докажите, что можно проехать по~всем городам, побывав в~каждом городе
по~1~разу.
\\
\sbp
Известно, что, кроме того, из~любого города можно добраться до~любого другого.
Докажите, что можно проехать по~всем городам, побывав в~каждом городе ровно
по~одному разу, и~вернуться в~город, из~которого было начато путешествие.

\item
В~государстве $n$ городов, каждые два из~которых соединены дорогой.
Правительство хочет ввести на~дорогах одностороннее движение, чтобы, выехав
из~любого города, в~него больше нельзя было вернуться.
\\
\sbp
Докажите, что это можно сделать.
\\
\sbp
Найдите количество способов ввести одностороннее движение таким образом.

\item
В~полном ориентированном графе есть $n$ вершин и~цикл, проходящий через все
вершины ровно по~1~разу.
Может~ли оказаться так, что после удаления одного из~ребер граф останется
связным, но~уже не~будет содержать цикла, проходящего по~всем вершинам ровно
по~одному разу?

\item
В~стране 101 город.
Некоторые пары городов соединены дорогами, причем никакие два города
не~соединены более, чем одной дорогой.
На~каждой дороге введено одностороннее движение.
Оказалось, что в~каждый город можно въехать по~40 дорогам, и~из~каждого города
можно выехать по~40 дорогам.
Докажите, что из~любого города можно доехать в~любой другой, заезжая по~пути
не~более чем в~два промежуточных города.

\item
В~турнире по~настольному теннису участвовали школьники и~преподаватели, причем
школьников было в~два раза больше, чем преподавателей.
Турнир проходил в~один круг.
Количество встреч, выигранных преподавателями, на~40 процентов больше, чем
количество встреч, выигранных школьниками.
Сколько людей участвовало в~турнире?

\end{problems}

