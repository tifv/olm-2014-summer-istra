% $date: 2014-06-16

% $timetable:
%   g1011:
%     2014-06-16:
%     - 2

\section*{Графы. Введение}

% $authors:
% - Антон Гусев

\begin{problems}

\item
Волейбольная сетка имеет вид прямоугольника $50 \times 600$ клеток.
Какое максимальное количество веревок можно разрезать, что бы сетка не
распалась на части?

\item
В стране $2014$ городов, и из каждого выходит не менее, чем $93$ дороги.
Известно, что из каждого города можно добраться до любого другого.
Докажите, что это можно сделать с не более, чем $63$ пересадками.

\item
На какое наименьшее число частей нужно разрезать проволоку длиной
$12$~сантиметров, что бы из нее можно было сложить каркас кубика со стороной
$1$~сантиметр?

\item
В королевстве 16 городов.
Король хочет соединить их дорогами так, что бы из каждого города выходило не
более 4 дорог и из любого можно было добраться до любого другого, сделав не
более 1 пересадки.
Сможет ли он это сделать?

\item
В стране несколько городов, некоторые пары которых соединены дорогами, причем
между любыми двумя городами существует единственный несамопересекающийся путь.
Также известно, что ровно из 100 городов выходит по 1 дороге.
Докажите, что в стране можно построить не более 50 дорог так, что при закрытии
на ремонт любой дороги, по прежнему можно будет добраться из каждого города в
каждый.

\item
Докажите, что хотя бы одна из граней многогранника имеет не более 5 сторон.

\item
На берегу круглого озера расположены населенные пункты.
Между некоторыми из них курсируют теплоходы.
Причем известно, что из $A$ в $B$ и обратно курсируют теплоходы тогда и только
тогда, когда между следующими за ними по кругу $A'$ и $B'$ теплоходы не
курсируют.
Докажите, что из любого населенного пункта можно добраться в любой другой,
сделав не более 2 пересадок.

\end{problems}

