% $date: 2014-06-26

% $timetable:
%   g10:
%     2014-06-26:
%     - 1
%     2014-06-27:
%     - 1

\section*{Раскраски графов, часть 2}

% $authors:
% - Антон Гусев

\emph{Красим ребра!}

\definition
\emph{Раскраска ребер} графа в~$k$~цветов~---
это разбиение множества его ребер на~$k$~подмножеств:
$C = (E_1, E_2, \ldots, E_k)$.
Мы будем говорить, что в~раскраске~$C$
в~вершине~$v$ \emph{представлен} цвет~$i$,
если хотя~бы одно из~инцидентных~$v$ ребер раскрашено в~цвет~$i$.
(Если таких ребер~$t$, то~можно сказать, что
цвет~$i$ представлен в~вершине~$v$ \emph{ровно $t$~раз}).

\definition
Раскраска ребер графа~$G$ называется
\emph{правильной},
если любые два ребра, имеющих общий конец, покрашены в~разные цвета.

\definition
\emph{Реберное хроматическое число графа $\chi'(G)$}~---
это минимальное количество цветов, для которого существует правильная раскраска
ребер графа~$G$.

\begin{problems}

\item
Ребра полного графа с~25~вершинами покрашены в~24~цвета.
Докажите, что найдется вершина, из~которой выходит 2~ребра одного цвета.

\item
Верно~ли, что ребра графа степени~3 всегда можно правильно раскрасить
в~3~цвета?

\item
\sbp
Докажите, что в~любой компании из~6~человек найдутся либо трое попарно
знакомых, либо трое попарно незнакомых.
\\
\sbp
Докажите, что в~любой компании из~9 человек найдутся либо четверо попарно
знакомых, либо трое попарно незнакомых.

\item
Пусть $G$~--- связный граф, отличный от~цикла нечетной длины.
Докажите, что ребра графа~$G$ можно так раскрасить в~два цвета, чтобы в~каждой
вершине были представлены ребра обоих цветов.

\end{problems}

\definition
Пусть $C$~--- раскраска графа~$G$.
Для каждой вершины~$v$ обозначим через~$c(v)$ количество цветов, в~которое
раскрашены ребра, инцидентные~$v$.

\definition
Будем говорить, что $C$~--- \emph{оптимальная} раскраска графа~$G$ в~$k$
цветов, если для любой другой раскраски $C'$ выполнено
\(
    \sum_{v \in V}
        c(v)
\geq
    \sum_{v \in V}
        c'(v)
\).

\begin{problems}

\item
Пусть $C = (E_1, \ldots, E_k)$~--- оптимальная раскраска графа~$G$
в~$k$~цветов.
Пусть вершина~$v$ и~цвета $i$ и~$j$ таковы, что в~вершине~$v$ дважды
представлен цвет~$i$ и~не~представлен цвет~$j$.
Докажите, что компонента связности графа $G[E_i \cup E_j]$, содержащая
вершину~$v$~--- простой цикл нечетной длины.

\item
Пусть $G$~--- двудольный граф.
Докажите, что $\chi'(G) = \Delta(G)$

\item
Пусть $G$~--- двудольный граф, $\delta(G) = d$.
Докажите, что существует правильная раскраска ребер графа~$G$ в~$d$~цветов,
в~которой в~каждой вершине представлены все $d$~цветов.

% spell "Вадим Визинг" -> ""

\item\emph{(Вадим Визинг, 1964)}
Докажите, что $\chi'(G) \leq \Delta(G) + 1$.

\end{problems}

