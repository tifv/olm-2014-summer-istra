% $date: 2014-06-24

% $timetable:
%   g10:
%     2014-06-24:
%     - 1

\section*{Раскраски графов, часть 1}

% $authors:
% - Антон Гусев

\begin{problems}

\item
\sbp
Вершины графа можно разбить на~две группы так, чтобы все ребра соединяли
вершины из~разных групп.
Докажите, что все циклы в~этом графе имеют четную длину.
\\
\sbp
В~графе любой цикл имеет четную длину.
Докажите, что его вершины можно разбить на~две группы так, чтобы все ребра
соединяли вершины из~разных групп.

\item
Докажите, что можно выбрать не~более половины вершин связного графа так, чтобы
каждая из~оставшихся вершин была соединена ребром с~одной из~выбранных.

\item
В~графе 2005 вершин, и~степени всех вершин равны.
Вершины графа раскрашены в~красный, синий и~зеленый цвет так, что концы любого
ребра~--- разноцветные.
Докажите, что найдется красная вершина, которая соединена и~с~синей,
и~с~зеленой.

\end{problems}

\definition
Пусть $G$~--- связный граф.
Будем обозначать через $\Delta (G)$ и~$\delta (G)$ максимальную и~минимальную
степень вершин графа~$G$ соответственно.

\begin{problems}

\item
Пусть $G$~--- связный граф, $\Delta(G) = d$.
Докажите, что граф~$G$ можно правильным образом раскрасить в~$d$~цветов, если
\\
\sbp
есть вершина степени меньше~$d$;
\\
\sbp
есть вершина, при удалении которой граф теряет связность;
\\
\sbp
$d > 2$ и~есть две вершины такие, что при удалении их~обеих граф теряет
связность;
\\
\sbp
Есть три вершины $u$, $v$, $w$ такие, что $u$ смежна с~$v$ и~$w$,
$v$ и~$w$ не~смежны между собой, и~при удалении вершин $v$ и~$w$ связность
не~нарушается.

% spell "Brooks" -> ""

\item\emph{(Brooks, 1941)}
Пусть $G$~--- связный граф, $\Delta(G) = d \geq 3$, отличный от~полного графа
$K_{d+1}$.
Докажите, что вершины графа $G$ можно правильным образом раскрасить
в~$d$~цветов.

\end{problems}

\definition
Обозначим через $\chi_G(k)$ количество правильных раскрасок $G$ в~$k$~цветов.
Хроматическое число графа $\chi(G)$~--- это наименьшее натуральное $k$ такое,
что $\chi_G(k) \neq 0$.

\begin{problems}

\item
\sbp
Докажите, что $\chi_G(k)$~--- многочлен от~$k$.
\\
\sbp
Найдите хроматический многочлен дерева на~$n$~вершинах.
\\
\sbp
Докажите, что любой граф с~таким хроматическим числом~--- дерево
на~$n$~вершинах.

\item
Докажите, что из~графа~$G$ можно удалить не~более $1 / n$ часть его ребер так,
чтобы полученный граф имел правильную раскраску в~$n$~цветов.

\end{problems}

