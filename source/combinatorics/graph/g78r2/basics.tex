% $date: 2014-06-22

% $timetable:
%   g78r2:
%     2014-06-22:
%     - 1

\section*{Графы, часть 1. Общие соображения}

% $authors:
% - Лев Шабанов


\subsection*{Определения}

\begin{itemize}

\item
Будем говорить, что задан \emph{граф}, если дано множество его \emph{вершин}, и
про каждую пару вершин известно, соединены ли они \emph{ребром}.

\item
\emph{Степенью вершины} называется количество исходящих из нее ребер.
Вершина, степень которой равна 1, называется \emph{висячей}.

\item
Граф называется \emph{связным}, если любые две вершины соединены \emph{путем}
по ребрам.

\item
\emph{Деревом} называется связный граф без \emph{циклов}.

\item
Граф называется \emph{полным}, если любые две его вершины соединены ребром.

\end{itemize}


\subsection*{Задачи}
  
\begin{problems}

\item
Сколько ребер в полном графе на $n$ вершинах?

\item
Можно ли на плоскости нарисовать $15$ отрезков так, чтобы каждый пересекал
ровно 7 других?

\item\emph{Лемма о рукопожатиях.}
Сумма степеней вершин графа~--- четное число.

\item
Докажите, что в любом дереве есть висячая вершина.

\item
Сколько ребер в дереве на $n$ вершинах?

\item
Докажите, что:
\\
\sbp
в дереве любые две вершины соединены ровно одним путем;
\\
\sbp
если в графе любые две вершины соединены ровно одним путем, то граф~--- дерево.

\item
У царя Гвидона было 7 сыновей.
Среди его потомков ровно 50 имели по 3 сына, ровно 100 по одному сыну, а у
остальных сыновей не было.
Сколько всего потомков было у царя Гвидона?

\item\emph{Остовное дерево.}
Докажите, что в любом связном графе можно удалить нес\-коль\-ко ребер так, чтобы
полученный граф стал деревом.

\item
В стране 101 город и из каждого города ведет не менее 50 дорог.
Докажите, что из любого города можно добраться в любой другой
(возможно с пересадками).

\item
В области 9 поселков и 29 дорог.
Докажите, что можно добраться по дорогам из любого поселка в любой другой.

\item
В лагере у каждого пионера 20 друзей.
Как только пионер узнает новость, он тут же сообщает ее своим друзьям.
За завтраком один из пионеров узнал новость, и к обеду ее знал весь лагерь.
За ужином двое пионеров поссорились.
На следующий день за завтраком пионеру (не обязательно тому же) сообщают
новость.
Докажите, что к обеду о ней опять узнает весь лагерь.

\item
В стране любые два города соединены либо авиалинией, либо железной дорогой.
Министерство транспорта в рамках программы экономии хочет закрыть один из этих
видов перевозок.
\\
\sbp
Докажите, что оно может это сделать так, чтобы из любого города можно было
добраться в любой другой.
\\
\sbp
Каким числом пересадок можно гарантированно обойтись после этого?

\end{problems}

