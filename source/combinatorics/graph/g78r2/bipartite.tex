% $date: 2014-06-23

% $timetable:
%   g78r2:
%     2014-06-23:
%     - 2

\section*{Графы, часть 2. Двудольные}

% $authors:
% - Лев Шабанов

\definition
Граф называется \emph{двудольным}, если его вершины можно разбить на два
множества так, что никакие две вершины из одного множества не соединены ребром.

\begin{problems}

\item
В классе каждый мальчик дружит с пятью девочками, а каждая девочка с семью
мальчиками.
В классе 17 парт (за каждой сидит не больше двоих) и 13 заядлых туристов.
Сколько всего человек в классе?

\item
Докажите, что в двудольном графе суммы степеней вершин каждого цвета равны
между собой.

\item
Какое максимальное количество ребер в двудольном графе:
\\
\sbp На $b$ черных и $w$ белых вершинах?
\\
\sbp На $2 n$ вершинах?
\qquad
\sbp На $2 n + 1$ вершинах?

\item
В ряд выписано 17 натуральных чисел.
Посчитали все суммы подряд идущих чисел (в том числе и суммы из одного числа).
Какое наибольшее нечетных чисел могло оказаться среди данных сумм?

% spell "\text{м}" -> "\text{метров}"

\item
Замок в форме треугольника со стороной $100\,\text{м}$ разбит на $100$
треугольных залов со сторонами $1\,\text{м}$.
В каждой стенке между залами есть дверь.
Какое наибольшее число залов сможет обойти турист, не заходя ни в какой зал
дважды?

\item
\sbp
В квадрате $8 \times 8$ расставлены числа от $1$ до $64$
(каждое по одному разу).
Можно узнавать сумму чисел в любых $2$ соседних клетках.
Можно ли понять, в какой клетке какое число?
\\
\sbp
А если можно узнавать сумму чисел в соседних по вершине клетках?
\\
\sbp
А если квадрат $9 \times 9$, и числа от $1$ до $81$ (оба варианта)?

\item
\sbp Докажите, что в двудольном графе нет циклов нечетной длины.
\\
\sbp Докажите, что дерево~--- двудольный граф.
\\
\sbp Докажите, что граф без нечетных циклов двудольный.

\item
В математическом классе 20 мальчиков, и некоторые враждуют, причем двое,
имеющих общего врага не враждуют между собой.
Какое наибольшее количество враждующих пар может быть?

\item
В классе каждый мальчик если идет в театр, то берет с собой всех знакомых
девочек.
Оказалось, что для любой компании мальчиков вместе с ними пойдет хотя бы
столько же девочек.
Докажите, что можно поставить в пару каждому мальчику знакомую девочку так, что
ни одна девочка не попадет в две пары.

\end{problems}

