% $date: 2014-06-24

% $timetable:
%   g78r2:
%     2014-06-24:
%     - 2

\section*{Графы, часть 3. Ориентированные}

% $authors:
% - Лев Шабанов

\subsection*{Определения}

\begin{itemize}

\item
Граф называется \emph{ориентированным}, если для каждого ребра задано
направление, в~котором идет обход ребра.

\item
Ориентированный граф называется \emph{сильно связным} если из~любой вершины
можно дойти по~ребрам до~любой другой.

\item
Ориентированный граф называется \emph{полным (турнирным)}, если любые две его
вершины соединены ребром.

\item
Ориентированный граф называется \emph{односторонне связным}, если для любых
двух вершин $a$ и~$b$ можно добраться либо из~$a$ в~$b$, либо из~$b$ в~$a$.

\item
Ориентированный граф называется \emph{слабо связным}, если из~любой вершины
можно добраться по~ребрам до~любой другой, возможно проходя против стрелок.

\end{itemize}

\subsection*{Задачи}

\begin{problems}

\item
Докажите, что в~любом ориентированном графе сумма степеней исходящих ребер
равна сумме степеней входящих ребер.

\item
Докажите, что любой сильно связный граф является односторонне связным,
а~любой односторонне связный~--- слабо связным.

\item
В~некоторый момент однокругового шахматного турнира среди 100 участников все
участники, кроме Барона Мюнхгаузена выиграли 21 партию и~проиграли 22 партии.
Сколько очков набрал Барон Мюнхгаузен к~концу турнира?

\item
Докажите, что в~сильно связном связном турнирном графе есть две вершины,
у~которых равны как количество исходящих ребер, так и~количество входящих.

\item
\sbp
Докажите, что в~любом турнирном графе есть путь, проходящий по~разу по~всем
вершинам.
\\
\sbp
А~если граф сильно связный, то~в~нем есть цикл, проходящий по~разу по~всем
вершинам.

\item
Докажите, что в~односторонне связном ориентированном графе можно нарисовать
одно ребро так, чтобы он~стал сильно связным.
(Можно проводить ребро, соединяющее две уже соединенные вершины в~обратном
направлении.)

\end{problems}

