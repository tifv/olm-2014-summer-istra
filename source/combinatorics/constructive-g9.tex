% $date: 2014-06-20

% $timetable:
%   g9:
%     2014-06-20:
%     - 1

\section*{Примеры}

% $authors:
% - Иван Митрофанов

\begin{problems}

\item
В~связном графе $2n$~вершин, все вершины имеют степень~$3$.
Обязательно~ли вершины графа можно разбить на~$n$~пар смежных?

\item
Можно~ли найти $5$ различных натуральных чисел таких, что сумма любых
$4$ из~них~--- точный квадрат?

\item
Существует~ли граф с~$10$~вершинами такой, что все вершины имеют степень~$3$
и~что из~любой вершины можно добраться до~любой, пройдя не~более, чем по~двум
ребрам?

\item
Разрежьте правильный десятиугольник на~$10$~ромбов.

\item
Вася ставит на~плоскости точки: сначала первую, потом вторую и~так далее.
Может~ли в~каждый момент времени множество поставленных~им точек иметь ось
симметрии, если никакие три точки не~ставятся на~одну прямую?

\item
Можно~ли разрезать круг на~несколько равных частей так, чтобы хотя~бы одна
из~них не~содержала центр?
Граница части ей~принадлежит.

\item
Какое наибольшее число диагоналей отдельных клеточек можно провести внутри
клетчатого квадрата $5 \times 5$ так, чтобы диагонали попарно не~имели общих
точек (даже концов)?

\end{problems}

