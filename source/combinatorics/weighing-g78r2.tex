% $date: 2014-06-17

% $timetable:
%   g78r2:
%     2014-06-17:
%     - 1

\section*{Алгоритмические задачи. Взвешивания}

% $authors:
% - Глеб Погудин

% $build$matter[print]: [[.], [.]]
% $build$style[print]:
% - .[tiled4,-print]

\subsection*{На занятии}

\begin{problems}

\item\emph{(разбор сразу)}
Есть три монетки, среди которых одна фальшивая.
Известно, что фальшивая монетка легче настоящей.
Как за одно взвешивание определить фальшивую?

\item
То же самое, но монеток уже $9$, а взвешиваний два.

\item
То же самое, но список взвешиваний надо предъявить заранее.

\item
Какое минимальное число взвешиваний потребуется для определения легкой фальшивой из
$26$ штук?

\item
Из трех монет одна фальшивая, но неизвестно, легче она настоящей или тяжелее.
За какое наименьшее число взвешиваний можно понять, какая монета фальшивая, и как её
вес соотносится с весом настоящей?

\end{problems}

\subsection*{На дом}

\begin{problems}

\item
Среди восьми монет есть либо одна легкая фальшивая, либо все настоящие.
Как за два взвешивания определить фальшивую, если такая есть?

\item
Как за три взвешивания найти легкую фальшивую из $27$, если список взвешиваний нужно
предъявить заранее?

\item
Есть $27$ монет.
Часть из них серебряные, остальные~--- медные
(настоящая медная отличается по весу от настоящей фальшивой).
Известно, что среди них ровно одна легкая фальшивая.
Как найти её за три взвешивания?

\end{problems}

