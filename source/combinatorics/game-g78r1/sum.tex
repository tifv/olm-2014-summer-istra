% $date: 2014-06-18

% $timetable:
%   g78r1:
%     2014-06-18:
%     - 1

\section*{Математические игры, часть 2}

% $caption: Математические игры, часть 2 (Сумма)

% $authors:
% - Владимир Шарич

\subsection*{Симметрия, однако}

\begin{problems}

\item
Коля и~Витя играют в~следующую игру на~бесконечной клетчатой бумаге.
Начиная с~Коли, они по~очереди отмечают узлы клетчатой бумаги~--- точки
пересечения вертикальных и~горизонтальных прямых.
При этом каждый из~них своим ходом должен отметить такой узел, что после этого
все отмеченные узлы лежали в~вершинах выпуклого многоугольника
(начиная со~второго хода Коли).
Тот из~играющих, кто не~сможет сделать очередного хода, считается проигравшим.
Кто выигрывает при правильной игре? 

\item
Петя и~Вася играют на~доске размером $7 \times 7$.
Они по~очереди ставят в~клетки доски цифры от~1 до~7 так, чтобы ни~в~одной
строке и~ни~в~одном столбце не~оказалось одинаковых цифр.
Первым ходит Петя.
Проигрывает тот, кто не~сможет сделать ход.
Кто из~них сможет выиграть, как~бы ни играл соперник?
%http://problems.ru/view_problem_details_new.php?id=64694

\end{problems}

\subsection*{Игры с~суммой выигрыша}

\begin{problems}

\item
Двое играют в~следующую игру.
Каждый игрок по~очереди вычеркивает 9 чисел (по~своему выбору)
из~последовательности 1, 2, \ldots, 100, 101.
После одиннадцати таких вычеркиваний останутся 2~числа.
Первому игроку присуждается столько очков, какова разница между этими
оставшимися числами.
Докажите, что первый игрок всегда сможет набрать по~крайней мере 55 очков,
как~бы ни~играл второй.
Может~ли второй игрок помешать первому набрать более 55 очков?
%http://problems.ru/view_problem_details_new.php?id=34863

\item
Имеется шоколадка с~пятью продольными и~восемью поперечными углублениями,
по~которым её~можно ломать (всего получается $9 \times 6 = 54$ дольки).
Играют двое, ходят по~очереди.
Играющий за~свой ход отламывает от~шоколадки полоску ширины 1 и~съедает её.
Другой играющий за~свой ход делает то~же самое с~оставшейся частью, и~т.~д.
Тот, кто разламывает полоску ширины 2 на~две полоски ширины 1, съедает одну
из~них, а~другую съедает его партнер.
Докажите, что начинающий игру может действовать таким образом, что ему
достанется по~крайней мере на~6 долек больше, чем второму.
Может~ли второй игрок помешать первому съесть больше чем на~6 долек больше?

\item
По~кругу стоит 101 блюдце, на~каждом по~конфете.
Сначала Малыш выбирает натуральное $m < 101$ и~сообщает его Карлсону, затем
Карлсон~--- натуральное $k < 101$.
Малыш берет конфету с~любого блюдца.
Отсчитав от~этого блюдца $k$-е блюдце по~часовой стрелке, Карлсон берет с~него
конфету.
Отсчитав уже от~этого блюдца $m$-е блюдце по~часовой стрелке, Малыш берет
с~него конфету (если она там еще есть).
Отсчитав от~блюдца Малыша $k$-е блюдце по~часовой стрелке, Карлсон берет с~него
конфету (если она там еще есть), и~т.~д.
Какое наибольшее число конфет может гарантировать себе Карлсон?

\end{problems}

