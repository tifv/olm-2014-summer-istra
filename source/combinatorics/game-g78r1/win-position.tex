% $date: 2014-06-21

% $timetable:
%   g78r1:
%     2014-06-21:
%     - 2

\section*{Математические игры, часть 4}

% $caption: Математические игры, часть 4 (Позиции)

% $authors:
% - Владимир Шарич

\subsection*{Выигрышные и проигрышные позиции}

\emph{%
В каждой из следующих игр ходят по очереди два игрока.
Укажите, кто из них выиграет при правильной игре.}

\begin{problems}

\item
В~левом нижнем углу клетчатого прямоугольника $15 \times 22$ стоит ладья.
Маша и~Саша по~очереди передвигают её~на~любое количество клеток либо вправо,
либо вверх.
Первой ходит Маша.
Выигрывает та~девочка, которая поставит ладью в~правый верхний угол доски.

\item
Имеются две кучи камней, в~первой 14 камней, во~второй~--- 21.
Маша и~Саша по~очереди берут сколько угодно камней с~любой кучи, но только
с~одной.
Выигрывает та девочка, которая возьмет последний камень.

\item
Шахматный король стоит в~левом нижнем углу шахматной доски.
За~один ход короля можно передвинуть на~одно поле вправо, на~одно поле вверх
или на~одно поле по~диагонали <<вправо-вверх>>.
Игрок, который поставит короля в~правый верхний угол доски,
\quad
\sbp выигрывает;
\quad
\sbp проигрывает.
\quad

\item
Игра начинается с~числа 2.
За~ход разрешается прибавить к~имеющемуся числу любое натуральное число,
меньшее его.
Выигравшим считается тот, в~результате хода которого получится 2014.

\item
Двое по~очереди выписывают на~доску натуральные числа от~1 до~1000.
Первым ходом первый игрок выписывает на~доску число 1.
Затем очередным ходом на~доску можно выписать либо число $2a$, либо число
$a + 1$, если на~доске уже написано число $a$.
При этом запрещается выписывать числа, которые уже написаны на~доске.
Выигрывает тот, кто выпишет на~доску число 1000.

\item
Игра начинается с~числа 60.
За~ход разрешается уменьшить имеющееся число на~любой из~его делителей.
Проигрывает тот, кто получит ноль.

\item
Имеются три кучи камней, в~первой 4~камня, во~второй~--- 5, в третьей~--- 7.
Маша и~Саша по очереди берут сколько угодно камней с~любой кучи, но~только
с~одной.
Выигрывает та~девочка, которая возьмет последний камень.

\item
В~левом нижнем ближнем углу клетчатого параллелепипеда $5 \times 6 \times 8$
стоит ладья.
Маша и~Саша по~очереди передвигают~ее на~любое количество клеток либо вправо,
либо вверх, либо вглубь.
Первой ходит Маша.
Выигрывает та~девочка, которая поставит ладью в~правый верхний дальний угол
параллелепипеда.

\end{problems}

