% $date: 2014-06-22

% $timetable:
%   g78r1:
%     2014-06-22:
%     - 1

\section*{Математические игры, часть 5}

% $caption: Математические игры, часть 5 (Разнобой)

% $authors:
% - Владимир Шарич

\subsection*{Разнобой}

\begin{problems}

% spell "Светопром" -> ""
% spell "Нью-Васюки" -> "Москва"

\item
Две компании, <<Светопром>> и~<<ООО ЕЭС>>, получили право освещать столицу
международной шахматной мысли Нью-Васюки, представляющую собой прямоугольную
сетку улиц.
Они по~очереди ставят на~еще~не~освещенный перекресток прожектор, который
освещает весь юго-западный угол города
(т.~е. все перекрестки, которые расположены не~севернее и~не~восточнее).
Премию О.\,Бендера получит та~компания, которой на~своем ходе нечего будет
освещать.
Начинает <<Светопром>>.
Кто выиграет при правильной игре?

\item
На~бесконечном листе клетчатой бумаги двое по~очереди соединяют узлы соседних
клеток по~вертикали или горизонтали, один красным цветом, другой~--- синим.
Нельзя обводить один отрезок дважды.
Выигрывает тот, кто первым нарисует замкнутый контур своего цвета.
Существует~ли выигрышная стратегия
\quad
\sbp у~второго игрока;
\quad
\sbp у~первого игрока?

\item
По~кругу расставлены 8~точек.
Двое по~очереди соединяют~их~отрезками.
Первый отрезок проводится произвольно, а~каждый следующий начинается из~конца
предыдущего.
Проигрывает тот, кто не~может провести новый отрезок
(дважды проводить один отрезок нельзя).
Кто выиграет при правильной игре?

\item
Двое играют на~клетчатом листе бумаги $30 \times 30$.
Начинающий делает разрез вдоль одной стороны квадратика от~края листа,
второй продолжает разрез вдоль одной стороны квадратика и~т.~д.
Выигрывает тот, после чьего хода отвалится кусок.
Кто выиграет при правильной игре?

\item
На~столе лежит 100 спичек.
Двое ходят по~очереди.
За~один ход можно взять 1, 2, 4, 8, \ldots (любую степень двойки) спичек.
Проигрывает тот, кому нечего брать.
Кто выиграет при правильной игре?

\item
Миша, Лёва и~Федя решили сыграть в~следующую игру.
В~кучке лежат 2014 спичек.
Миша и~Лёва имеют право брать 1 или 2 спички, а~Федя~--- 1, 2 или 3.
При этом Миша и~Лёва объединяют свои усилия против Феди, а~Федя имеет право
выбрать очередь своего хода~--- первый, второй или третий.
Выигрывает тот, кто возьмет последнюю спичку.
Может~ли Федя выбрать себе такую очередь, что при правильной игре выиграет
именно~он?

\item
Двое играющих по~очереди переводят часовую стрелку на~2 или 3~часа вперед.
Вначале часовая стрелка указывает на~12, выигравшим объявляется тот, после
чьего хода она укажет на~6.
Кто выиграет при правильной игре?
(До~того, как остановиться на~цифре~6, стрелка может сделать сколько угодно
оборотов.)

\item
В~одной кучке 18~конфет.
В~другой~--- 23.
Двое по~очереди съедают одну из~куч, а~другую делят на~две~кучи.
Кто не~может поделить (в~куче осталась одна конфета), проигрывает.
Есть~ли у~начинающего выигрышная стратегия?

\end{problems}

