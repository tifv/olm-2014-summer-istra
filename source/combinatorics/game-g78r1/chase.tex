% $date: 2014-06-19

% $timetable:
%   g78r1:
%     2014-06-19:
%     - 2
%     2014-06-20:
%     - 2

\section*{Математические игры, часть 3}

% $caption: Математические игры, часть 3 (Преследования)

% $authors:
% - Владимир Шарич

\subsection*{Преследования}

\begin{problems}

\item
Белая ладья преследует черного слона на~доске $3 \times 2014$ клеток
(они ходят по~очереди по~обычным правилам).
Как должна играть ладья, чтобы взять слона?
Первый ход делают белые.

\item
В~одной из~вершин куба сидит заяц, но~охотникам он~не~виден.
Три охотника стреляют залпом, при этом они могут <<поразить>> любые три вершины
куба.
Если они не~попадают в~зайца, то~до~следующего залпа заяц перебегает в~одну
из~трех соседних (по~ребру) вершин куба.
Укажите, как стрелять охотникам, чтобы обязательно попасть в~зайца за~четыре
залпа.
%http://problems.ru/view_problem_details_new.php?id=103852

\item
На~плоскости расположены 100 точек-овец и~одна точка-волк.
За~один ход волк передвигается на~расстояние не~больше 1, после этого одна
из~овец передвигается на~расстояние не~больше 1, после этого снова ходит волк
и~т.~д.
При~любом~ли начальном расположении точек волк сможет поймать одну из~овец?

\item
Миша стоит в~центре круглой лужайке радиуса 100~метров.
Каждую минуту он~делает шаг длиной 1~метр.
Перед каждым шагом он~объявляет направление, в~котором хочет шагнуть.
Катя имеет право заставить его сменить направление на~противоположное.
Может~ли Миша действовать так, чтобы в~какой-то момент обязательно выйти
с~лужайки, или Катя всегда сможет ему помешать?

\item
Дана клетчатая полоска (шириной в~одну клетку), бесконечная в~обе стороны.
Две клетки полоски являются ловушками, между ними~--- $N$~клеток, на~одной
из~которых сидит кузнечик.
На~каждом ходу мы~называем натуральное число, после чего кузнечик прыгает
на~это число клеток влево или вправо (по~своему выбору).
При каких $N$ можно называть числа так, чтобы гарантированно загнать кузнечика
в~одну из~ловушек?
(Мы~всё время видим, где сидит кузнечик.)
%http://problems.ru/view_problem_details_new.php?id=64606

\item
Дорожки в~зоопарке образуют равносторонний треугольник, в~котором проведены
средние линии.
Из~клетки сбежала обезьянка.
Её~ловят два сторожа.
Смогут~ли они поймать обезьянку, если все трое будут бегать только по~дорожкам,
они видят друг друга и~обезьянка в~три раза быстрее каждого из~сторожей?
%http://problems.ru/view_problem_details_new.php?id=78751

\item
В~центре квадрата сидит заяц, а~в~каждом из~четырех углов по~одному волку.
Может~ли заяц выбежать из~квадрата, если волки могут бегать только по~сторонам
квадрата с~максимальной скоростью в~$1.4$~раза большей, чем максимальная
скорость зайца?
%http://problems.ru/view_problem_details_new.php?id=79470

\item
В~центре квадрата сидит волк, а~в~вершинах~--- сидят собаки.
Волк может бегать по~внутренности квадрата с~максимальной скоростью $v$,
а~собаки~--- только по~сторонам квадрата с~максимальной скоростью $1.5v$.
Известно, что волк задирает собаку, а~две собаки задирают волка.
Всегда~ли волк сможет выбежать из~квадрата?
% http://problems.ru/view_problem_details_new.php?id=35686

\end{problems}

