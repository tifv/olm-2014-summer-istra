% $date: 2014-06-17

% $timetable:
%   g78r1:
%     2014-06-17:
%     - 1

\section*{Математические игры, часть 1}

% $caption: Математические игры, часть 1 (Симметрия)

% $authors:
% - Владимир Шарич

\subsection*{Простая симметрия}

\emph{%
В~каждой из~следующих игр ходят по~очереди два игрока.
Укажите, кто из~них выиграет при правильной игре.}

\begin{problems}

\item
На~доске написано число~1.
Два игрока по~очереди прибавляют любое число от~1 до~5 к~числу на~доске
и~записывают вместо него сумму.
Выигрывает игрок, который первый запишет на~доске число тридцать.

\item
На~столе лежат две стопки монет: в~одной из~них 30 монет, а~в~другой~--- 20.
За~ход разрешается взять любое количество монет из~одной стопки.
Проигрывает тот, кто не~сможет сделать ход.

\item
У~ромашки $n$ лепестков.
За~ход разрешается сорвать либо один лепесток, либо два рядом растущих
лепестка.
Проигрывает игрок, который не~сможет сделать ход.

\item
В~каждой клетке доски $11 \times 11$ стоит шашка.
За~ход разрешается снять с~доски любое количество подряд идущих шашек либо
из~одного вертикального, либо из~одного горизонтального ряда.
Выигрывает снявший последнюю шашку.

\item
Вершины правильного $n$-угольника закрашены черной и~белой краской через одну.
Двое играют в~следующую игру.
Каждый по~очереди проводит отрезок, соединяющий вершины одинакового цвета.
Эти отрезки не~должны иметь общих точек (даже концов) с~проведенными ранее.
Побеждает тот, кто сделал последний ход.
\qquad
\sbp $n = 10$.
\qquad
\sbp $n = 12$.
   
\item
Дан прямоугольный параллелепипед размерами
\quad
\sbp $4 \times 4 \times 4$,
\quad
\sbp $4 \times 4 \times 3$,
\quad
\sbp $4 \times 3 \times 3$,
\quad
составленный из~единичных кубиков.
За~ход разрешается проткнуть спицей любой ряд, если в~нем есть хотя~бы один
непроткнутый кубик.
Проигрывает тот, кто не~может сделать ход.

\item
Двое играют на~доске $20 \times 14$ клеток.
Каждый по~очереди отмечает квадрат по~линиям сетки (любого возможного размера)
и~закрашивает его.
Выигрывает тот, кто закрасит последнюю клетку.
Дважды закрашивать клетки нельзя.

\item
Двое игроков по~очереди расставляют в~каждой из~24 клеток поверхности куба
$2 \times 2 \times 2 $ числа 1, 2, 3, \ldots, 24
(каждое число можно ставить один раз).
Второй игрок хочет, чтобы суммы чисел в~клетках каждого кольца из~8 клеток,
опоясывающего куб, были одинаковыми, а~первый хочет ему помешать.

\item
Двое игроков по~очереди выставляют на~доску $65 \times 65$ по~одной шашке.
При этом ни в~одной линии (горизонтали или вертикали) не~должно быть больше
двух шашек.
Кто не~может сделать ход~--- проиграл.
%http://problems.ru/view_problem_details_new.php?id=105123

\end{problems}

