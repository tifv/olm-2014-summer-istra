% $date: 2014-06-16

% $timetable:
%   g78r1:
%     2014-06-16:
%     - 2

\section*{Математические игры, часть 0}

% $caption: Математические игры, часть 0 (Шутки)

% $authors:
% - Владимир Шарич

\subsection*{Шутки}

\emph{%
В каждой из следующих игр ходят по очереди два игрока.
Укажите, кто из них выиграет при правильной игре.}

\begin{problems}

\item
Имеется три кучки камней: в первой~--- 10, во второй~--- 15, в третьей~--- 20.
За ход разрешается разбить любую кучку на две меньшие;
проигрывает тот, кто не сможет сделать ход.

\item
Двое по очереди ставят ладей на шахматную доску так, чтобы ладьи не били друг
друга.
Проигрывает тот, кто не может сделать ход.

\item
Числа от 1 до 20 выписаны в строчку.
Игроки по очереди расставляют между ними плюсы и минусы.
После того, как все места заполнены, подсчитывается результат.
Если он чётен, то выигрывает первый игрок, если нечётен, то второй.

\item
На доске написаны 10 единиц и 10 двоек.
За ход разрешается стереть две любые цифры и, если они были одинаковыми,
написать двойку, а если разными~--- единицу.
Если последняя оставшаяся на доске цифра~--- единица, то выиграл первый игрок,
если двойка~--- то второй.

\item
На доске написаны числа 25 и 36.
За ход разрешается дописать еще одно натуральное число~--- разность любых двух
имеющихся на доске чисел, если она еще не встречалась.
Проигрывает тот, кто не может сделать ход.

\item
Дана клетчатая доска размерами
\quad
\sbp $9 \times 10$;
\quad
\sbp $10 \times 12$;
\quad
\sbp $9 \times 11$.
\quad
За ход разрешается вычеркнуть любую горизонталь или любую вертикаль, если в ней
к моменту хода есть хотя бы одна невычеркнутая клетка.
Проигрывает тот, кто не может сделать ход.

\item
Двое играют в двойные шахматы:
все фигуры ходят как обычно, но каждый делает по два шахматных хода подряд.
Докажите, что первый может как минимум сделать ничью.

\item
Двое по очереди ломают шоколадку $6 \times 8$.
За ход разрешается сделать прямолинейный разлом любого из кусков вдоль
углубления.
Проигрывает тот, кто не сможет сделать ход.

\item
На квадратном поле размерами $n \times n$, разграфленном на клетки размерами
$1 \times 1$, играют двое.
Первый игрок ставит крестик на центр поля;
вслед за этим второй игрок может поставить нолик на любую из восьми клеток,
окружающих крестик первого игрока.
После этого первый ставит крестик на любое из полей рядом с уже занятыми, и~т.~д.
Выигрывает тот, кто первым коснется края игрового поля (поставит свою отметку в
клетку, примыкающую к краю).

\end{problems}

