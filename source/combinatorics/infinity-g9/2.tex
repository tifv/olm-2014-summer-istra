% $date: 2014-06-22

% $timetable:
%   g9:
%     2014-06-22:
%     - 1

\section*{Взгляд в бесконечность, часть 2}

% $authors:
% - Владимир Брагин

\begin{problems}

\item
\sbp\label{problem:combinatorics/infinity:infinite-line}%
За дядькой Черномором выстроилось чередой бесконечное число богатырей
различного роста, причем рост каждого составляет натуральное число сантиметров.
Доказать, что он может приказать части из них выйти из строя так, чтобы в
строю осталось бесконечное число богатырей, стоящих в порядке возрастания.
\\
\sbp
Имеется таблица из трех строк и бесконечного числа столбцов, занумерованных
натуральными числами.
В каждой клетке таблицы стоит натуральное число.
Доказать, что можно так выбрать последовательность столбцов в таблице, что в
каждой из строк число, стоящее в $i$-м столбце последовательности будет
не~меньше числа, стоящего в $(i+1)$-м столбце последовательности.
\\
\sbp
То же условие, что в пункте \ref{problem:combinatorics/infinity:infinite-line},
только на этот раз рост богатырей не обязательно выражается натуральным числом
сантиметров.
Доказать, что Черномор может приказать части из них выйти из строя так, чтобы в
строю осталось бесконечное число богатырей, стоящих в порядке возрастания или
убывания.

\item
Допустим, что любую конечную карту можно правильным образом раскрасить в 4
цвета.
Докажите, что тогда и бесконечную карту тоже можно раскрасить 4 цвета.

\item
В 6 коробках, выложенных в ряд, лежит по одной монете.
За один ход можно сделать одну из двух операций:
\\
\emph{(1)}
Убрать одну монету из коробки $i$ (при $i < 6$) и положить 2 монеты в коробку
$i + 1$;
\\
\emph{(2)}
Убрать одну монету из коробки $i$ (при $i < 5$) и поменять содержимое коробок
$i + 1$ и $i + 2$ местами.
\\
\sbp Докажите, что нельзя получить суммарно сколь угодно много монет.
\\
\sbp А можно ли получить в одной из коробок не менее $2014!$ монет?

\end{problems}

