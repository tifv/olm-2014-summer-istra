% $date: 2014-06-21

% $timetable:
%   g9:
%     2014-06-21:
%     - 2

\section*{Взгляд в бесконечность, часть 1}

% $authors:
% - Владимир Брагин

\begin{problems}

\item
\sbp
Докажите, что существует сколь угодно много последовательных составных чисел,
идущих подряд.
\\
\sbp
А существует ли бесконечно много последовательных составных чисел?

\item
В стране Фибоначчи есть купюры достоинством
$1$, $2$, $3$, $5$, $8$, $13$, $21$, $34$, $55$ лир.
У Леонардо есть купюра 55 лир.
Каждый день он может пойти в банк и обменять любую имеющуюся у него купюру на
любое количество купюр меньшего достоинства.
Кроме того, каждый день Леонардо должен тратить 1 лиру на еду.
Докажите, что Леонардо сможет существовать сколь угодно долго, но не бесконечно
долго.

\item
Известно, что человечество бессмертно, а каждый человек смертен.
Число людей в каждом поколении конечно.
Докажите, что найдется бесконечная мужская цепочка, начинающаяся с Адама.

\item
Две шайки гангстеров охотятся друг за другом.
Каждый гангстер охотится ровно за одним противником, и за каждым гангстером
охотится не более одного противника.
Главарь одной из шаек обнаружил, что не за всеми противниками охотятся.
Докажите, что обе шайки бесконечны.

\item В отеле $\infty$ бесконечно много номеров, занумерованных натуральными
числами.
Сейчас в каждом номере живет по гостю.
\\
\sbp
Приехал еще один гость.
Можно ли так переселить гостей, чтобы все поместились?
\\
\sbp
У каждого гостя приехало ровно по одному другу.
Можно ли переселить гостей теперь?
\\
\sbp\label{problem:finite-subsets}%
Приехало еще какое-то множество посетителей.
И оказалось, что каждый из приехавших знает конечное число постояльцев, причем
множества знакомых постояльцев ни у каких двоих из новых посетителей не
совпадает. Обязательно ли удастся вновь расселить?
\\
\sbp
Как изменится ответ на вопрос пункта \ref{problem:finite-subsets}, если
сказать, что у новых приехавших не обязательно конечное число знакомых
постояльцев?

\item
Натуральные числа раскрасили в два цвета.
Обязательно ли существует одноцветная бесконечная арифметическая прогрессия?

\item
Хан и Банах играют в игру с бесконечным количеством ходов.
Они по очереди выписывают цифры в последовательность.
Причем Хан пишет любое число цифр, а Банах только одну.
Хан хочет, чтобы последовательность получилась периодической, а Банах пытается
ему помешать.
Кто из них преуспеет?

\end{problems}

