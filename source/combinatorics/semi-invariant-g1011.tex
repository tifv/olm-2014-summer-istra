% $date: 2014-06-21

% $timetable:
%   g1011:
%     2014-06-21:
%     - 2

\section*{Инварианты и полуинварианты}

% $authors:
% - Сергей Беляков

\begin{problems}

\item
На доске вначале выписаны два числа: $3$ и $6$.
За один ход разрешается увеличить любое число на доске на сумму цифр любого из
выписанных.
Можно~ли добиться того, чтобы каждое число превратилось в 2010?

\item
Имеются три автомата.
На вход им подается карточка $(a, b)$.
Первый печатает карточку $(b, a)$.
Второй~--- $(a, a + b)$.
Третий, при $a > b$, печатает $(a - b, b)$.
Возможно ли при помощи этих автоматов получить из карточки $(1, 1)$ карточку
$(1000, 2000)$?

\item
В строчку выписаны $n$ натуральных чисел.
Разрешается взять любые два числа $a$ и $b$ такие, что $a$ стоит левее $b$ и
$b \neq k a$, и заменить $a$ на $(a, b)$, $b$ на $[a, b]$.
Докажите, что такие операции не могут продолжаться бесконечно долго.

\item
На Архипелаге Сыщик гоняется за Шпионом.
Оба используют только маршрутные корабли, которые курсируют ежедневно между
некоторыми островами.
Каждый корабль отплывает утром и приплывает на остров назначения к~вечеру.
С пересадками можно добраться с любого острова на любой.
Сыщик всегда знает, где сейчас Шпион, и поймает его, если окажется с ним на
одном острове.
Сыщик может плыть в любой день, Шпион не плавает по пятницам.
Как Сыщику поймать Шпиона?

\item
На доске написаны несколько натуральных чисел.
Каждую минуту выбирают какие-то два из них ($x$ и $y$) и заменяют их на числа
$x - 2$ и $y + 1$.
Докажите, что рано или поздно на доске появится отрицательное число.

\item
\sbp
В клетках таблицы $99 \times 99$ расставлены плюсы и минусы.
Если в каком-то ряду (строке или столбце) минусов больше чем плюсов,
разрешается в этом ряду поменять все знаки на противоположные.
Докажите, что через некоторое время и во всех строках, и во всех столбцах
плюсов будет больше чем минусов.
\\
\sbp
В клетках таблицы $99 \times 99$ расставлены целые числа.
Если в каком-то ряду (строке или столбце) сумма отрицательна, разрешается в
этом ряду поменять все знаки всех чисел на противоположные.
Докажите, что через некоторое время сумма чисел в каждом из рядов будет
неотрицательной.
\\
\sbp
В клетках таблицы $99 \times 99$ расставлены числа (не обязательно целые).
Если в каком-то ряду (строке или столбце) сумма отрицательна, разрешается в
этом ряду поменять все знаки всех чисел на противоположные.
Докажите, что через некоторое время сумма чисел в каждом из рядов будет
неотрицательной.

\item
В строке в произвольном порядке записаны числа $1$, $2$, $\ldots$, $100$.
Петя находит пару рядом стоящих чисел, где правое меньше левого, и меняет их
местами.
Докажите, что рано или поздно числа расположатся по порядку
$1$, $2$, $\ldots$, $100$.

\item
Из всех замкнутых ломаных с вершинами в данных точках выбрали самую короткую.
Докажите, что эта ломаная несамопересекающаяся.

\item
Есть куча из $n$ камней.
Разрешается заменять кучу на любое количество куч с меньшим количеством камней
(возможно, различным в разных кучах).
Докажите, что наступит момент, когда уже нельзя будет сделать ни одной такой
операции.

\item
На плоскости дано $100$ красных и $100$ синих точек, никакие три из которых
не~лежат на одной прямой.
Докажите, что можно провести $100$ непересекающихся отрезков с разноцветными
концами.

\item
По одной стороне бесконечного коридора расположено бесконечное число комнат,
занумерованных по порядку целыми числами, и в каждой стоит по роялю.
В этих комнатах живет некоторое количество пианистов
(в одной комнате могут жить несколько пианистов).
Каждый день какие-то два пианиста, живущие в соседних комнатах~--- $k$-й и
$(k + 1)$-й, приходят к выводу, что они мешают друг другу и переселяются
соответственно в $(k - 1)$-ю и $(k + 2)$-ю комнаты.
Докажите, что через конечное число дней эти переселения прекратятся.

\item
В год выборов в Лапландии все города подняли над ратушами флаги~--- голубые
либо оранжевые.
Каждый день жители узнают цвета флагов у соседей в~радиусе 100~км.
Один из городов, где у большинства соседей флаги другого цвета, меняет свой
флаг на этот другой цвет.
Докажите, что со временем смены цвета флагов прекратятся.

\end{problems}

