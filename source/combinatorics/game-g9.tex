% $date: 2014-06-25

% $timetable:
%   g9:
%     2014-06-25:
%     - 1

\section*{Игры}

% $authors:
% - Михаил Харитонов

\begin{problems}

\itemy{0}
Коля и~Витя играют в~следующую игру.
На~столе лежит куча из~$31$~камня.
Мальчики делают ходы поочередно, а~начинает Коля.
Делая ход, играющий делит одну из~кучек, в~которых больше одного камня, на~две
меньшие кучки.
Выигрывает тот, кто после своего хода оставляет кучки по~одному камню в~каждой.
Сможет~ли Коля сделать так, чтобы выиграть при любой игре Вити?

\item
Коля и~Витя играют в~следующую игру.
На~столе лежит куча из~$31$ камня.
Мальчики делают ходы поочередно, а~начинает Коля.
Делая ход, играющий делит каждую кучку, в~которой больше одного камня, на~две
меньшие кучки.
Выигрывает тот, кто после своего хода оставляет кучки по~одному камню в~каждой.
Сможет~ли Коля сделать так, чтобы выиграть при любой игре Вити?

\item
Двое играют на~доске $25 \times 62014$ клеток.
Каждый по~очереди отмечает квадрат по~линиям сетки (любого возможного размера)
и~закрашивает его.
Выигрывает тот, кто закрасит последнюю клетку.
Дважды закрашивать клетки нельзя.
Кто выиграет при правильной игре и~как надо играть?

\item
Капитан Врунгель в~своей каюте разложил перетасованную колоду из~52 карт
по~кругу, оставив одно место свободным.
Матрос Фукс с~палубы, не~отходя от~штурвала и~не~зная начальной раскладки,
называет карту.
Если эта карта лежит рядом со~свободным местом, Врунгель ее~туда передвигает,
не~сообщая Фуксу.
Иначе ничего не~происходит.
Потом Фукс называет еще одну карту, и~так сколько угодно раз, пока сам не
скажет <<стоп>>.
Может~ли Фукс добиться того, чтобы после <<стопа>> каждая карта наверняка
оказалась не~там, где была вначале?

\item
Лежит кучка в~146 миллионов спичек.
Двое играют в~следующую игру.
Ходят по~очереди.
За~один ход играющий может взять из~кучки спички в~количестве $p^n$, где
$p$~--- простое число, $n = 0, 1, 2, 3, \ldots$
(например, первый берёт 25 спичек, второй~--- 8, первый~--- 1, второй~--- 5,
первый~--- 49 и~т.~д.).
Выигрывает тот, кто берёт последнюю спичку.
Кто выигрывает при правильной игре?

\item
На~горе 1001 ступенька, на~некоторых лежат камни, по~одному на~ступеньке.
Сизиф берёт любой камень и~переносит его на~ближайшую сверху свободную
ступеньку (то~есть если следующая ступенька свободна, то~на~неё, а~если занята,
то~на~несколько ступенек вверх до~первой свободной).
После этого Аид скатывает на~одну ступеньку вниз один из~камней, у~которых
предыдущая ступенька свободна.
Камней 500, и~первоначально они лежали на~нижних 500 ступеньках.
Сизиф и~Аид действуют по~очереди, начинает Сизиф.
Его цель~--- положить камень на~верхнюю ступеньку.
Может~ли Аид ему помешать?

\item
На~столе лежат три кучки спичек.
В~первой кучке находится 100 спичек, во второй~--- 200, а~в~третьей~--- 300.
Двое играют в~такую игру.
Ходят по~очереди, за~один ход игрок должен убрать одну из~кучек, а~любую
из~оставшихся разделить на две непустые части.
Проигравшим считается тот, кто не~может сделать ход.
Кто выигрывает при правильной игре: начинающий или его партнер?

\end{problems}

