% $date: 2014-06-21

% $timetable:
%   g78r2:
%     2014-06-21:
%     - 1

\section*{Комбинаторный разнобой}

% $authors:
% - Глеб Погудин

\begin{problems}

% spell "правдивец" -> "рыцарь"
% spell "правдивцем" -> "рыцарем"

\item
Сколькими способами можно расставить числа $1$ и $-1$ в клетки квадрата
$4 \times 4$, чтобы все суммы по строкам и столбцам были равны нулю?

\item
Есть шоколадка $10 \times 10$.
За одну операцию можно выбрать квадратик и съесть все квадратики, которые не
ниже и не левее его.
Сколько разных фигур можно получить таким образом?

\item
Лампочки расставлены в виде квадрата $6 \times 6$.
Исходно все они выключены.
За одну операцию можно изменить состояние всех лампочек в некотором квадрате
$2 \times 2$ на противоположное.
Сколько разных конфигураций можно таким образом получить?

\item
Есть два человека, $A$ и $B$.
Каждый из них либо лжец, либо правдивец, либо нормальный.
$A$: <<$B$ правдивец>>.
$B$: <<$A$ не правдивец>>.
Докажите, что хотя бы один из них говорит правду, но не является правдивцем.

\item
Куб со стороной $20$ разбит на $8000$ единичных кубиков.
В каждом кубике написано число.
Известно, что в любом столбике высоты $20$ (в любом из трех направлений) сумма
чисел равна единице.
Выбран некоторый кубик, в нем написано число $10$.
Через этот кубик проходит три слоя $1\times 20 \times 20$.
Найдите сумму чисел вне этих слоев.

\item
Петя загадал пару натуральных чисел и сообщил Вите, что их произведение равно
$120$.
Помогите Вите создать три карточки так, чтобы, задав Пете вопросы
<<Есть ли задуманное тобой число на этой карточке>> и получив на них ответы,
угадать эту пару.
Порядок чисел в паре не имеет значения.

\end{problems}

