% $date: 2014-06-25

% $timetable:
%   g78r2:
%     2014-06-25:
%     - 2
%     2014-06-26:
%     - 1

\section*{Принцип крайнего}

% $authors:
% - Лев Шабанов

\begin{problems}

\item
Можно~ли разрезать кубик $3 \times 3 \times 3$ на~кубики $1 \times 1 \times 1$,
используя менее 6~разрезов?

\item
На~листке написано несколько целых чисел.
Среди них, для любых двух чисел $a$ и $b$ есть третье, которое делится на~$a$ и~на~$b$.
Докажите, что есть число, которое делится на каждое из остальных.

\item
\sbp
Числа от~1 до~2013 выписали в~ряд.
Может~ли разность любых двух соседних быть не~менее 1007?
\\
\sbp
Тот~же вопрос, если числа от~1 до~2014.

\item
По~кругу стоят 100 действительных чисел так, что каждое число равняется
полусумме двух соседей.
Докажите, что все числа равны.

\item
По~кругу записаны 30 действительных чисел, каждое равное модулю разности двух
следующих за~ним по~часовой стрелке (то~есть разности с~отброшенным знаком).
Сумма всех чисел равна 300.
Что это за~числа и~в~каком порядке записаны?

\item
На~доске были написаны 5~действительных чисел.
Сложив их~попарно, получили следующие 10 чисел:
0, 2, 4, 4, 6, 8, 9, 11, 13, 15.
Какие числа были написаны?

\item
На~столе лежат несколько круглых монет, касаясь, но~не~перекрывая друг друга.
Докажите, что какая-то монета касается не~более пяти других.
(Размеры монет могут быть любыми).

\end{problems}

