% $date: 2014-06-15

% $timetable:
%   g10:
%     2014-06-15:
%     - 2

\section*{Рекурренты в комбинаторике, часть 1}

% $caption: Рекурренты в комбинаторике, часть 1 (Зацикливания)

% $authors:
% - Владимир Шарич

\subsection*{Немного зацикливаний}

%\textbf{Обсуждение:}
%\begin{itemize}
%\item периодичность десятичной записи
%\item рекурренты $a_n = a_{n-1} - a_{n-2}$, $b_n = b_{n-1} / b_{n-2}$
%\item арифметика периодических последовательностей
%\item \ldots
%\end{itemize}

\begin{problems}

\item
В~последовательности 2014\ldots каждая цифра, начиная с~пятой, равна последней
цифре суммы предыдущих четырех цифр.
\\
\sbp Встретится~ли в~этой последовательности набор цифр 2013?
\\
\sbp Докажите, что эта последовательность периодична.
\\
\sbp Есть~ли у~этой последовательности предпериод?
\\
\sbp Встретится~ли в~этой последовательности набор цифр 1201?

\item
Числа $w_n$ образуют последовательность
1, 3, 13, 63, 313, \ldots\ (каждое число, начиная с~третьего, вычисляется по~формуле
$w_n = 5 w_{n-1} - 2$).
Докажите, что в~этой последовательности бесконечно много членов, дающих остаток
1 при делении на~101.

\item
Кубик Рубика вывели из~исходного состояния некоторой последовательностью
поворотов граней.
Докажите, что если повторять эту последовательность поворотов достаточно долго,
то~кубик в~конце концов вернется в~исходное состояние.

\item
На~бесконечной в~обе стороны ленте записан текст на~русском языке.
Известно, что в~этом тексте число различных кусков из~15 символов равно числу
различных кусков из~16 символов.
Докажите, что на~ленте записан периодический \emph{в~обе стороны} текст,
например: <<\ldots мама мыла раму мама мыла раму \ldots>>

% spell "Элмышатия" -> ""
% spell "четырехдневку" -> "серию из четырех дней"
% spell "четырехдневок" -> "серий из четырех дней"

\item
Государство Элмышатия всегда существовало и~всегда будет существовать.
Каждый день в~этом государстве либо идет дождь, либо бушует буря, либо светит
солнце.
Известно, что погода в~данный день однозначно определяется погодой
за~предшествующую четырехдневку.
Всю последнюю четырехдневку шел дождь.
Докажите, что и~до и~после этого дождливых четырехдневок было бесконечно много.

\item
Докажите, что в~ряду Фибоначчи существует число, делящееся на~2014.

\item
$\{ a_n \}$~--- последовательность чисел между 0 и~1, в~которой следом за~$x$
идет $1 - |1 - 2 x|$.
Докажите, что эта последовательность, начиная с~некоторого места,
периодическая, если и~только если $a_1$ рациональное.

\item
Последовательность задается следующими соотношениями:
\[
    x_{2n} = x_n
,\quad
    x_{4n+1} = 1
,\quad
    x_{4n+3} = 0
.\]
Докажите, что эта последовательность не~может быть периодической.

\end{problems}

