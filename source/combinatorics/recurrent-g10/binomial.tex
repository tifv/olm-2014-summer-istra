% $timetable:
%   g10: {}

\section*{Рекурренты в комбинаторике, часть 4}

% $caption: Рекурренты в комбинаторике, часть 4 (Комбинации)

% $authors:
% - Владимир Шарич

\subsection*{Числа сочетаний}

\definition
\emph{Число сочетаний} $\dbinom{n}{k}$, $0 \leq k \leq n$,
задается рекуррентно:
\[
    \binom{n}{0} = 1
,\quad
    \binom{n}{n} = 1
,\qquad
    \binom{n}{k} = \binom{n - 1}{k} + \binom{n - 1}{k - 1}
.\]

\observation
Известной формулой $\dbinom{n}{k} = \dfrac{n!}{k! (n - k)!}$
\emph{пользоваться нельзя}.

\begin{problems}

\item\emph{Определение.}
\\
\sbp\emph{Треугольник Паскаля.}
\emph{Хромым королём} назовем <<шахматную>> фигуру, которая может ходить только
на~одну клетку и~только вправо и~вниз.
Допустим, хромой король стоит в~левом верхнем углу бесконечной вправо и~вниз
<<шахматной>> доски, вертикали и~горизонтали которой занумерованы,
соответственно, справа налево и~сверху вниз неотрицательными целыми числами
(начиная с~нуля).
Сколькими способами хромой король может добраться с~поля $(0, 0)$ на~поле
$(p, q)$?
\\
\sbp\emph{Бином Ньютона.}
При раскрытии скобок в~выражении $(a + b)^n$ получается сумма выражений вида
$a^{k} b^{n-k}$.
Сколько раз встречается каждое из~таких слагаемых?
\\
\sbp\emph{Подмножества.}
Сколько существует $k$-эле\-мен\-тных подмножеств $n$-эле\-мен\-тного множества?

\end{problems}

\subsubsection*{Забавные свойства}

\begin{problems}

\item
\(\displaystyle
    \sum_{k}
        \binom{n}{k}
=
    2^n
\);
\quad
\(\displaystyle
    \sum_{k}
        (-1)^k \binom{n}{k}
=
    0
\).
\qquad
\problem
\(\displaystyle
    \sum_{k}^n
        (-1)^k \binom{m}{k}
=
    (-1)^{n-1} \binom{m-1}{n}
\).

\item
\(\displaystyle
    (n - k) \binom{n}{k}
=
    n \binom{n - 1}{k}
\);
\quad
\(\displaystyle
    k \binom{n}{k}
=
    n \binom{n - 1}{k - 1}
\).

\itemy{6}% XXX
\(\displaystyle
    \binom{n}{m} \binom{m}{k}
=
    \binom{n}{k} \binom{n - k}{m - k}
=
    \binom{n}{m - k} \binom{n - m + k}{k}
\).

\itemy{5}% XXX
\(\displaystyle
    \sum_{k}
        \binom{p}{k} \binom{q}{n - k}
=
    \binom{p + q}{n}
\).
\qquad\setproblem{6}% XXX
\problem
\(\displaystyle
    \sum_{k}
        \binom{n - k}{k}
=
    F_n
\).

\item
На~прямой в~точке с~координатой ноль сидит бактерия.
Каждую минуту бактерия делится
(если бактерия находилась в~точке $J$, то~через минуту две бактерии находятся
в~точках с~координатами $J - 1$ и $J + 1$).
Если в~одну точку попадают две бактерии, то~они обе погибают.
Как будут расположены бактерии на~прямой через 2 часа и~8 минут?

\end{problems}

