% $date: 2014-06-15

% $timetable:
%   g10:
%     2014-06-15:
%     - 2

\section*{Рекурренты в комбинаторике, часть 2}

% $caption: Рекурренты в комбинаторике, часть 2 (Одномерные задачи)

% $authors:
% - Владимир Шарич

\subsection*{Одномерные задачи}

\begin{problems}

\item\emph{Игра <<Ханойская башня>>.}
Имеется пирамида из~$n$ колец, надетых на~стержень, и~два пустых стержня той~же
высоты.
Диаметры колец убывают от~основания пирамиды к~ее~вершине
(т.~е. у~основания находится самое большое кольцо, наверху~---
самое маленькое).
Разрешается перекладывать верхнее кольцо с~одного стержня на~другой,
но~при этом запрещается класть большее кольцо на~меньшее.
Требуется переложить все кольца с~одного стержня на~другой.
За~какое наименьшее количество перекладываний это можно сделать?

\item
На~какое максимальное число частей можно разрезать головку сыра при помощи
$n$ разрезов?
\\
\sbp Головка плоская, разрезы прямолинейные.
\\
\sbp Головка объемная, разрезы плоские.

\item\emph{Задача Иосифа Флавия.}
$n$ человек выстраиваются по~кругу и~нумеруются числами от~$1$ до~$n$.
Затем из~них исключается каждый второй до~тех пор, пока не~останется только
один человек.
Для данного $n$ будем обозначать через $J(n)$ номер последнего оставшегося
человека.
Докажите, что
\[
    J(2 n) = 2 J(n) - 1
;\qquad
    J(2 n + 1) = 2 J(n) + 1
.\]

\item
Рассмотрим все возможные наборы чисел из~множества $\{ 1, 2, 3, \ldots, n \}$,
не~содержащие двух соседних чисел.
Докажите, что сумма квадратов произведений чисел в~этих наборах равна
$(n + 1)! - 1$.

\item\emph{Обобщенная задача Иосифа Флавия.}
Допустим, что в~круг поставлено $2 n$ человек, первые $n$ из~которых~---
славные ребята, а~последние $n$~--- гадкие парни.
Покажите, что всегда найдется такое целое $m$, зависящее от~$n$, такое что
если двигаясь по~кругу мы~наказываем каждого $m$-го, то~все гадкие парни будут
наказаны прежде, чем будет наказан хотя~бы один из~славных.
Наказанный человек выбывает из~круга сразу после наказания.

\item
Банкир узнал, что среди одинаковых на~вид монет одна~--- фальшивая
(более легкая).
Он попросил эксперта определить эту монету с~помощью чашечных весов без гирь,
причем потребовал, чтобы каждая монета участвовала во~взвешиваниях не~более
двух раз.
Какое наибольшее число монет может быть у~банкира, чтобы эксперт заведомо смог
выделить фальшивую за~$n$ взвешиваний? 
%http://problems.ru/view_problem_details_new.php?id=107826

\item
В~семейном альбоме есть десять фотографий.
На~каждой из~них изображены три человека: в~центре стоит мужчина, слева
от~мужчины~--- его сын, а~справа~--- его брат.
Какое наименьшее количество различных людей может быть изображено на~этих
фотографиях, если известно, что все десять мужчин, стоящих в~центре, различны?
%http://problems.ru/view_problem_details_new.php?id=109524   

\end{problems}

