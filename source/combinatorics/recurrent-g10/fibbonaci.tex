% $date: 2014-06-17

% $timetable:
%   g10:
%     2014-06-17:
%     - 2

\section*{Рекурренты в комбинаторике, часть 3}

% $caption: Рекурренты в комбинаторике, часть 3 (Фиббоначи)

% $authors:
% - Владимир Шарич

\subsection*{Числа Фибоначчи}

\begin{problems}

\item
На~первой клетке полоски $1 \times n$ сидит математический кузнечик.
За~один ход он~может прыгнуть на~одну или две клетки вправо.
Сколько у~него способов допрыгать из~первой клетки в~клетку с~номером $n$?

% spell "доминошки" -> "фигурки домино"

\item
Сколько способов разрезать полоску $2 \times n$ на~доминошки?

\item
Сколько среди $n$-значных чисел, состоящих из~цифр 2 и~5, таких, у~которых две
двойки не~стоят подряд?

\item
Фермер купил овцу, которая тут~же родила овечку.
С~тех пор каждый следующий год эта овца приносила по~одной овечке.
Каждая родившаяся овца через три года также начинает приносить по~одной овечке
в~год.
Допустим, что овцы бессмертны, а~фермер их~не~продает и~не~режет.
Сколько овец будет в~его отаре через $n$ лет после покупки?

\item
Даны две последовательности $a_n$, $b_n$, в~каждой из~которых каждый член
равен сумме двух предыдущих.
При этом $a_0 = 1$, $a_1 = 2$ и $b_0 = 2$, $b_1 = 1$.
Найдите количество совпадающих членов
\\
\sbp с~одинаковыми номерами, т.~е. $a_k = b_k$;
\\
\sbp с~необязательно одинаковыми номерами, т.~е. $a_l = b_m$;
\\
в~этих последовательностях.

\item
Рассматривается последовательность слов из~букв \texttt{A} и~\texttt{B}.
Первое слово~--- \texttt{A}, второе~--- \texttt{B},
а~$k$-е слово получается приписыванием к~$(k-2)$-му слову справа $(k-1)$-го
(так что начало последовательности имеет вид:
\texttt{A}, \texttt{B}, \texttt{AB}, \texttt{BAB}, \texttt{ABBAB}, \ldots).
Может~ли в~последовательности встретиться <<периодическое>> слово, то~есть
слово, состоящее из~нескольких (по~меньшей мере двух) одинаковых кусков, идущих
друг за~другом, и~только из~них?

\item
Требуется сделать набор гирек, каждая из~которых весит целое число граммов,
с~помощью которых можно взвесить любой целый вес от~1 до~55 граммов
включительно даже в~том случае, если некоторые гирьки потеряны
(гирьки кладутся на~одну чашку весов, измеряемый вес~--- на~другую).
Рассмотрите два варианта задачи:
\\
\sbp необходимо подобрать 10~гирек, может быть потеряна любая одна;
\\
\sbp необходимо подобрать 12~гирек, могут быть потеряны любые две.
  
\item
Имеется набор гирь, веса которых в~граммах: 1, 2, 4, \ldots, 512
(последовательные степени двойки)~--- по~одной гире каждого веса.
Груз разрешается взвешивать с~помощью этого набора, кладя гири на~обе чашки
весов.
Каково наибольшее возможное количество способов взвесить некоторый груз
указанным образом?
%http://problems.ru/view_problem_details_new.php?id=98344

\end{problems}

