% $date: 2014-06-19

% $timetable:
%   g10:
%     2014-06-19:
%     - 1
%     2014-06-20:
%     - 1
%     2014-06-21:
%     - 1

\section*{Рекурренты в комбинаторике, часть 5}

% $caption: Рекурренты в комбинаторике, часть 5 (Каталан)

% $authors:
% - Владимир Шарич

\subsection*{Числа Каталана}

\definition
\emph{Числа Каталана} $C_n$, $n \geq 0$,
задаются рекуррентно:
\[
    C_n = C_0 C_{n-1} + C_1 C_{n-2} + \ldots + C_{n-1} C_0
,\qquad
    C_0 = 1
\;.\]

Докажите, что:

\begin{problems}

\item
$C_n$ равно количеству способов разбить выпуклый $(n + 2)$-угольник
на~треугольники непересекающимися диагоналями.

\item
$C_n$ равно количеству способов расставить в~ряд $n$ открывающихся
и~$n$ закрывающихся скобок так, чтобы запись была корректна
(то~есть, среди любого количества первых элементов ряда открывающихся скобок
не~меньше, чем закрывающихся).
%Например, если $n=3$, то таких способов пять: ((())), (())(), $()(())$,
%$(()())$, $()()()$.
%Следовательно, $C_3=5$.

\item
$C_n$ равно количеству плоских бинарных деревьев с~$n + 1$ листьями.
Бинарным называется дерево с~выделенной вершиной \emph{(корнем)} степени 2,
все остальные вершины которого имеют степень 1 или 3.

\item
$C_n$ равно количеству путей из~точки $(0, 0)$ в~точку $(n, n)$ по~линиям
клетчатой бумаги, идущих вверх и~вправо и~не~поднимающихся выше прямой $y = x$.

\item
$C_n$ равно количеству последовательностей $a_1$, $a_2$, \ldots, $a_{2n}$,
состоящих только из~$1$ или $-1$, и~для всякого
$k \in \{ 1, \ldots, 2n \}$ выполнено $a_1 + a_2 + \ldots + a_k \geq 0$, причем
$a_1 + a_2 + \ldots + a_n = 0$.
%Например, если $n=3$, то таких последовательностей пять: $+1+1+1-1-1-1$,
%$+1+1-1-1+1-1$, $+1-1+1+1-1-1$, $+1+1-1+1-1-1$, $+1-1+1-1+1-1$.
%Следовательно, $C_3=5$.

\item
$C_n$ равно количеству способов разбить $2n$ точек на окружности на пары и
соединить точки из одной пары отрезком так, чтобы отрезки не пересекались.

\item
$C_n = \dfrac{1}{n + 1}\dbinom{2n}{n}$.

\end{problems}

