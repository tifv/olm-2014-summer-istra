% $date: 2014-06-19

% $timetable:
%   g9:
%     2014-06-19:
%     - 1

\section*{Полуинварианты и процессы}

% $authors:
% - Иван Митрофанов

\begin{problems}

\item
В клетках прямоугольной таблицы записаны числа $-1$ и $+1$.
Если в строке/столбце сумма чисел отрицательная, разрешается менять знак у всех
чисел в строке/столбце.
\\
\sbp
Можно ли из первой таблицы получить вторую?
\begin{center}
\jeolmfigure[width=0.3\textwidth]{signes-1}
\quad
\jeolmfigure[width=0.3\textwidth]{signes-2}
\end{center}
\sbp
Докажите, что какая бы изначально ни была расстановка чисел, можно сделать лишь
конечное число таких операций.
\\
\sbp
То же самое, но числа в таблице произвольные действительные.
\\
\sbp
Дана прямоугольная таблица, в клетках которой записаны действительные числа.
Разрешается менять знак у всех чисел в любом квадратике $3 \times 3$.
Докажите, что можно сделать так, чтобы в любом квадратике $3 \times 3$ сумма
чисел была неотрицательная.
\\
\sbp
Всегда ли можно сделать так, чтобы сумма всех чисел стала неотрицательна?

\item
Если на доске написан квадратный трехчлен $x^2 + bx + c$, Петя может выбрать
произвольное $d$ и написать новый трехчлен  $x^2 + (b + 2 d) x + (c + b d)$.
Может ли на доске после нескольких таких операций появиться трехчлен
$x^2 - 2000 x + 1000000$, если изначальный трехчлен имел свободный член $-1$?

\item
В квадрате $10 \times 10$ покрашено $9$ клеток.
Каждый день закрашивают те клетки, у которых не менее двух соседних по стороне
уже покрашены.
Докажите, что полностью квадрат закрашен никогда не будет.


\subsection*{Доказать, что процесс остановится}

% spell "б\'{о}льшая" -> "большая"

\item
Изначально в связном графе несколько вершин покрашены в черный цвет, а
несколько в белый.
За одну операцию можно перекрасить какую-нибудь вершину в тот цвет, в который
покрашена б\'{о}льшая  часть её соседей.
Докажите, что когда-нибудь перекрашивать будет нечего.

\item
Дан невыпуклый многоугольник $A_1 A_2 \ldots A_n$.
Если несмежные вершины $A_i$ и $A_j$ многоугольника таковы, что он лежит
целиком по одну сторону от прямой $A_iA_j$, то можно взять одну из двух
ломаных, на которые точки $A_i$ и $A_j$ его разбивают, и отразить симметрично
центра отрезка $A_i A_j$.
Докажите, что рано или поздно многоугольник станет выпуклым.

\item
В колоде часть карт лежит <<рубашкой вниз>>.
Время от времени Петя вынимает из колоды пачку из одной или нескольких подряд
идущих карт, в которой верхняя и нижняя карты лежат <<рубашкой вниз>>,
переворачивает всю пачку как одно целое и вставляет её в то же место колоды.
Докажите, что в конце концов все карты лягут <<рубашкой вверх>>, как бы ни
действовал Петя.

%\item
%По кругу выписано несколько чисел.
%Если для некоторых четырех идущих подряд чисел %$a$, $b$, $c$, $d$
%оказывается, что $(a - d) (b - c) < 0$, то числа $b$ и $c$ можно поменять
%местами.
%Докажите, что такую операцию можно проделать лишь конечное число раз.


\subsection*{Придумать процесс}

\item
В парламенте у каждого депутата не более трех врагов
(если $A$ враг $B$, то $B$ враг $A$).
Докажите, что депутаты могут разбиться на две партии так, чтобы внутри каждой
партии у каждого депутата было бы не более одного врага.

\item
На окружности расставлено несколько положительных чисел, каждое из которых
не больше 1.
Докажите, что можно разделить окружность на три дуги так, что суммы чисел на
соседних дугах будут отличаться не больше, чем на 1.
(Если на дуге нет чисел, то сумма на ней считается равной нулю.)

\end{problems}

