% $date: 2014-06-15

% $timetable:
%   g9: 
%     2014-06-15:
%     - 3

\section*{Раскраски}

% $authors:
% - Иван Митрофанов

\begin{problems}

% spell "доминошки" -> "фигурки домино"
% spell "доминошек" -> "фигурок домино"

\item
Доска $8 \times 8$ разрезана на фигурки домино $1 \times 2$.
Докажите, что количество горизонтальных и вертикальных фигурок четно.
Фигурки разрешается поворачивать и переворачивать.

\item
Куб размером $3 \times 3 \times 3$ состоит из $27$ единичных кубиков.
Можно ли побывать в каждом кубике по одному разу, двигаясь следующим образом:
из кубика можно пройти в любой кубик, имеющий с ним общую грань, причем
запрещено ходить два раза подряд в одном направлении?

\item
Каждая сторона равностороннего треугольника разбита на $n$ равных частей.
Через точки деления проведены прямые, параллельные сторонам.
В результате треугольник разбит на $n^2$ треугольничков.
Назовём цепочкой последовательность треугольничков, в которой ни один не
появляется дважды и каждый последующий имеет общую сторону с предыдущим.
Каково наибольшее возможное количество треугольничков в цепочке?

\item
Концы $N$ хорд разделили окружность на $2N$ дуг единичной длины.
Известно, что каждая из хорд делит окружность на две дуги четной длины.
Докажите, что число $N$ четно.

\item
Можно ли доску размером
\quad
\sbp $10 \times 10$
\quad
\sbp $12 \times 12$
\quad
клеток разрезать на фигурки
\jeolmfigure[height=1.5ex]{t-tetramino}
из четырех клеток?
Фигурки разрешается поворачивать и переворачивать.

\item
Для каких $n$ доску размера $n \times n$ можно разбить на прямоугольники
$1 \times 4$?
Фигурки разрешается поворачивать и переворачивать.

\item
Из $54$ одинаковых единичных картонных квадратов сделали незамкнутую цепочку,
соединив их \emph{шарнирно} вершинами.
Любой квадрат (кроме крайних) соединен с соседями двумя противоположными
вершинами.
Докажите, что этой цепочкой квадратов нельзя полностью закрыть поверхность куба
$3 \times 3 \times 3$.

\item
Каждая сторона равностороннего треугольника разделена на 6 равных частей, через
точки деления проведены прямые, параллельные сторонам, делящие исходный
треугольник на 36 маленьких треугольничков.
В каждой из вершин этих треугольничков сидит по жуку.
Они одновременно начинают двигаться по линиям деления с равными скоростями.
Когда жук попадает в вершину треугольничка, он поворачивает на 60 или 120
градусов.
Докажите, что через некоторое время какие-то два жука окажутся в одной вершине
маленького треугольничка.

\end{problems}
