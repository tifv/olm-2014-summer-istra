% $date: 2014-06-20

% $timetable:
%   g9:
%     2014-06-20:
%     - 2

\section*{Серия 3, про радикальную ось и не только}

% $authors:
% - Фёдор Бахарев

\definition
Величина $OA^2 - r^2$ называется \emph{степенью точки $A$} относительно
окружности с~радиусом $r$ и центром~$O$.

\begin{problems}

\item
\sbp
Докажите, что геометрическое место точек с равными степенями относительно двух
неконцентрических окружностей есть прямая, перпендикулярная их линии центров.
Эта прямая называется \emph{радикальной осью} двух окружностей.
\\
\sbp
Докажите, что радикальные оси трех окружностей, центры которых не лежат на
одной прямой, пересекаются в одной точке.
Эта точка называется \emph{радикальным центром} трёх окружностей.

\item
\sbp
Докажите, что радикальная ось делит отрезок общей касательной двух окружностей
пополам.
\\
\sbp
В угол вписаны две окружности.
Одна из них касается сторон угла в точках $A$ и $B$, а другая~--- в точках $C$
и $D$ соответственно.
Докажите, что прямая $AD$ высекает на этих окружностях равные хорды.

\item
На стороне $BC$ треугольника $ABC$ взята точка $A'$.
Серединный перпендикуляр к отрезку $A'B$ пересекает сторону $AB$ в точке $M$,
а серединный перпендикуляр к отрезку $A'C$ пересекает сторону $AC$ в точке $N$.
Докажите, что точка, симметричная точке $A'$ относительно прямой $MN$, лежит на
описанной окружности треугольника $ABC$.

\item
Через вершину $B$ остроугольного треугольника $ABC$ проведено две окружности,
которые касаются стороны $AC$ в точках $A$ и $C$ и пересекаются вторично
в~точке~$M$.
\\
\sbp
Докажите, что $M$ лежите на медиане треугольника, выходящей из вершины $B$.
\\
\sbp
Докажите, что $A$, $C$, $M$ и ортоцентр треугольника $H$ лежат на одной
окружности.

\item
\sbp
Докажите, что точка, симметричная ортоцентру $H$ треугольника $ABC$
относительно середины стороны лежит на описанной окружности треугольника $ABC$.
\\
\sbp
Докажите, что $A$, $C$, $H$ и проекция $H$ на медиану треугольника, выходящую
из вершины $B$, лежат на одной окружности.

\item
Внутри выпуклого многоугольника расположено несколько попарно непересекающихся
кругов различных радиусов.
Докажите, что многоугольник можно разрезать на маленькие многоугольники так,
чтобы все они были выпуклыми и в каждом из них содержался ровно один из данных
кругов.

\item
\sbp
Окружность $s_1$ касается окружности $s$ внутренним образом в точке $N$.
Хорда $AB$ окружности $s$ касается окружности $s_1$ в точке $M$.
Докажите, что $MN$ делит дугу $AB$, не содержащую точку $N$, пополам.
\\
\sbp
Окружности $s_1$ и $s_2$ касаются окружности $s$ внутренним образом.
Хорда $AB$ окружности $s$ является общей внешней касательной окружностей $s_1$
и $s_2$.
Докажите, что их радикальная ось проходит через середину дуги $AB$.

\end{problems}

