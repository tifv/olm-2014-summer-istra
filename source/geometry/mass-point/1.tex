% $date: 2014-06-16

% $timetable:
%   g10:
%     2014-06-16:
%     - 1

\section*{Центр масс, часть 1}

% $authors:
% - Андрей Меньщиков

\begingroup\let\ov\overrightarrow

\definition
\emph{Материальная точка}~--- это точка вместе с~размещенной в~ней массой $m$.
Далее, под записью вида $mA$ мы будем подразумевать, что в~точку~$A$ помещена
масса~$m$.

\definition
\emph{Центром масс} материальных точек $m_1A_1$, $\ldots$, $m_n A_n$ называется
материальная точка $mZ$, для которой
\(
    m_1 \ov{Z A_1} + m_2 \ov{Z A_2}
    + \ldots +
    m_n \ov{Z A_n}
=
    \ov{0}
\)
и~$m = m_1 + m_2 + \ldots + m_n$.

При решении задач будем считать, что сумма всех масс не~равна нулю.

\claim{Правило рычага}
Для двух материальных точек $m_1 A_1$ и~$m_2 A_2$ их~центр масс~--- это
материальная точка $(m_1 + m_2) Z$, где $Z$~--- такая точка прямой $A_1 A_2$,
для которой $m_1 \cdot \ov{A_1 Z} = m_2 \cdot \ov{Z A_2}$.

\claim{Основная теорема}
Если $Z$~--- центр масс системы материальных точек
$m_1 M_1$, $\ldots$, $m_n M_n$, причем $m_1 + \ldots + m_n \neq 0$,
то~для любой точки $O$ выполняется равенство
\[
    \ov{OZ}
=
    \frac{
        m_1 \ov{O M_1} + m_2 \ov{O M_2} + \ldots + m_n \ov{O M_n}
    }{
        m_1 + m_2 + \ldots + m_n
    }
\;.\]
И обратно, если хотя~бы для одной точки $O$ выполнено это равенство, то~$Z$~---
центр масс системы.

\claim{Существование и~единственность}
Центр масс системы точек существует и~единственен.

\claim{Правило группировки}
Если часть точек системы заменить их~центром масс, то~центр масс системы
не~изменится.

\claim{Упражнение 1}
Три медианы треугольника пересекаются в~одной точке и~делятся этой точкой
в~отношении $2 : 1$, считая от~вершины.

\claim{Упражнение 2}
Докажите, что середины диагоналей, а~также точка пересечения отрезков,
соединяющих середины противоположных сторон четырехугольника, лежат на~одной
прямой.

\claim{Упражнение 3}
Докажите, что можно расставить в~вершины треугольника положительные массы так,
чтобы их~центр масс попал в~заданную внутреннюю треугольника точку.

\begin{problems}

\item
Пусть $P$~--- середина медианы $AM$ треугольника $ABC$.
Прямая $BP$ пересекает $AC$ в~точке $E$.
В~каком отношении отрезок $AC$ делится точкой $E$?

\item
Какие массы нужно расположить в~трех вершинах параллелограмма, чтобы их~центр
масс оказался в~четвертой вершине?

\end{problems}

\definition
\emph{Чевианой} называется отрезок, соединяющий вершину треугольника с~точкой
на~противолежащей стороне.

\begin{problems}

\item\emph{Теорема Чевы.}
На~сторонах $AB$, $BC$ и~$AC$ треугольника $ABC$ взяты точки
$C_1$, $A_1$ и~$B_1$ соответственно.
Докажите, что чевианы $A A_1$, $B B_1$, $C C_1$ пересекаются в~одной точке
тогда и~только тогда, когда
\(
    \frac{A C_1}{C_1 B} \cdot \frac{B A_1}{A_1 C} \cdot \frac{C B_1}{B_1 A}
=
    1
\).

\item\emph{Точка Жергонна.}
Пусть вписанная в~треугольник окружность касается сторон $BC$, $CA$, $AB$
в~точках $A_1$, $B_1$, $C_1$ соответственно.
Докажите, что чевианы $A A_1$, $B B_1$, $C C_1$ пересекаются в~одной точке.

\item\emph{Теорема ван Обеля.}
Чевианы $A A_1$ , $B B_1$ и~$C C_1$ треугольника $ABC$ пересекаются в~одной
точке $K$.
Докажите, что $A K / K A_1 =A B_1 / B_1 C + A C_1 / C_1 B$.

\item\emph{Теорема Менелая.}
На~сторонах $AB$, $BC$ и~продолжении стороны $AC$ за~точку $C$ треугольника
$ABC$ взяты точки $C_1$, $A_1$ и~$B_1$ соответственно.
Докажите, что $A_1, B_1, C_1$ лежат на~одной прямой тогда и~только тогда, когда
\(
    \frac{A C_1}{C_1 B} \cdot \frac{B A_1}{A_1 C} \cdot \frac{C B_1}{B_1 A}
=
    1
\).

\item
Старый пират зарыл клад на~острове среди $20$ деревьев.
После этого он~написал завещание, в~котором указал, как искать клад: надо
встать к~первому дереву, пройти половину расстояния до~второго дерева, затем
повернуть к~третьему и~пройти треть расстояния до~него, затем повернуть
к~четвертому и~пройти четверть расстояния до~него, и~т.~д., наконец, повернуть
к~двадцатому и~пройти двадцатую часть расстояния до~него.
К~сожалению, пират забыл указать, как занумерованы деревья!
Сколько разных ям необходимо выкопать потомкам пирата, чтобы все-таки найти
клад?

\item
На~окружности дано $n$ точек одинаковой массы.
Через центр масс любых $n - 2$ точек проводится прямая, перпендикулярная хорде,
соединяющей 2 оставшиеся точки.
Докажите, что все такие прямые пересекаются в~одной точке.

\end{problems}

\endgroup % \let\ov\overrightarrow

