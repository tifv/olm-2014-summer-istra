% $date: 2014-06-18

% $timetable:
%   g10:
%     2014-06-18:
%     - 2

\section*{Центр масс, часть 3}

% $authors:
% - Андрей Меньщиков

\begingroup\let\ov\overrightarrow

\begin{problems}

\item
Пусть $O$~--- произвольная точка внутри треугольника $ABC$.
Докажите, что
\[
    S_{BOC} \cdot \ov{OA} + S_{AOC} \cdot \ov{OB} + S_{AOB} \cdot \ov{OC}
=
    \ov{0}
.\]

\item
На~сторонах $BC$ и~$CD$ параллелограмма $ABCD$ взяты точки $K$ и~$L$ так, что
$BK : KC = CL : LD$.
Докажите, что точка пересечения медиан треугольника $AKL$ лежит на~диагонали
$BD$.

\item
В~окружность вписан четырехугольник $ABCD$, $M$~--- точка пересечения его
диагоналей, $Q$~--- середина стороны $CD$, $AD = a$, $BC = b$.
В~каком отношении прямая $MQ$ делит сторону $AB$?

\item
Через точку $P$, расположенную внутри параллелограмма $ABCD$, проведены прямые,
параллельные сторонам параллелограмма.
Они пересекают стороны $AB$, $BC$, $CD$ и~$AD$ в~точках $K$, $L$, $M$ и~$N$
соответственно.
Пусть $Q$~--- точка пересечения средних линий четырехугольника $KLMN$,
а~$S$~--- центр параллелограмма.
Докажите, что $Q$ лежит на~отрезке $PS$ и~найдите, в~каком отношении она его
делит.

\item
На~прямой $l$ отмечены $57$ точек $A_1$, $A_2$, $\ldots$, $A_{57}$, а~вне
неё~--- точка $P$.
Возможно~ли на~отрезках $P A_1$, $P A_2$, $\ldots$, $P A_{57}$ нарисовать
стрелки так, чтобы сумма полученных 57 векторов равнялась $\ov{0}$?

\item
В~вершинах правильного $n$-угольника расставлены числа: $(n-1)$ нулей и~одна
единица.
Разрешается увеличить на~1 все числа в~вершинах любого правильного
$k$-угольника, вписанного в~данный многоугольник.
Можно~ли такими операциями сделать все числа равными?

\item
Центрально симметричная фигура на~клетчатой бумаге состоит из~$n$ <<уголков>>
из~четырех клеток (в~виде буквы \textsf{Г}) и~$k$ прямоугольников размером
$1 \times 4$.
Докажите, что $n$ четно.

\item
В~квадрате $10 \times 10$ расставлены числа от~1 до~100 следующим образом:
в~первой строке (слева направо по~порядку) $1$, $2$, $\ldots$, $10$,
во~второй~--- $11$, $12$, $\ldots$, $20$,
и~т.~д.,
в~десятой~--- $91$, $92$, $\ldots$, $100$.
Разрешается взять любой прямоугольник $1 \times 3$ и~сделать следующую
операцию: прибавить к~крайним числам по~$1$, а~из~среднего отнять $2$, или
сделать обратную операцию.
Через некоторое время оказалось, что в~квадрате опять присутствуют все числа
от~$1$ до~$100$.
Докажите, что они расположены на~первоначальных местах.

\item
Дана пирамида $SABCD$, в~основании которой лежит параллелограмм $ABCD$.
Плоскость $\alpha$ пересекает ребра $SA$, $SB$, $SC$, $SD$
в~точках $K$, $L$, $M$, $N$ соответственно.
Докажите, что
\(
    \frac{AK}{KS} + \frac{CM}{MS}
=
    \frac{BL}{LS} + \frac{DN}{NS}
\).

\end{problems}

\endgroup % \let\ov\overrightarrow

