% $date: 2014-06-17

% $timetable:
%   g10:
%     2014-06-17:
%     - 1

\section*{Центр масс, часть 2}

% $authors:
% - Андрей Меньщиков

\begin{problems}

\item
В~четырехугольнике $ABCD$ точка $E$~--- середина стороны $AB$,
а~точка $K$~--- середина стороны $CD$.
Докажите, что середины отрезков $AK$, $CE$, $BK$ и~$DE$ являются вершинами
параллелограмма.

\item
Пусть $M$, $N$ и~$P$~--- точки, расположенные на~сторонах треугольника $ABC$
и~делящие эти стороны в~одинаковых отношениях
(т.~е. $\frac{AM}{MB} = \frac{BN}{NC} = \frac{CP}{PA}$).
Докажите, что точка пересечения медиан треугольника $MNP$ совпадает с~точкой
пересечения медиан треугольника $ABC$.

\item
На~сторонах $AB$, $BC$ и~$AC$ треугольника $ABC$ взяты точки $K$, $L$ и~$M$
соответственно так, что $AK : KB = 2 : 1$, $BL : LC = CM : AM = 1 : 2$.
$ML$ и~$CK$ пересекаются в~точке $P$.
Найдите, в~каком отношении точка $P$ делит отрезки $CK$ и~$ML$.

\item
На~сторонах $AB$, $BC$, $CD$ и~$AD$ выпуклого четырехугольника $ABCD$ взяты
точки $K$, $L$, $M$ и~$N$ соответственно, причем $AK : KB = DM : MC = p$,
$BL : LC = AN : ND = q$.
Пусть точка $P$~--- точка пересечения отрезков $KM$ и~$LN$.
Найдите, в~каком отношении точка $P$ делит каждый из~отрезков $KM$ и~$LN$.

\item
На~биссектрисе угла $\angle{A}$ неравнобедренного треугольника $ABC$ выбрана
точка $K$.
Прямые $BK$ и~$CK$ пересекают стороны $AC$ и~$AB$ в~точках $L$ и~$M$.
Докажите, что прямая $LM$ проходит через основание биссектрисы внешнего угла
$\angle A$ треугольника.

\item\emph{Точка Нагеля.}
Докажите, что прямые, соединяющие вершины треугольника с~точками касания
вневписанных окружностей с~соответствующими сторонами, пересекаются в~одной
точке.

\item
Дан треугольник со~сторонами $a$, $b$ и~$c$.
Какие массы нужно поместить в~его вершины, чтобы центром масс получившейся
системы оказался
\\
\sbp центр вписанной окружности
\\
\sbp центр вневписанной окружности, касающейся стороны $b$
\\
\sbp точка Жергонна
\qquad
\sbp точка Нагеля
\\
данного треугольника?

\item
Пусть точки $A_1$, $B_1$, $C_1$~--- середины сторон $BC$, $AC$, $AB$
треугольника $ABC$.
Докажите, что центр вписанной окружности треугольника $ABC$ совпадает с~точкой
Нагеля треугольника $A_1 B_1 C_1$.

\item
Пусть $I$~--- центр вписанной окружности треугольника $ABC$,
$B_1$~--- точка касания его вневписанной окружности со~стороной $AC$,
и~$M$~--- середина стороны $AC$.
Докажите, что $B B_1 \parallel I M$.

\item
В~треугольнике $ABC$ отмечены
$M$~--- точка пересечения медиан,
$I$~--- центр вписанной окружности,
$O$~--- центр описанной окружности,
$H$~--- точка пересечения высот,
$N$~--- точка Нагеля,
и~$S$~--- центр окружности, вписанной в~треугольник, образованный средними
линиями треугольника $ABC$.
\\
\sbp
Докажите, что точки $N$, $M$, $I$ и~$S$ лежат на~одной прямой, причем
$MN = 2 \, IM$ и~$IS = SN$.
\\
\sbp
Докажите, что $HN \parallel OI$, причем $HN = 2 \, OI$.

\end{problems}

