% $date: 2014-06-25

% $timetable:
%   g11:
%     2014-06-25:
%     - 1
%     2014-06-26:
%     - 1
%     2014-06-27:
%     - 1

\section*{Поляры}

% $authors:
% - Андрей Кушнир

\begin{enumerate}

\item
Докажите, что точки пересечения касательных в~вершинах треугольника к~описанной
окружности этого треугольника с~противоположными сторонами лежат на~одной
прямой.

\item\emph{Теорема Брианшона для четырехугольника.}
Докажите, что в~описанном четырехугольнике точка пересечения хорд, соединяющих
точки касания вписанной окружности с~противоположными сторонами, совпадает
с~точкой пересечения диагоналей.

\item
Вписанная окружность с~центром $I$ касается сторон $BC$, $CA$, $AB$
треугольника $ABC$ в~точках $A_1$, $B_1$, $C_1$.
Прямые $B_1 C_1$ и~$BC$ пересекаются в~точке $A_2$.
Докажите, что $A_2 I \perp A A_1$.

\item
В~условиях предыдущей задачи докажите, что перпендикуляр из~$I$ на~$A A_2$
проходит через точку пересечения прямых $A A_1$ и~$B_1 C_1$.

\item
Во~вписанном четырехугольнике отметили точки пересечения пар противоположных
сторон и~диагоналей.
Докажите, что ортоцентр треугольника с~вершинами в~отмеченных точках совпадает
с~центром описанной окружности четырехугольника.

\item
Окружность $\omega_2$ проходит через центр $O$ окружности $\omega_1$
и~пересекает её~в~точках $M$, $N$.
На~дуге $\omega_2$, лежащей вне $\omega_1$, берется произвольная точка $L$.
$OL$ и~$MN$ пересекаются в~точке $K$.
Через $K$ проводится хорда $XY$ окружности $\omega_1$.
Докажите, что $LK$~--- биссектриса угла $XLY$.

\item
Вписанная окружность с~центром $I$ касается сторон $BC$, $CA$, $AB$
треугольника $ABC$ в~точках $A_1$, $B_1$, $C_1$.
Окружность с~центром $A$, проходящая через $B_1$ и~$C_1$, пересекает прямую,
проведённую сквозь~$A$ параллельно $BC$, в~точках $B_2$, $C_2$;
причем порядок троек точек $B$, $A_1$, $C$ и~$C_2$, $A$, $B_2$ на~параллельных
прямых один и~тот~же.
Докажите, что $B B_2$, $C C_2$, $B_1 C_1$, $A_1 I$ пересекаются в~одной точке.

\item
Докажите, что в~одновременно вписанном и~описанном четырехугольнике прямая,
проходящая через точки пересечения пар противоположных сторон, перпендикулярна
прямой, соединяющей центры вписанной и~описанной окружностей.

\item
$A_1$~--- такая точка на~прямой $BC$, что $A A_1$ касается описанной окружности
треугольника $ABC$.
Из~$A_1$ проводятся касательные $AX$ и~$AY$ к~окружности девяти точек
треугольника $ABC$ ($X$ и~$Y$~--- точки касания).
Докажите, что $XY$ проходит через $A$.

\end{enumerate}

