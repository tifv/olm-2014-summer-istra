% $date: 2014-06-25

% $timetable:
%   g9:
%     2014-06-25:
%     - 2
%     2014-06-26:
%     - 2

\section*{Комбинаторная геометрия}

% $authors:
% - Фёдор Ивлев

\begin{problems}

\item
На~плоскости нарисовано несколько многоугольников, каждые два из~которых имеют
общую точку.
Докажите, что найдется прямая, пересекающая все эти многоугольники.
%http://problems.ru/view_problem_details_new.php?id=3494

\item
Существует~ли такой выпуклый (то~есть со~всеми углами меньше $180^\circ$)
пятиугольник $ABCDE$, что все углы
$ABD$, $BCE$, $CDA$, $DEB$ и $EAC$~--- тупые?
%(комбинаторная геометрия) К.Кноп ВМО окружной тур 95.4.8.3

\item
У~правильного $2011$-угольника отмечены $64$ вершины.
Докажите, что существует трапеция с~вершинами в~отмеченных точках.
%ММО 1981.9.5

\item
Каждый из~трех синих квадратов пересекается с~каждым из~трех красных.
Верно~ли, что какие-то два одноцветных квадрата тоже пересекаются?
%http://problems.ru/view_problem_details_new.php?id=35629

\item
Единичный квадрат разбит на~конечное число квадратиков
(размеры которых могут различаться).
Может~ли сумма периметров квадратиков, пересекающихся с~главной диагональю,
быть больше 1993?
%ММО 1993.11.2

\end{problems}

\subsection*{И немного про выпуклость\ldots}

\definition
Фигура называется \emph{выпуклой}, если вместе с~любыми двумя своими
точками~$A$ и~$B$ содержит весь отрезок~$AB$.

\begin{problems}

\item
Докажите, что у~выпуклого многоугольника все углы меньше $180^\circ$.

\item
Докажите, что прямая, содержащая сторону выпуклого многоугольника,
не~пересекает этот многоугольник по~внутренним точкам.
Докажите, что выпуклый многоугольник является пересечением некоторого
количества полуплоскостей.

\item
На~плоскости дано пять точек, причем никакие три из~них не~лежат на~одной
прямой.
Докажите, что четыре из~этих точек расположены в~вершинах выпуклого
четырехугольника.

\item\emph{Теорема Хана\,--\,Банаха.}
Докажите, что любые два выпуклых непересекающихся многоугольника можно
разделить прямой.

\end{problems}

