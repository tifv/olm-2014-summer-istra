% $date: 2014-06-25

% $timetable:
%   g1011:
%     2014-06-25:
%     - 1

\section*{Немного об Эйлере}

% $authors:
% - Алексей Доледенок

\begin{problems}

\item
\sbp
Пусть $O$~--- центр описанной окружности треугольника $ABC$,
$H$~--- его ортоцентр,
$A_1$~--- середина стороны~$BC$.
Докажите, что $AH = 2 O A_1$.
\\
\sbp
Докажите, что в~треугольнике $ABC$ точка пересечения высот, точка пересечения
медиан и~центр описанной окружности лежат на~одной прямой, причем $HM = 2 MO$.

\item
Дан вписанный четырехугольник $ABCD$.
$H_A$ и~$H_B$~--- ортоцентры треугольников $BCD$ и~$ACD$ соответственно.
Докажите, что $H_A H_B = AB$.

\item\label{problem:euler-line/g1011/quadrilateral anticenter}%
В~условиях предыдущей задачи аналогично определим точки $H_C$ и~$H_D$.
Докажите, что отрезки $A H_A$, $B H_B$, $C H_C$ и~$D H_D$ пересекаются
в~одной точке~$H$.

\item
Из~середины каждой стороны вписанного четырехугольника $ABCD$ опустили
перпендикуляр на~противоположную сторону.
Докажите, что эти четыре прямых пересекаются в~одной точке.
Что это за~точка?

\item\label{problem:euler-line/g1011/quadrilateral vertex centroid}%
Докажите, что прямые, соединяющие середины противоположных сторон
четырехугольника $ABCD$, и~прямая, соединяющая середины диагоналей,
пересекаются в~одной точке~$M$.

\item
Докажите, что точка~$H$
из~задачи~\ref{problem:euler-line/g1011/quadrilateral anticenter},
точка~$M$ из~задачи~\ref{problem:euler-line/g1011/quadrilateral vertex centroid},
а~также центр $O$ описанной окружности вписанного четырехугольника $ABCD$,
лежат на~одной прямой, причем $OM = MH$.

\end{problems}

