% $date: 2014-06-25

% $timetable:
%   g10:
%     2014-06-25:
%     - 2

\section*{Немного об Эйлере}

% $authors:
% - Алексей Доледенок

\begin{problems}

\item
Пусть $O$~--- центр описанной окружности треугольника $ABC$,
$H$~--- его ортоцентр,
$A_1$~--- середина стороны~$BC$.
Докажите, что $AH = 2 O A_1$.

\item
Дан вписанный четырехугольник $ABCD$.
$H_A$ и~$H_B$~--- ортоцентры треугольников $BCD$ и~$ACD$ соответственно.
Докажите, что $H_A H_B = AB$.

\item\label{problem:euler-line/g10/quadrilateral anticenter}%
В~условиях предыдущей задачи аналогично определим точки $H_C$ и~$H_D$.
Докажите, что отрезки $A H_A$, $B H_B$, $C H_C$ и~$D H_D$ пересекаются
в~одной точке~$H$.

\item
Обозначим за~$l_a$ прямую Симсона точки~$A$ относительно треугольника $BCD$.
Аналогично определим $l_b$, $l_c$ и~$l_d$.
Докажите, что эти прямые пересекаются в~одной точке.
Что это за~точка?

\item
Из~середины каждой стороны вписанного четырехугольника $ABCD$ опустили
перпендикуляр на~противоположную сторону.
Докажите, что эти четыре прямых пересекаются в~одной точке.
Что это за~точка?

\item
Пусть прямые, соединяющие середины противоположных сторон четырехугольника
$ABCD$, и~прямая, соединяющая середины диагоналей, пересекаются в~точке~$M$.
Докажите, что точка~$H$
из~задачи~\ref{problem:euler-line/g10/quadrilateral anticenter},
точка~$M$, а~также центр $O$ описанной окружности вписанного четырехугольника
$ABCD$, лежат на~одной прямой, причем $OM = MH$.

\item
Докажите, что центры окружностей Эйлера треугольников
$ABC$, $ABD$, $ACD$, $BCD$ лежат на~одной окружности.

\item
Докажите, что окружности Эйлера треугольников $ABC$, $ABD$, $ACD$, $BCD$
пересекаются в~точке~$H$.

\end{problems}

