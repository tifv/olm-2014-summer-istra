% $date: 2014-06-22

% $timetable:
%   g78r2:
%     2014-06-22:
%     - 3
%     2014-06-24:
%     - 1
%     2014-06-26:
%     - 2

\section*{Комбинаторная геометрия}

% $authors:
% - Андрей Кушнир

\begin{problems}

\item
На~дне морском сидят $n$~крабов.
После прилива они поменялись местами.
Известно, что расстояние между любыми двумя крабами не~увеличилось.
Докажите, что на~самом деле расстояние между любыми двумя крабами не~изменилось.

\item
Можно~ли двумя прямыми разрезать выпуклый четырёхугольник на~шесть частей?
А~невыпуклый четырёхугольник?

\item
Докажите, что среди любых пяти точек \emph{общего положения}
(т.~е. таких точек, что никакие три из~них не~лежат на~одной прямой) можно выбрать
четыре, являющиеся вершинами выпуклого четырёхугольника.

\item
На~плоскости проведено $n$~прямых.
Докажите, что части, на~которые эти прямые делят плоскость, можно раскрасить в~два
цвета правильным образом
(т.~е. так, чтобы соседние части были окрашены в~разные цвета).

\item
На~плоскости даны три шайбы.
Хоккеист выбирает одну из~шайб и~бьёт по~ней так, чтобы она пролетела между двумя
другими.
Может ли так оказаться, что после $25$ бросков шайбы окажутся на~исходных позициях?

\item
Из~точки плоскости выпущено $n$~лучей.
Известно, что угол между любыми двумя из~них меньше $120^{\circ}$.
Докажите, что из~них можно выбрать два, так чтобы угол, образованный этими двумя
лучами, содержал все остальные лучи.

\item
\sbp
Существует~ли замкнутая шестизвенная ломаная, пересекающая каждое своё звено ровно один
раз?
\\
\sbp
При каких~$n$ существует замкнутая $n$-звенная ломаная, пересекающая каждое своё звено
ровно один раз?

\item
На~плоскости проведено $n$~прямых \emph{общего положения}, т.~е. никакие три из~этих
прямых не~пересекаются в~одной точке и~никакие две не~параллельны.
\\
\sbp
На~сколько частей разбита плоскость?
\\
\sbp
Сколько из~этих частей неограниченны?
\\
\sbp
Докажите, что одна из~неограниченных частей~--- угол.

\item
Докажите, что правильный треугольник со~стороной $3$ можно разбить на~$2014$
треугольников, все стороны которых больше~$1$.

\end{problems}

