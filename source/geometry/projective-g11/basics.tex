% $date: 2014-06-17

% $timetable:
%   g11:
%     2014-06-17:
%     - 1
%     - 2

\section*{Проективная геометрия}

% $authors:
% - Андрей Кушнир

\begin{problems}

\item
На недорисованной картине изображена железная дорога, проложенная через равнину
и уходящая за горизонт, а также две рядом лежащие шпалы, параллельные линии
горизонта. Как с помощью линейки определить, где надо рисовать третью шпалу?

\item\emph{Теорема Дезарга.}
Докажите, что прямые $A_1 A_2$, $B_1 B_2$, $C_1 C_2$
пересекаются в одной точке тогда и только тогда, когда точки пересечения прямых
$A_1 B_1$ и $A_2 B_2$, $B_1 C_1$ и $B_2 B C_2$, $C_1 A_1$ и $C_2 A_2$ лежат на
одной прямой (считайте, что треугольники $A_1 B_1 C_1$ и $A_2 B_2 C_2$
невырожденные).

\item\emph{Теорема Паппа.}
Точки $A_1$, $B_1$, $C_1$ лежат на одной прямой;
$A_2$, $B_2$, $C_2$ лежат на другой прямой.
Докажите, что точки пересечения пар прямых $A_1 B_2$ и $A_2 B_1$,
$B_1 C_2$ и $B_2 C_1$, $C_1 A_2$ и $C_2 A_1$ лежат на одной прямой.

\item\emph{Теорема о трижды перспективных треугольниках.}
Два треугольника назовем перспективными, если прямые, соединяющие их
соответственные вершины пересекаются в одной точке.
Известно, что треугольники $A B C$ и $A' B' C'$ перспективны и треугольники
$A B C$ и $B' C' A'$ перспективны.
Докажите, что треугольники $A B C$ и $C' A' B'$ тоже перспективны.

\item
Даны две пронумерованные четверки точек общего положения.
Докажите, что проективное преобразование, переводящее одну четверку в другую
\\
\sbp существует;
\quad
\sbp единственно.

\item
\sbp
На плоскости даны точка и две прямые, пересекающиеся в луже.
Постройте одной линейкой прямую, проходящую через точку пересечения прямых и
данную точку.
\\
\sbp
На плоскости даны две точки и прямая.
К точкам нельзя приложить линейку из-за лужи между ними.
Постройте линейкой точку пересечения данной прямой и прямой, проходящей через
данные точки.

\item
Докажите, что с помощью одной линейки невозможно разделить данный отрезок
пополам.

\item
На сторонах $AB$, $AC$ треугольника $ABC$ отмечены точки $X$, $Y$;
$XY \parallel BC$.
Чевианы $AP$, $AQ$ треугольника $ABC$ пересекают отрезок $XY$ в точках $M$, $N$
соответственно.
Докажите, что прямая, соединяющая точки пересечения пар прямых $PX$ и $CN$,
$QY$ и $BM$ проходит через вершину $A$.

\item
Внутри треугольника $ABC$ отмечена точка $P$, через которую проведены хорды
треугольника $A_1 B_2$, $B_1 C_2$, $C_1 A_2$
(предполагается, что вы догадаетесь о том, что такое хорды треугольника и на
каких сторонах какие точки лежат).
Оказалось, что прямые $A P$, $C_2 A_1$, $B_1 A_2$ пересекаются в одной точке.
Докажите, что точки пересечения пар прямых $A_1 C_2$ и $CP$, $A_2 B_1$ и $BP$
попадут на прямую $B_2 C_1$.

\item
На сторонах $AC$ и $AB$ треугольника $ABC$ отмечены точки $B'$ и $C'$
соответственно,
а на стороне $BC$ отмечены точки $A_A$, $A_P$, $A_B$, $A_C$.
Прямые $B B'$, $C C'$ пересекаются в точке $P$.
Известно, что $A A_A$, $P A_P$, $B' A_B$ и $C' A_C$ пересекаются в одной точке.
Докажите, то $A A_P$, $P A_A$, $B' A_C$, $C' A_B$ также пересекаются в одной
точке.

\end{problems}

