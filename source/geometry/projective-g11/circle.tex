% $date: 2014-06-20

% $timetable:
%   g11:
%     2014-06-20:
%     - 2

\section*{Проективные преобразования. Окружность}

% $authors:
% - Андрей Кушнир

\begin{problems}

\item
Внутри окружности даны точки $P$, $O$.
На окружности отмечают произвольную точку $X_0$.
Затем проводят хорды $X_0 X_1$, $X_1 X_2$, $X_2 X_3$ через точки $O$, $P$, $O$
соответственно.
Докажите, что все хорды $X_0 X_3$ проходят через одну точку, не зависящую от
выбора $X_0$.

\item\emph{Задача о бабочке.}
$K$~--- середина хорды $MN$ окружности.
Через неё проводятся еще две хорды: $AC$, $BD$.
Хорды $AB$, $CD$ пересекают отрезок $MN$ в точках $X$, $Y$.
Докажите, что $KX = KY$.

\item\emph{Теорема Паскаля.}
Докажите, что точки пересечения пар противоположных сторон вписанного в
окружность шестиугольника лежат на одной прямой.

\item
Четыре прямые, проведённые из точки $P$ пересекают окружность в парах точек:
$A_1$ и $A_2$, $B_1$ и $B_2$, $C_1$ и $C_2$, $D_1$ и $D_2$.
Докажите, что двойные отношения четвёрок точек $A_1$, $B_1$, $C_1$, $D_1$ и
$A_2$, $B_2$, $C_2$, $D_2$ на окружности равны.

\item
На окружности в указанном порядке отметили шесть точек:
$A_1$, $A_2$, $B_1$, $B_2$, $C_1$, $C_2$.
Вершины треугольника, сторонами которого являются прямые $A_1 A_2$, $B_1 B_2$,
$C_1 C_2$, обозначили $A$, $B$, $C$ (где какая вершина угадайте).
Касательные к парам точек $A_1$ и $A_2$, $B_1$ и $B_2$, $C_1$ и $C_2$
пересекаются в точках $A'$, $B'$, $C'$ соответственно.
Докажите, что $A A'$, $B B'$, $C C'$ пересекаются в одной точке.

\item
Окружность пересекает стороны треугольника как в предыдущей задаче.
$A''$, $B''$, $C''$~--- точки пересечения $B B_1$ и $C C_2$, $C C_1$ и $A A_2$,
$A A_1$ и $BB_2$ соответственно.
Докажите, что $A A''$, $B B''$, $C C''$ пересекаются в одной точке.

\item
Вписанная в четырехугольник $ABCD$ окружность касается сторон
$AB$, $BC$, $CD$, $DA$ в точках $K$, $L$, $M$, $N$ соответственно.
Докажите, что точки пересечения пар прямых $BC$ и $KM$, $AL$ и $KN$, $DL$ и
$MN$ лежат на одной прямой.

\item
В треугольнике $ABC$ проведены чевианы $A A_1$, $B B_1$, $C C_1$,
пересекающиеся в одной точке.
Из точек $A_1$, $B_1$, $C_1$ проведены касательные ко вписанной окружности,
отличные от сторон треугольника.
Точки касания обозначили $A_2$, $B_2$, $C_2$.
Докажите, что $A A_2$, $B B_2$, $C C_2$ пересекаются в одной точке.

\item
На плоскости нарисована парабола.
С помощью циркуля и линейки отметьте её вершину.

\end{problems}

