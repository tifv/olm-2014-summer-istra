% $date: 2014-06-15

% $timetable:
%   g1011:
%     2014-06-15:
%     - 3

\section*{Геометрия. Листик первый}

% $authors:
% - Андрей Кушнир

\begin{problems}

\item
Внутри остроугольного треугольника $ABC$ отмечена точка $P$, а на его стороне
$BC$~--- точка $A_1$.
Описанные окружности треугольников $B A_1 P$ и $C A_1 P$ пересекают стороны
$AB$ и $AC$ в точках $C_1$, $B_1$ соответственно.
Докажите, что четырехугольник $A B_1 P C_1$ - вписанный.

\item
Окружности $\omega_1$, $\omega_2$ пересекаются в точках $A$, $B$.
На дуге окружности $\omega_1$, лежащей вне $\omega_2$, отмечена точка $X$.
Прямые $XA$, $XB$ вторично пересекают $\omega_2$ в точках $P$, $Q$.
Докажите, что прямая $PQ$ параллельна касательной, проведённой к окружности
$\omega_1$ в точке $X$.

\item
Из точки $X$, лежащей на высоте $AH$ остроугольного треугольника $ABC$ опущены
перпендикуляры $X C_1$, $X B_1$ на стороны $AB$, $AC$.
Докажите, что точки $B$, $C$, $B_1$, $C_1$ лежат на одной окружности.

\item
$PA$~--- касательная к описанной окружности треугольника $ABC$.
Из точки $P$ опущены перпендикуляры $P B_1$, $P C_1$ на прямые $AC$,
$AB$ соответственно.
Докажите, что прямые $BC$ и $B_1 C_1$ перпендикулярны.

\item
Вписанная окружность касается стороны $BC$ в точке $P$;
$PR$~--- диаметр вписанной окружности.
Вневписанная окружность касается стороны $BC$ в точке $Q$.
Докажите, что $A$, $R$, $Q$ лежат на одной прямой.

\item
$O$~--- центр описанной окружности остроугольного треугольника $ABC$,
$H$~--- его ортоцентр.
Докажите, что расстояние от точки $O$ до стороны $BC$ вдвое меньше длины
отрезка $AH$.

\item
В угол вписаны две окружности, касающиеся друг друга.
Докажите, что в четырехугольник, вершинами которого являются точки касания
окружностей со сторонами угла, можно вписать окружность.

\item
Дан описанный четырехугольник $ABCD$.
Докажите, что центры вписанных окружностей треугольников
$ABC$, $BCD$, $CDA$, $DAB$ являются вершинами прямоугольника.

\end{problems}

