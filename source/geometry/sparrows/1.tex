% $date: 2014-06-21

% $timetable:
%   g11:
%     2014-06-21:
%     - 1
%   g10:
%     2014-06-21:
%     - 2

\section*{Воробьями по пушкам и окрестности}

% $authors:
% - Александр Полянский

% $matter[g10,-guard]:
% - verbatim: \begingroup \let\ifgroupten\iftrue \let\ifgroupeleven\iffalse
% - .[guard]
% - verbatim: \endgroup% \let\ifgroupten \let\ifgroupeleven

% $delegate$groups: false

% $matter[g11,-guard]:
% - verbatim: \begingroup \let\ifgroupeleven\iftrue \let\ifgroupten\iffalse
% - .[guard]
% - verbatim: \endgroup% \let\ifgroupten \let\ifgroupeleven

% $matter[-g10,-g11,-guard]:
% - verbatim: \begingroup \def\jeolmgroupname{Тигры, Зубры}
% - .[guard]
% - verbatim: \endgroup% \def\jeolmgroupname

\begingroup
\providecommand{\ifgroupten}{\iftrue}
\providecommand{\ifgroupeleven}{\iftrue}

\begin{problems}

\item\textbf{Воробей 1.}
Задан неравнобедренный треугольник $ABC$, на его сторонах $AB$ и $BC$ выбраны
точки $C_0$ и $A_0$ соответственно,
$B_1$~--- середина дуги $ABC$ описанной окружности треугольника $ABC$.
Докажите, что равенство $A C_0 = C A_0$ выполняется тогда и только тогда, когда
точки $A_0$, $C_0$, $B_1$, $B$ лежат на одной окружности.

\item\textbf{Воробей 2.}
Задан треугольник $ABC$, на его сторонах $AB$ и $BC$ выбраны точки $C_0$ и
$A_0$ соответственно,
$I$~--- центр вписанной окружности в $ABC$.
Докажите, что окружность, описанная около треугольника $A_0 B C_0$, проходит
через $I$ тогда и только тогда, когда $A C_0 + C A_0 = AC$.

\item
Дан неравнобедренный треугольник $ABC$ ($AB < BC$),
$B_1$~--- середина дуги $ABC$ описанной окружности треугольника $ABC$,
$M$~--- середина стороны $AC$.
Докажите, что центры $I_A$ и $I_C$ окружностей, вписанных в треугольники $AMB$
и $CMB$, и точки $B$ и $B_1$ лежат на одной окружности.

\item
Дан неравнобедренный треугольник $ABC$ ($AB < BC$),
$B_1$~--- середина дуги $ABC$ описанной окружности $w$ треугольника $ABC$,
$I$~--- центр вписанной в $ABC$ окружности,
$M$~--- середина стороны $AC$.
Докажите, что $\angle I B_1 B = \angle IMA$.

\item
Точки $A_1$, $B_1$, $C_1$ выбраны на сторонах $BC$, $CA$ и $AB$ треугольника
$ABC$ следующим образом:
$A B_1 - A C_1 = C A_1 - C B_1 = B C_1 - B A_1$.
Пусть $I_A$, $I_B$ и $I_C$~--- центры окружностей, вписанных в треугольники
$A B_1 C_1, A_1 B C_1$ и $A_1 B_1 C$.
Докажите, что центр описанной окружности треугольника $I_A I_B I_C$,
совпадает с центром вписанной окружности треугольника $ABC$.

\item
Точки $A_1$, $B_1$, $C_1$ выбраны на сторонах $BC$, $CA$ и $AB$ треугольника
$ABC$ следующим образом: $A B_1 - A C_1 = C A_1 - C B_1 = B C_1 - B A_1$.
Пусть $O_A$, $O_B$ и $O_C$~--- центры окружностей, описанных около
треугольников $A B_1 C_1$, $A_1 B C_1$ и $A_1 B_1 C$.
Докажите, что центр вписанной окружности треугольника $O_A O_B O_C$,
совпадает с центром вписанной окружности треугольника $ABC$.

\item
Точки $E$ и $F$~--- середины большой дуги $AC$ и малой дуги $AC$ описанной около
остроугольного неравнобедренного треугольника $ABC$ ($AB < BC$).
Пусть $G$~--- проекция $E$ на $BC$.
Докажите, что описанная окружность около треугольника $ABG$ проходит через
середину отрезка $BF$.

\item\label{problem:sparrow-3}\textbf{Воробей 3.}
Точки $X$ и $Y$ движутся с постоянными скоростями (не обязательно равными)
по двум прямым, пересекающимся в точке $O$.
Докажите, что окружность, описанная около треугольника $XYO$, проходит через 2
фиксированные точки $O$ и $Z$, где $Z$ является центром поворотной гомотетии,
переводящий местоположения точек $X$ в местоположения точек $Y$.

\end{problems}

\observation
Осмыслите первые две задачи с помощью задачи \ref{problem:sparrow-3}.

\begin{problems}

\ifgroupten
\item
Дан треугольник $ABC$ и окружность с центром $O$, проходящая через вершины $A$
и $C$ и повторно пересекающая отрезки $AB$ и $BC$ в различных точках $K$ и $N$
соответственно.
Окружности, описанные около треугольников $ABC$ и $KBN$, имеют ровно две общие
точки $B$ и $M$.
Докажите, что угол $OMB$~--- прямой.
\fi

\item
Пусть на дуге $BC$ (не содержащей точки $A$) описанной окружности $w$
треугольника $ABC$ выбрана точка $E$, а на стороне $AC$~--- точка $F$.
Докажите, что через луч $EF$~--- биссектриса угла $AEC$ тогда и только тогда,
когда $\angle IEB = \angle IFA$, где $I$~--- центр вписанной окружности
треугольника $ABC$.

\end{problems}

\observation
Эти удивительно важное свойство!
Особенно запомните ситуацию, когда $F$ совпадает с точкой касания вписанной
окружности в треугольник $ABC$ со стороной $BC$.

\ifgroupeleven
\begin{problems}

\item
На дуге $AC$ описанной окружности треугольники $ABC$, не содержащей точку $B$,
выбрана точка $E$.
Докажите, что центры $I_a$ и $I_c$ вписанных окружностей треугольников $AEB$ и
$CEB$, точка $E$ и точка $T_b$ касания полувписанной
(соответствующей вершине $B$) и описанной окружностей лежат на одной
окружности.

\end{problems}
\fi

\endgroup% \ifgroupten \ifgroupeleven

