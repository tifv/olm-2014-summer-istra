% $date: 2014-06-22

% $timetable:
%   g11:
%     2014-06-22:
%     - 1
%   g10:
%     2014-06-22:
%     - 2

\section*{Воробьями по пушкам, продолжение}

% $authors:
% - Александр Полянский

% $matter[g10,-guard]:
% - verbatim: \begingroup \let\ifgroupten\iftrue \let\ifgroupeleven\iffalse
% - .[guard]
% - verbatim: \endgroup% \let\ifgroupten \let\ifgroupeleven

% $delegate$groups: false

% $matter[g11,-guard]:
% - verbatim: \begingroup \let\ifgroupeleven\iftrue \let\ifgroupten\iffalse
% - .[guard]
% - verbatim: \endgroup% \let\ifgroupten \let\ifgroupeleven

% $matter[-g10,-g11,-guard]:
% - verbatim: \begingroup \def\jeolmgroupname{Тигры, Зубры}
% - .[guard]
% - verbatim: \endgroup% \def\jeolmgroupname

\begingroup
\providecommand{\ifgroupten}{\iffalse}
\providecommand{\ifgroupeleven}{\iffalse}

\setproblem{11}

\ifgroupten
\setproblem{10}
\fi

\ifgroupeleven
\setproblem{10}
\fi

\begin{problems}

\item
На сторонах $AB$ и $BC$ треугольника $ABC$ выбраны точки $C_1$ и $A_1$.
Пусть $K$~--- середина $A_1 C_1$, а $I$~--- центр вписанной окружности в
треугольник $ABC$.
Оказалось, что четырехугольник $A_1 B C_1 I$ вписанный.
Докажите, что угол $AKC$ тупой.

\item
Пусть на стороне $AC$ выбрана точка $D$.
Обозначим через $I_A$ и $I_C$ центры вписанных окружностей в треугольники
$ABD$ и $CBD$, а через $B'$ точку касания вписанной окружности со
стороной $AC$.
Докажите, что угол $I_A B' I_C$~--- прямой.

\item
Пусть $A_0, B_0$ и $C_0$~--- точки касания вневписанных окружностей с
соответствующими сторонами треугольника $ABC$.
Описанные окружности треугольников $A_0 B_0 C, A B_0 C_0$ и $A_0 B C_0$
пересекают второй раз описанную окружность $w$ треугольника $ABC$ в~точках
$C_1$, $A_1$ и $B_1$ соответственно.
Докажите, что треугольник $A_1 B_1 C_1$ подобен треугольнику, образованному
точками касания вписанной окружности треугольника $ABC$ с его сторонами.

\item
Дан треугольник $ABC$ и окружность с центром $O$, проходящая через вершины $A$
и $C$ и повторно пересекающая отрезки $AB$ и $BC$ в различных точках $K$ и $N$
соответственно.
Окружности, описанные около треугольников $ABC$ и $KBN$, имеют ровно две общие
точки $B$ и $M$.
Докажите, что угол $OMB$~--- прямой.

\item
В остроугольном неравнобедренном треугольнике $ABC$ биссектриса острого угла
между высотами $A A_1$ и $C C_1$ пересекает стороны $AB$ и $BC$ в точках $P$ и
$Q$ соответственно.
Биссектриса угла $B$ пересекает отрезок, соединяющий ортоцентр треугольника
$ABC$ с серединой $AC$, в точке $R$.
Докажите, что точки $P$, $B$, $Q$ и $R$ лежат на одной окружности.

\item
Дан четырехугольник $ABCD$ такой, что $BC = AD$ и $BC$ не параллельна $AD$.
На сторонах $BC$ и $AD$ выбраны такие точки $E$ и $F$, что $BE = DF$.
Диагонали $AC$ и $BD$ пересекаются в точке $P$, прямые $BD$ и $EF$~--- в~точке
$Q$, прямые $EF$ и $AC$~--- в~точке $R$.
Докажите, что окружности, описанные около треугольников $PQR$ (при изменении
положений точек $E$ и $F$), проходят через одну точку, отличную от $P$.

\end{problems}

\endgroup% \ifgroupten \ifgroupeleven

