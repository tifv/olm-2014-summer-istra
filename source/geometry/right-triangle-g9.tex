% $date: 2014-06-15

% $timetable:
%   g9:
%     2014-06-15:
%     - 2
%     2014-06-16:
%     - 2

\section*{Серия 1, о прямоугольном треугольнике}

% $authors:
% - Фёдор Бахарев

В треугольнике $ABC$ с прямым углом $\angle C$ построены:
$CH$~--- высота,
$O$, $O_1$, $O_2$~--- центры окружностей, вписанных в треугольники
$ABC$, $ACH$ и $CBH$ соответственно,
$r$, $r_1$, $r_2$ --- их радиусы.
Прямая $O_1 O_2$ пересекает стороны $AC$ и $BC$ в точках $U$ и $V$
соответственно.
Прямые $C O_1$ и $C O_2$ пересекают сторону $AB$ в точках $P$ и $Q$
соответственно.
Точка $T$~--- основание перпендикуляра из точки $O$ на $AB$.
Докажите следующие утверждения.

\begin{problems}

\item
Треугольники $ACQ$ и $BCP$ равнобедренные.

\item
Точка $O$~--- ортоцентр (точка пересечения высот) треугольника $C O_1 O_2$.

\item
Треугольники $ACH$ и $BCH$ подобны $ABC$ с коэффициентами $k_1$ и $k_2$ такими,
что $k_1^2 + k_2^2 = 1$.

\item
Для радиусов вписанных окружностей выполнено равенство $r^2 = r_1^2 + r_2^2$.

\item
И еще одно волшебное наблюдение: $r + r_1 + r_2 = CH$.

\item
Точки $A$, $O_1$, $O_2$, $B$ лежат на одной окружности.

\item
Точки $A$, $P$, $O$, $C$, чудесным образом, тоже лежат на одной окружности.

\item
Описанные окружности треугольников $A C O_1$ и $B C O_2$ касаются в точке $C$
с общей касательной $OC$.

\item
Внезапно, $O_1 O_2 = C O$.
\qquad
\problem
А еще $CU = CV = CH$.

\item
Неожиданно $P O_2 \parallel A O$ и $Q O_1 \parallel B O$.
\qquad
\item
Треугольники $O_1 O_2 H$ и $ABC$ подобны.

\item
Как это ни странно, $\angle O_1 T O_2 = 90^\circ$.

\item
А вот и еще две параллельности: $T O_1 \parallel B C$, $T O_2 \parallel A C$.

\item
Аж пять отрезков равны: $O_1 T = O_2 T = O T = P T = Q T$.

\end{problems}

