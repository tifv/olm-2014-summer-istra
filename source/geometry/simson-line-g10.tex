% $date: 2014-06-24

% $timetable:
%   g10:
%     2014-06-24:
%     - 2

\section*{Прямая Симсона}

% $authors:
% - Алексей Доледенок

\begin{problems}

\item
Рассмотрим треугольник $ABC$ и~произвольную точку~$P$.
Пусть $A_1$, $B_1$ и~$C_1$~--- основания перпендикуляров из~точки~$P$ на~прямые
$BC$, $AC$ и~$AB$ соответственно.
\\
\sbp
Докажите, что если $P$ лежит на~описанной окружности треугольника $ABC$,
то~точки $A_1$, $B_1$ и~$C_1$ лежат на~одной прямой.
\\
\sbp
Докажите, что если точки $A_1$, $B_1$ и~$C_1$ лежат на~одной прямой,
то~точка~$P$ лежит на~описанной окружности треугольника $ABC$.

\item
\sbp
Даны точки $A$, $B$, $C$, лежащие на~одной прямой, и~точка~$P$ вне этой прямой.
Докажите, что центры описанных окружностей треугольников $ABP$, $ACP$, $BCP$
и~точка~$P$ лежат на~одной окружности.
\\
\sbp\emph{Точка Микеля.}
Четыре попарно пересекающиеся прямые образуют четыре треугольника.
Докажите, что описанные окружности этих треугольников пересекаются в~одной
точке.
\\
\sbp
Докажите, что центры описанных окружностей этих треугольников и~точка Микеля
лежат на~одной окружности.

\item
В~треугольнике $ABC$ провели биссектрису~$A A_1$, из~точки~$A_1$ опустили
перпендикуляры $A_1 B_1$ и~$A_1 C_1$ на~стороны $AC$ и~$AB$ соответственно.
На~отрезке~$B_1 C_1$ выбрана точка~$M$ так, что $M A_1 \perp BC$.
Докажите, что точка~$M$ лежит на~медиане треугольника $ABC$, проведенной
из~вершины~$A$.

\item
Окружность с~центром в~точке~$I$, вписанная в~треугольник $ABC$, касается
сторон $AB$ и~$BC$ в~точках $C_1$ и~$A_1$ соответственно.
Окружность, проходящая через точки $B$ и~$I$, пересекает стороны $AB$ и~$BC$
в~точках $M$ и~$N$.
Докажите, что середина отрезка~$MN$ лежит на~прямой~$A_1 C_1$.

\item
Дан вписанный четырехугольник $ABCD$ и~точка~$P$ на~его описанной окружности.
Докажите, что проекции точки~$P$ на~прямые Симсона треугольников
$ABC$, $ABD$, $ACD$ и~$BCD$ лежат на~одной прямой.

\item
Точка~$P$ движется по~описанной окружности треугольника $ABC$.
Докажите, что при этом прямая Симсона точки~$P$ относительно $ABC$
поворачивается на~угол, равный половине угловой величины дуги, пройденной $P$.

\item
Докажите, что прямые Симсона двух диаметрально противоположных точек описанной
окружности треугольника $ABC$ перпендикулярны, а~их~точка пересечения лежит
на~окружности Эйлера.

\item
Пусть $H$~--- ортоцентр треугольника $ABC$,
$P$~--- произвольная точка на~описанной окружности $ABC$.
Докажите, что прямая Симсона точки~$P$ делит отрезок~$PH$ пополам.

\item
Докажите, что на~описанной окружности треугольника существует ровно три точки
такие, что их~прямая Симсона касается окружности Эйлера, причем они образуют
равносторонний треугольник.
 
\end{problems}

