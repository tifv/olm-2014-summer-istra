% $date: 2014-06-24

% $timetable:
%   g78r1:
%     2014-06-24:
%     - 1
%     2014-06-25:
%     - 3

\section*{Геометрический разнобой}

% $authors:
% - Михаил Ягудин

\begin{problems}

\item
Четырехугольник $ABCD$ вписан в~окружность.
Обозначим середины дуг $AB$, $BC$, $CD$ и~$DE$, не~содержащих других вершин
четырехугольника, через $P$, $Q$, $R$ и~$S$ соответственно.
Докажите, что прямые $PR$ и~$QS$ перпендикулярны.

\item
Три равные окружности пересекаются в~одной точке.
Докажите, что треугольник с~вершинами в~остальных точках попарного пересечения
окружностей равен треугольнику с~вершинами в~центрах окружностей.

\item
Продолжения сторон $AB$ и~$CD$ вписанного четырехугольника~$ABCD$ пересекаются
в~точке~$P$, а~продолжения сторон~$BC$ и $AD$~--- в~точке~$Q$.
Докажите, что точки пересечения биссектрис углов~$AQB$ и~$BPC$ со~сторонами
четырехугольника являются вершинами ромба.
% Прасолов 2.43

\item
Отрезки, соединяющие некоторую внутреннюю точку выпуклого неравностороннего
$n$-угольника с~его вершинами, делят $n$-угольник на~$n$~равных треугольников.
При каком наименьшем~$n$ это возможно?
% Шарыгин 2007 3

\item
Внутри квадрата $ABCD$ взята точка~$P$ так, что
$\angle PBA = \angle PAB = 15^\circ$.
Докажите, что $CPD$~--- равносторонний треугольник.

\item
Две окружности пересекаются в~точках~$A$ и~$B$.
В~точке~$A$ к~обеим проведены касательные, пересекающие окружности
в~точках~$M$ и~$N$.
Прямые~$BM$ и~$BN$ пересекают окружности еще раз в~точках~$P$ и~$Q$
($P$~--- на~прямой~$BM$, $Q$~--- на~прямой~$BN$).
Докажите, что отрезки~$MP$ и~$NQ$ равны.
% КВАНТ М1442

\end{problems}

