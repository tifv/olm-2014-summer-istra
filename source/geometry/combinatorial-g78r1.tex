% $date: 2014-06-26

% $timetable:
%   g78r1:
%     2014-06-26:
%     - 2

\section*{Комбинаторная геометрия}

% $authors:
% - Фёдор Ивлев

\begin{problems}

\item
В~круге отмечена некоторая точка.
Разрежьте его на~две части, из~которых можно сложить круг, так, чтобы эта точка
оказалась его центром.
%http://problems.ru/view_problem_details_new.php?id=32093

\item
На~плоскости нарисовано несколько прямых (не~меньше двух), никакие две
из~которых не~параллельны и~никакие три не~проходят через одну точку.
Докажите, что среди частей, на~которые эти прямые делят плоскость, найдется
хотя~бы один угол.

\item
\sbp
Существуют~ли два равных семиугольника, все вершины которых совпадают,
но~никакие стороны не~совпадают?
\\
\sbp
А~три таких семиугольника?
\\
\emph{(Напоминание: многоугольник на~плоскости ограничен несамопересекающейся
замкнутой ломаной.)}

\item
Каждый из~трех синих квадратов пересекается с~каждым из~трех красных.
Верно~ли, что какие-то два одноцветных квадрата тоже пересекаются?
%http://problems.ru/view_problem_details_new.php?id=35629

\item
На~прямой расположено 100 точек.
Отметим середины всевозможных отрезков с~концами в~этих точках.
Какое наименьшее число отмеченных точек может получиться?

\end{problems}

