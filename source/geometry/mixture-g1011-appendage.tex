% $date: 2014-06-18

% $timetable:
%   g1011:
%     2014-06-18:
%     - 2
%     2014-06-20:
%     - 1

\section*{Добавка по геометрии}

% $authors:
% - Андрей Кушнир

\begin{problems}

\item
В~квадрате $ABCD$ отмечена такая точка $X$, что
$\angle BDX = \angle CBX  = \alpha$.
Найдите $\angle DAX$.

\item
$K$~--- основание перпендикуляра, опущенного из~ортоцентра $H$ на~касательную
к~описанной окружности треугольника $ABC$ в~точке $A$.
$M$~--- середина стороны $BC$.
Докажите, что треугольник $AMK$ равнобедренный.

\item
Окружности $\omega_1$ и~$\omega_2$ пересекаются в~точках $A$ и~$B$.
Окружность $\omega_3$ с~центром в~точке $A$ пересекает $\omega_1$ в~точках $M$,
$N$ и~пересекает окружность $\omega_2$ в~точках $P$, $Q$.
Докажите, что биссектрисы углов $\angle PBM = \angle QBN$.

\item\emph{Лемма о велосипедистах.}
Даны две пересекающиеся в~точках $A$ и~$B$ окружности.
Из~точки $A$ одновременно стартуют два велосипедиста и~едут каждый по~своей
окружности в~направлении против часовой стрелки с~равными угловыми скоростями.
\\
\sbp
Докажите, что в~любой момент времени прямая, соединяющая велосипедистов
проходит через точку $B$.
\\
\sbp
Докажите, что на~плоскости найдется точка, равноудаленная от велосипедистов
в~любой момент времени.

\item
В~сегмент окружности, натянутый на~меньшую дугу $AB$ вписываются всевозможные
окружности.
$M$~--- середина большей дуги $AB$.
Из~точки $M$ ко~всевозможным окружностям проводятся касательные $MX$,
где $X$~--- точка касания со~всевозможной окружностью.
Найдите геометрическое место точек $X$.

\item
В~треугольнике $ABC$ проведена биссектриса $B B_1$.
Перпендикуляр из~$B_1$ на $BC$ пересекает дугу $BC$ описанной окружности
треугольника $ABC$ в~точке~$K$.
Перпендикуляр из~$B$ на~$AK$ пересекает $AC$ в~точке~$L$.
Докажите, что точки $K$, $L$ и~середина дуги $AC$ (не~содержащей точку~$B$)
лежат на~одной прямой.

\end{problems}

