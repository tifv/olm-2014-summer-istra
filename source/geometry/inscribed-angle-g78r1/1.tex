% $date: 2014-06-20

% $timetable:
%   g78r1:
%     2014-06-20:
%     - 3

\section*{Вписанные углы, начало}

% $authors:
% - Фёдор Ивлев

На окружности с центром $O$ даны точки $A$, $B$ и $C$.
Тогда, если $A$ находится в той же полуплоскости относительно стороны $BC$, что
и точка $O$, то $\angle BOC = 2 \angle BAC$,
а если точки $O$ и $A$ лежат по разные стороны относительно стороны $BC$, то
$\angle BOC = 360^\circ - 2 \angle BAC$.

\corollary
Если точки $A$, $B$, $C$ и $D$ лежат на окружности в указанном порядке, то
$\angle ABD = \angle ACD$ и $\angle ABC + \angle CDA = 180^\circ$.

\begin{problems}

\item\textbf{Лемма об угле между хордой и касательной.}
Прямая~$AB$ касается окружности~$\omega$ в точке~$B$.
Так же на окружности выбраны точки~$C$ и~$D$.
Докажите, что либо $\angle BDC = \angle DCA$, либо
$\angle BDC + \angle DCA = 180^\circ$.
От чего это зависит?

% spell "чебурашьи уши" -> "уши Чебурашки"

\item
\sbp
Докажите, что ГМТ из которых данный отрезок $AB$ виден под данным углом
$\alpha$ есть объединение двух дуг, симметричных относительно этого отрезка и с
концами в концах отрезка (так называемые <<чебурашьи уши>>).
\\
\sbp
Выведите из этого, что четырехугольник является вписанным тогда и только тогда,
когда сумма двух его противоположных углов равна~$180^\circ$.
\\
\sbp
Докажите, что если точки $A$ и $B$ лежат по одну сторону от прямой $CD$, то
точки $A$, $B$, $C$ и $D$ лежат на одной окружности тогда и только тогда, когда
$\angle CAD = \angle CBD$.
\\
\sbp
Докажите, что если отрезок $AB$ виден из точек $C$ и $D$ под прямым углом, то
точки $A$, $B$, $C$ и $D$ лежат на одной окружности.

\item
Все углы треугольника меньше $120^\circ$.
Докажите, что внутри него существует точка, из которой все стороны видны под
углом $120^\circ$.

\item
Две окружности пересекаются в точках $P$ и $Q$.
Через $P$ и $Q$ проведены прямые $AB$ и $CD$, пересекающие первую окружность
в~точках $A$ и $C$, а вторую~--- в~точках $B$ и $D$.
Докажите, что $AC \parallel BD$.

\item
Вершина $A$ остроугольного треугольника $ABC$ соединена отрезком с центром~$O$
описанной окружности.
Из вершины~$A$ проведена высота~$AH$.
Докажите, что~$\angle BAH = \angle OAC$.

\item
На окружности даны точки $A$, $B$, $M$ и $N$.
Из точки~$M$ проведены хорды~$M A_1$ и~$M B_1$, перпендикулярные прямым~$NB$
и~$NA$ соответственно.
Докажите, что $A A_1 \parallel B B_1$.

\item
Касательная в точке~$A$ к описанной окружности треугольника~$ABC$ пересекает
прямую~$BC$ в точке~$E$; $AD$~---биссектриса треугольника~$ABC$.
Докажите, что~$AE = ED$.

\end{problems}

