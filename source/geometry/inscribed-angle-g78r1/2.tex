% $date: 2014-06-21

% $timetable:
%   g78r1:
%     2014-06-21:
%     - 3
%     2014-06-22:
%     - 2

\section*{Вписанные углы, продолжение}

% $authors:
% - Фёдор Ивлев

\begin{problems}

\item
Дан вписанный четырехугольник $ABCD$.
Его диагонали пересекаются в~точке~$M$, а лучи~$AB$ и~$DC$~--- в~точке~$P$.
Докажите, что угол $AMB$ равен полусумме дуг $AB$ и~$CD$.
Докажите, что угол $BPC$ равен полуразности дуг $AD$ и~$BC$.

\item
Две окружности пересекаются в~точках~$P$ и~$Q$.
Третья окружность с центром~$P$ пересекает первую окружность в~точках~$A$
и~$B$, а вторую~--- в~точках~$C$ и~$D$.
Докажите, что $\angle AQD = \angle BQC$.

\item
\sbp
Продолжение биссектрисы угла $B$ треугольника~$ABC$ пересекает описанную
окружность в~точке~$M$; $I$~--- центр вписанной окружности, $I_b$~--- центр
вневписанной окружности, касающейся стороны~$AC$.
Докажите, что точки~$A$, $C$, $I$ и~$I_b$ лежат на окружности с~центром~$M$.
\\
\sbp
Точка $O$, лежащая внутри треугольника $ABC$, обладает тем свойством, что
прямые~$AO$, $BO$ и~$CO$ проходят через центры описанных окружностей
треугольников~$BCO$, $ACO$ и~$ABO$.
Докажите, что $O$~--- центр вписанной окружности треугольника~$ABC$.

\item
Прямоугольный треугольник $ABC$ с прямым углом~$A$ движется так, что его
вершины~$B$ и~$C$ скользят по сторонам данного прямого угла.
Докажите, что множеством точек~$A$ является отрезок и~найдите его длину.

\item
В треугольнике $ABC$ проведены медианы $A A_1$ и~$B B_1$.
Докажите, что если $\angle C A A_1 = \angle C B B_1$, то $AC = BC$.

\item
В остроугольном треугольнике $ABC$ проведены высоты $A A_1$, $B B_1$, $C C_1$.
Докажите, что эти высоты являются биссектрисами углов
треугольника~$A_1 B_1 C_1$.
Выведите из этого, что высоты любого треугольника пересекаются в~одной точке.

\item
На окружности даны точки $A$, $B$, $C$, $D$ в~указанном порядке.
$M$~---середина дуги~$AB$.
Обозначим точки пересечения хорд~$MC$ и~$MD$ с~хордой~$AB$ через~$E$ и~$K$.
Докажите, что $KECD$~--- вписанный четырехугольник.

\end{problems}

