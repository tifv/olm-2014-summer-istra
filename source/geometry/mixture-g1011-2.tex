% $date: 2014-06-16

% $timetable:
%   g1011:
%     2014-06-16:
%     - 1

\section*{Геометрия. Листик второй}

% $authors:
% - Андрей Кушнир

\begin{problems}

\item
В окружность, на которой отмечены неподвижные точки $B$ и $C$, вписываются
всевозможные треугольники $ABC$.
Найдите ГМТ центров вписанных окружностей всех таких треугольников.

\item\emph{Лемма Архимеда.}
В сегмент, ограниченный хордой $MN$, вписана окружность, касающаяся хорды $MN$
в точке $X$, дуги $MN$ в точке $Y$.
Докажите, что прямая $XY$ проходит через середину дуги $NM$.

\item
На сторонах $AB$, $AC$ отмечены точки $X$, $Y$ соответственно так, что
$BX = CY$.
Докажите, что описанная окружность треугольника $AXY$ проходит через середину
дуги $BAC$ описанной окружности треугольника $ABC$.

\item
К точкам $B$ и $C$, лежащим на окружности $\omega$, проведены касательные,
пересекающиеся в точке $A$.
$M$~--- середина $BC$.
$X$~--- такая точка на $\omega$, что $\angle AXB = 90^{\circ}$.
Докажите, что $\angle CXM = 90^{\circ}$.

\item
В угол $AXB$ вписана окружность, касающаяся сторон угла в точках $A$ и $B$.
На меньшей дуге $AB$ отмечена точка $K$.
Прямая, проходящая через $K$, параллельно $AX$, пересекает $XB$ в точке~$L$.
Прямая $AK$ вторично пересекает описанную окружность $\omega$ треугольника
$KLB$ в точке $M$.
Докажите, что $MX$ касается $\omega$.

\item
$K$, $L$~--- точки касания вписанной и вневписанной окружностей треугольника
$ABC$ с отрезком $BC$ соответственно; $I$, $I_A$~--- их центры.
Докажите, что $IL$ и $I_AK$ пересекаются на высоте треугольника, опущенной из
точки $A$.

\item
В остроугольном треугольнике $ABC$ проведены высоты $A A_1$, $B B_1$, $C C_1$,
пересекающиеся в точке $H$.
$P$~--- точка пересечения $B_1 C_1$ и $A A_1$; $Q$~--- точка пересечения $AO$ и
$BC$, где $O$~--- центр описанной окружности треугольника.
$M$~--- середина $BC$.
Докажите, что $PQ \parallel HM$.

\item
В треугольнике $ABC$ выполнено $AB < BC$.
$I$~--- центр вписанной окружности, $M$~--- середина $AC$, $N$~--- середина
дуги $ABC$.
Докажите, что $\angle IMA = \angle INB$.

\end{problems}

