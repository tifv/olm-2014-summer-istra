% $date: 2014-06-17

% $timetable:
%   g9:
%     2014-06-17:
%     - 1
%     2014-06-18:
%     - 1

\section*{Серия 2, про геометрические места точек}

% $authors:
% - Фёдор Бахарев

\begin{problems}

\item
На плоскости даны две точки $A$ и $B$.
Кроме того, дано число $a > 0$.
Найдите геометрическое место точек $X$ таких, что $AX^2 - BX^2 = a$.

\item
На плоскости дан треугольник $ABC$.
Рассмотрим геометрическое место точек плоскости $X$, для которых
$AX / BX = AC / BC$.
\\
\sbp
Укажите четыре различных точки на плоскости, принадлежащих этому ГМТ.
\\
\sbp
Найдите это геометрическое место точек.

\item
Дана окружность $\omega$ и точки $A$ и $B$ на ней.
Точка $C$ движется по окружности $\omega$.
Какую траекторию описывает
\\
\sbp
ортоцентр $H$ треугольника $ABC$?
\\
\sbp
точка пересечения медиан треугольника $ABC$?
\\
\sbp центр вписанной окружности $I$?
\\
\sbp центр вневписанной окружности $I_c$?
\\
\sbp центр вневписанной окружности $I_a$?

\item
Даны две равные окружности, пересекающиеся в точках $A$ и $B$.
Через точку $A$ проводится прямая $\ell$, пересекающая вторично первую
окружность в точке $X$, а вторую в точке $Y$.
Найдите геометрическое место середин отрезков $XY$, когда прямая $\ell$
меняется.

\item
Дан треугольник $ABC$.
Найдите геометрическое место середин отрезков с концами на контуре треугольника
$ABC$.

\item
Дан прямой угол с вершиной $O$.
Найдите геометрическое место середин отрезков единичной длины с концами на
сторонах угла.

\item
На плоскости дан равносторонний треугольник $ABC$.
\\
\sbp
Докажите, то если точка $P$ лежит внутри треугольника, то из отрезков
$PA$, $PB$ и $PC$ можно составить треугольник.
\\
\sbp
Найдите геометрическое место таких точек $P$ внутри $ABC$, что треугольник,
составленный из $PA$, $PB$ и $PC$, является остроугольным.

\end{problems}

