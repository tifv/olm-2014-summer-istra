% $date: 2014-06-17

% $timetable:
%   g78r1:
%     2014-06-17:
%     - 3
%     2014-06-18:
%     - 3

\section*{Входной разнобой}

% $authors:
% - Фёдор Ивлев

\begin{problems}

\item
На стороне $CD$ квадрата $ABCD$ построен правильный треугольник $CDN$,
вершина~$N$ которого лежит вне квадрата, а на диагонали $AC$~--- правильный
треугольник $ACM$, внутри которого лежит точка $D$.
Докажите, что $MN$ равно стороне квадрата.

\item
В выпуклом четырёхугольнике $ABCD$ диагонали $AC$ и~$BD$ равны, а серединный
перпендикуляр к~стороне $BC$ проходит через середину стороны $AD$.
Докажите, что $AB = CD$.

\item
В выпуклом четырехугольнике $ABCD$ углы $\angle B$ и $\angle C$ равны
$120^\circ$, а $BC + CD = AB$.
Докажите, что $AC = AD$. 

\item
Треугольники~$ABC$ и~$A_1 B_1 C_1$ таковы, что $AB = A_1 B_1$, $BC = B_1 C_1$
и~$\angle A = \angle A_1$.
Докажите, что либо эти треугольники равны, либо
$\angle C + \angle C_1 = 180^{\circ}$.

\item
Пусть~$B B_1$~--- биссектриса неравнобедренного треугольника~$ABC$ с
углом~$\angle B = 48^{\circ}$.
Из точки~$O$, лежащей на луче~$BB_1$, опустили перпендикуляр~$OH$ на
сторону~$AC$.
Оказалось, что~$AH = HC$.
Найдите угол~$\angle OAC$.

\item
Дана трапеция $ABCD$ такая, что $AD \parallel BC$, $AB = BC$, $AC = CD$ и
$AD = BC + CD$.
Найдите ее углы.

\item
Внутри равнобедренного треугольника $ABC$ ($AB = BC$) отмечена точка $D$.
Известно, что $\angle DAC = 30^\circ$, $\angle DCA = 10^\circ$ и
$\angle ABC = 80^\circ$.
Найдите $\angle BDC$.

\item
Точка~$M$ взята на стороне~$AC$ равностороннего треугольника $ABC$, а на
продолжении стороны~$BC$ за вершину~$C$ отмечена точка~$N$ так, что $BM = MN$.
Докажите, что $AM = CN$.

\end{problems}

