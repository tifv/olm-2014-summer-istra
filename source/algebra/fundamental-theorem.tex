% $date: 2014-06-15

% $timetable:
%   g11:
%     2014-06-15:
%     - 3

\section*{Основная теорема алгебры}

% $authors:
% - Олег Орлов

%\claim{Основная теорема алгебры}
\theorem
Любой многочлен степени не~меньше единицы с~комплексными коэффициентами имеет
хотя~бы один комплексный корень.

Рассмотрим многочлен степени $n$ с~комплексными коэффициентами:
\[
    P(z)
=
    a_n z^n + a_{n-1} z^{n-1} + \ldots + a_1 z + a_0
\]

\begin{problems}

\item
Докажем основную теорему алгебры:
\\
\sbp
Докажите, что можно выбрать такое действительное положительное число $R$, что
для любого комплексного $z$, такого, что $|z| > R$ выполняется неравенство
$|P(z)| > |a_0|$.
\emph{В~последующих задачах имеется в~виду именно этот $R$.}
\\
\sbp
Докажите, в~круге $|z| \leq R$ у~функции $|P(z)|$ существует минимум.
\emph{(Попробуйте рассмотреть квадрат, который содержит круг $|z| \leq R$.)}
\\
\sbp
Пусть в~круге $|z| \leq R$ у~функции $|P(z)|$ достигается минимум в~точке $z'$.
Допустим $|P(z')| > 0$.
Докажите тогда, что модуль многочлена $P(z+z') / P(z')$ достигает минимума
(на~всей комплексной плоскости) в~точке $0$, и~минимум этот равен единице.
\\
\sbp
Докажите, что если у~многочлена $n$-ой степени $k$~--- наименьшая ненулевая
степень с~ненулевым коэффициентом, то~заменой $y = c z$, подбирая $c$, можно
добиться, чтобы коэффициент многочлена $a_k$ стал равен $-1$.
\\\emph{В~результате предыдущих пунктов получим, что модуль многочлена
$Q(z) = P(c z + z') / P(z')$ принимает минимум (на~всей комплексной плоскости)
в~точке $0$, минимум этот равен единице, и
\(
    Q(z)
=
    b_n z^n + b_{n-1} z^{n-1} + \ldots + b_{k+1} z^{k+1} - z^k + 1
\).}
\\
\sbp
Докажите, что существует такое действительное положительное число
$\varepsilon < 1$, что внутри круга $|z| \leq \varepsilon$ выполняется
неравенство
$|z^k| > |b_n z^n + \ldots + b_{k+1} z^{k+1}|$.
\\
\sbp
Докажите, что $|Q(x)| < 1$ для произвольного действительного положительного
$x < \varepsilon$.
\\
\sbp
Докажите основную теорему алгебры.

\item
\sbp
Докажите, что многочлен $P(z)$ степени $n$ с~комплексными коэффициентами имеет
ровно $n$ комплексных корней, с~учетом кратности.
(Корень $z_0$ многочлена $P(z)$ имеет кратность $k$, если $P(z)$ делится
на~$(z - z_0)^k$ и~не~делится на~$(z - z_0)^{k+1}$.)
\\
\sbp
Докажите, что любой многочлен с~вещественными коэффициентами разлагается
в~произведение многочленов первой и~второй степени с~вещественными
коэффициентами.

\item
Докажите, что если многочлен $P(x)$ с~действительными коэффициентами принимает
при всех действительных $x$ неотрицательные значения, то~он~представим в~виде
$P(x) = Q^2(x) + R^2(x)$, для некоторых многочленов $Q(x)$, $R(x)$
с~действительными коэффициентами.

\end{problems}

