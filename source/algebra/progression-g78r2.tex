% $date: 2014-06-23

% $timetable:
%   g78r2:
%     2014-06-23:
%     - 1
%     2014-06-25:
%     - 1

\section*{Прогрессии и другая борьба с многоточием}

% $authors:
% - Алексей Пономарёв

Последовательность, члены которой связаны соотношением вида $a_n = a_{n-1} + d$,
называется \emph{арифметической прогрессией}.
Число $d$ называется \emph{разностью} прогрессии, а число $a_1$~--- первым членом.

Последовательность \emph{без нулевых членов}, для которой рекуррентное соотношение
имеет вид $b_n = q \cdot b_{n-1}$, называется \emph{геометрической прогрессией}.
Число $q$ называется \emph{знаменателем} этой прогрессии.

Последовательность $\{a_n\}$ является арифметической прогрессией тогда и только тогда,
когда любой её член равен среднему арифметическому двух своих соседей.

Последовательность $\{b_n\}$ является геометрической прогрессией тогда и только тогда,
когда она не содержит нулей и для любого $n$ справедливо
$a_n^2={a_{n-1}}\cdot{a_{n+1}}$.

Пусть $\{a_n\}$~--- арифметическая прогрессия с разностью $d$.
Докажем и \emph{запомним}, что
\[
    a_1 + a_2 + \ldots + a_n
=
    n a_1 + \frac{n (n - 1) d}{2}
\qquad
    a_1 + a_2 + \ldots + a_n
=
    n \frac{a_1 + a_n}{2}
\;.\]

Пусть $\{b_n\}$~--- геометрическая прогрессия со знаменателем $q \neq 1$.
Докажем, что
\[
\underbrace{
    b_1 + b_2 + \ldots + b_n
=
    b_1 \cdot \frac{1 - q^{n}}{1 - q}
}_{\text{\em запомните!}}
\qquad
    b_1 b_2 \ldots b_n
=
    b_1^n \cdot q^{\tfrac{n(n-1)}{2}}
\;.\]

Интересно, а как устроены последовательности, основанные на других средних\ldots

\begin{problems}

\item
В концах диаметра окружности стоят единицы.
На первом шаге каждая из получившихся дуг делится пополам, и в её середине пишется
сумма чисел, стоящих в концах.
Затем то же самое делается с каждой из четырех полученных дуг и~т.~д.
Такая операция проделывается $n$ раз. Найдите сумму всех полученных чисел.

\item
Найдите сумму $1 + 3 + 5 + \ldots + (2 n - 1)$.

\item
Найдите сумму $1 + 2 + 4 + \ldots + (2^n)$.

\item
Найдите сумму $n$ чисел $1 + 11 + 111 + \ldots + 11 \ldots 11$.

\item
Сумма первых $n$ членов некоторой последовательности равна $n^2$ при любом $n$.
Верно ли, что данная последовательность~--- арифметическая прогрессия?

\item
Пусть $S_n$~--- сумма первых $n$ членов арифметической прогрессии.
Пусть для некоторых $m \neq n$ справедливо $S_m = S_n$.
Найдите $S_{m + n}$.

\item
Даны две арифметические прогрессии $a_1$, $a_2$, $a_3$ и $b_1$, $b_2$, $b_3$.
Известно, что числа $a_1 + b_1$, $a_2 + b_2$, $a_3 + b_3$ образуют геометрическую
прогрессию и что $a_1 + a_2 + a_3 = b_1 + b_2 + b_3$.
Докажите, что $a_1 = b_3$, $a_2 = b_2$, $a_3 = b_1$.

\item
Найдите произведение первых трех членов арифметической прогрессии, если сумма ее первых
трех членов равна $27$, а сумма их квадратов равна $275$.

\item
Сумма первых трех членов геометрической прогрессии равна $91$.
Если увеличить эти члены на $25$, $27$ и $1$ соответственно, то получатся три числа,
образующие геометрическую прогрессию.
Найдите ее седьмой член.

\item
Сумма первого и третьего членов геометрической прогрессии равна $10$;
сумма второго и четвертого членов этой же прогрессии равна $20$.
Найдите эту геометрическую прогрессию.

\item
Длины сторон прямоугольного треугольника являются последовательными членами некоторой
геометрической прогрессии.
Чему равна длина гипотенузы данного треугольника, если его площадь равна $1$?

\item
Числа $a$, $b$, $c$ образуют арифметическую прогрессию.
Докажите, что числа $a^2 + a b + b^2$, $b^2 + b c + c^2$, $c^2 + c a + a^2$ также
образуют арифметическую прогрессию.

\item
Вычислите <<хитрым>> образом значение дроби
\[
    \dfrac{1 + 3 + 3^{2} + \ldots + 3^{11}}{1 + 3 + 3^{2} + \ldots + 3^{5\ \mathstrut}}
\]

\item
Докажите
\[
    \frac{1}{2} - \frac{1}{3} + \frac{1}{4} - \frac{1}{5}
    + \ldots +
    \frac{1}{98} - \frac{1}{99} + \frac{1}{100}
>
    \frac{1}{5}
\]

\item
Докажите
\[
    \frac{1}{10} + \frac{1}{11} + \frac{1}{12} + \ldots + \frac{1}{100}
>
    1
\]

\item
Сравните числа
\[
    A = 1 \cdot 2 \cdot 3 \cdot \ldots \cdot 19 \cdot 20
\quad\text{и}\quad
    B = 1 + 2 + 3 + \ldots + 999999 + 1000000
\]

\item
Докажите, что найдется $n$ такое, что
\[
    \frac{1}{1} + \frac{1}{2} + \frac{1}{3} + \ldots + \frac{1}{n}
>
    1000000
\]

\item
Докажите
\[
    \frac{1}{3} + \dfrac{1}{8} + \frac{1}{15} + \ldots + \frac{1}{2014^2 - 1}
<
    \frac{3}{4}
\]

\item
Докажите
\[
    \frac{1}{2^2} + \frac{1}{3^2} + \ldots + \frac{1}{99^2} + \frac{1}{100^2}
<
    \frac{99}{100}
\]

\end{problems}

