% $date: 2014-06-16

% $timetable:
%   g10:
%     2014-06-16:
%     - 2

\section*{Многочлены: китайская теорема об остатках, лемма Гаусса, \ldots}

% $authors:
% - Алексей Устинов

\subsection*{Предварительные сведения}

\begin{enumerate}

\item
$\mathbb{Z}[x]$~--- множество многочленов от переменной $x$ с целыми
коэффициентами,
$\mathbb{R}[x]$~--- с действительными коэффициентами,
\ldots

\item
Если $a$, $b$~--- целые числа и $f \in \mathbb{Z}[x]$, то
$(b - a) \mid (f(b) - f(a))$.

\item\emph{Деление многочленов с остатком.}
Если $p(x)$ и $q(x)$~--- многочлены, причем $q(x)$ не равен нулю
тождественно, то существуют многочлены $t(x)$ и $r(x)$ такие, что
$p(x) = q(x) t(x) + r(x)$, и $\deg r(x)<\deg q(x)$;
при этом $t(x)$ и $r(x)$ определяются однозначно.

\item\emph{Теорема Безу.}
Остаток от деления многочлена $p(x)$ на $(x - c)$ равен $p(c)$.

\item
Для двух многочленов $p(x)$, $q(x) \in \mathbb{R}[x]$ всегда определен их
\emph{наибольший общий делитель} $d(x) = (p(x), q(x)) \in \mathbb{R}[x]$
(многочлен максимальной степени, на который делится каждый из данных).
При этом существуют $u(x), v(x) \in \mathbb{R}[x]$ такие, что
$u(x) p(x) + v(x) q(x) = d(x)$.
При этом $\mathbb{R}$ можно заменять на $\mathbb{Q}$ или $\mathbb{Z}$.

\item
Два многочлена $a(x)$ и $b(x)$ называются \emph{сравнимыми по модулю $m(x)$,}
если их разность делится на $m(x)$.
Как и для чисел, соотношение сравнимости для двух многочленов записывается в
виде $a(x) \equiv b(x) \pmod{m(x)}$.

\end{enumerate}

\subsection*{Задачи}

\begin{problems}

\item
Пусть $a$, $b$ и $n$~--- натуральные числа.
Известно, что при любом натуральном $k$ ($k \neq b$) число $a - k^n$ делится
без остатка на $b-k$.
Докажите, что $a = b^n$.

\item
\sbp
Пусть $a$, $m$ и $n$~--- натуральные числа, $a > 1$.
Докажите, что если $a^m + 1$ делится на $a^n + 1$, то $m$ делится на $n$.
\\
\sbp
Пусть $a$, $b$, $m$, $n$~--- натуральные числа, причем $(a, b) = 1$, и
$a > 1$.
Докажите, что если $a^m + b^m$ делится на $a^n + b^n$, то $m$ делится на $n$.

\item
Пусть $f(x) = x^2 - x + 1$.
Докажите, что для любого натурального $m>1$ числа
$m$, $f(m)$, $f(f(m))$, $\ldots$
попарно взаимно просты.

\item
Пусть $f$ и $g$~--- взаимно простые многочлены с целыми коэффициентами.
Докажите, что последовательность чисел $a_n = (f(n), g(n))$ периодическая.

\item\textbf{Китайская теорема об остатках для многочленов.}
Пусть $m_1(x)$, \ldots, $m_n(x)$ попарно взаимно простые многочлены, то есть
$(m_i(x), m_j(x)) = 1$ при $i \neq j$,
$a_1(x)$, \ldots, $a_n(x)$~--- произвольные многочлены.
Докажите, что тогда существует ровно один многочлен $p(x)$ такой, что
$\deg p(x) < \deg m_1(x) + \ldots + \deg m_n(x)$ и
\[ \left\{ \begin{array}{c}
    p(x) \equiv a_1(x) \pmod{m_1(x)}
\\ \ldots \\
    p(x) \equiv a_n(x) \pmod{m_n(x)}.
\end{array} \right. \]

% spell "Лемма Гаусса II" -> "Лемма Гаусса, вторая"
% spell "Лемма Гаусса I" -> "Лемма Гаусса, первая"

\item\textbf{Лемма Гаусса I.}
Многочлен с целыми коэффициентами называется \emph{примитивным}, если НОД его
коэффициентов равен $1$.
Докажите, что произведение двух примитивных многочленов также является
примитивным многочленом.

\item\textbf{Лемма Гаусса II.}
Если многочлен $f \in \mathbb{Z}[x]$ можно разложить в произведение двух
многочленов $g, h \in \mathbb{Q}[x]$, то его можно представить и в виде
$f = \widetilde{g} \widetilde{h}$, где
$\widetilde{g}, \widetilde{h} \in \mathbb{Z}[x]$.

\item
Пусть $p \in \mathbb{Z}[x]$, $\deg p > 1$.
Докажите, что существует бесконечная последовательность целых чисел, не
являющихся значениями многочлена $p$.

\item\emph{Лемма Шура.}
Пусть $f \in \mathbb{Z}[x]$, $\deg f > 0$.
Докажите, что множество простых чисел, делящих хотя бы одно из ненулевых чисел
$f(1)$, $f(2)$, $\ldots$, $f(n)$, $\ldots$ бесконечно.

\item
Пусть $f, g \in \mathbb{Z}[x]$~--- неприводимые многочлены с единичными
старшими коэффициентами, и $\deg f, \deg g > 0$.
Известно, что для достаточно больших $n$ множества простых делителей чисел
$f(n)$ и $g(n)$ совпадают.
Докажите, что $f = g$.

\item
Пусть $f \in \mathbb{Z}[x]$, $\deg f > 0$ и $n$, $k$~--- натуральные числа.
Докажите, что найдется натуральное $a$ такое, что каждое из чисел $f(a)$,
$f(a + 1)$, \ldots, $f(a + n - 1)$ имеет по крайней мере $k$ различных простых
чисел.

\end{problems}

