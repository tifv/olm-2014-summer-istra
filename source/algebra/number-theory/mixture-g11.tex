% $date: 2014-06-20

% $timetable:
%   g11:
%     2014-06-20:
%     - 1

\section*{Теория чисел}

% $authors:
% - Олег Орлов

\begin{problems}

\item
Докажите, что если $\sqrt{7} - m / n > 0$,
то $\sqrt{7} - m / n > 1 / (mn)$,
где $m$ и $n$~--- натуральные числа.

\item
Докажите, что уравнение
\[
    x^2 + y^2 + z^2 + t^2 = 2 x y z t
\]
не имеет решений в натуральных числах.

\item
Даны натуральные числа $a$, $b$, $c$, взаимно простые в совокупности.
Верно ли, что обязательно существует такое натуральное $n$, что число
$a^k + b^k + c^k$ не делится на $2^n$ ни при одном натуральном $k$?

\item
Даны натуральные числа $m$ и $n$.
Докажите, что число $2^n - 1$ делится на число $(2^m - 1)^2$ тогда и только
тогда, когда число $n$ делится на число $m \cdot (2^m - 1)$.

\item
Дано натуральное число $c$ и последовательность
$p_1$, $p_2$, $\ldots$, $p_n$, $\ldots$ простых чисел, удовлетворяющая условию:
$p_i + c$ делится на $p_{i+1}$ (для любого натурального $i$).
Докажите, что последовательность $\{p_n\}$ ограничена.

\item
Докажите, что для любого натурального числа $n \neq 3$ и для любого
натурального числа $k$, меньшего $n + 2$ существует простое число $p$ такое,
что $n! + k$ делится на $p$, но никакое из чисел
$n! + 1$, $n! + 2$, $\ldots$, $n! + n$, $n! + n + 1$,
кроме $n! + k$, не делится на $p$.

\item
При каких натуральных $n > 1$ существуют такие натуральные
$b_1$, $b_2$, $\ldots$, $b_n$ (не все из которых равны), что при всех
натуральных $k$ число
$(b_1 + k) \cdot (b_2 + k) \cdot \ldots \cdot (b_n + k)$
является степенью натурального числа?
(Показатель степени может зависеть от $k$, но должен быть всегда больше $1$.)

%\item
%Для заданного натурального числа $k > 1$ через $Q(n)$, $n \in \mathbb{N}$,
%обозначим наименьшее общее кратное чисел $n$, $n + 1$, $\ldots$, $n + k$.
%Докажите, что существует бесконечно много $n \in \mathbb{N}$ таких, что
%$Q(n) > Q(n + 1)$.

%\item
%Даны натуральные числа $x$ и $y$ из отрезка $[2; 100]$.
%Докажите, что при некотором натуральном $n$ число $x^{2^n} + y^{2^n}$~---
%составное.

\end{problems}

