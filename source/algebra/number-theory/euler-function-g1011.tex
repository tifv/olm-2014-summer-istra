% $date: 2014-06-15

% $timetable:
%   g1011:
%     2014-06-15:
%     - 2

\section*{Функция Эйлера}

% $authors:
% - Олег Орлов

\definition
Пусть $n$~--- натуральное число.
\emph{Функцией Эйлера $\varphi(n)$}
называется количество натуральных чисел не больших $n$ и взаимно простых с ним.
\par
Далее во всех задачах $n$, $m$, $a$~--- натуральные числа, $p$~--- простое.
Через $(m, n)$ обозначается НОД чисел $m$ и $n$.

\begin{problems}

\item
\sbp Докажите, что $\varphi(p^n) = p^{n-1} (p-1)$.
\\
\sbp Пусть $(m, n) = 1$.
Докажите, что $\varphi(m n) = \varphi(m) \cdot \varphi(n)$.
Это свойство называется \emph{мультипликативностью}.
\emph{(Подсказка:
рассмотрим остаток $r$ при делении на $m n$ взаимно простой с $m n$.
Докажите, что ни для какого другого остатка $q$, взаимно простого с $mn$, не могут
одновременно совпадать остатки при делении $q$ и $r$ на $m$ и остатки при делении $q$ и
$r$ на $n$.)}
\\
\sbp
Пусть $m = p_1^{t_1} \ldots p_k^{t_k}$~--- разложение на простые множители.
Докажите, что
\[
    \varphi(m)
=
    m \left(1 - \frac{1}{p_1}\right) \ldots \left(1 - \frac{1}{p_k}\right)
\;.\]

\item
Докажите, что $\varphi(d_1) + \varphi(d_2) + \ldots + \varphi(d_s) = m$, где
суммирование ведется по всем делителям числа $m$.
\emph{(Подсказка: посчитайте количество дробей $1 / m$, $2 / m$, $\ldots$, $m / m$.)}

\item
Найдите сумму всех правильных несократимых дробей со знаменателем $n$.

\item
\sbp
Докажите \emph{теорему Эйлера}.
Пусть $(a, m) = 1$, тогда $a^{\varphi(m)} - 1$ делится на $m$.
\emph{(Подсказка: если $x$ и $y$~--- различные остатки по модулю $m$ и взаимно простые
с ним, то $a x$ и $a y$ тоже различные остатки по модулю $m$ и взаимно простые с ним.)}
\\
\sbp
Докажите усиление теоремы Эйлера.
Пусть $m = p_1^{t_1} \ldots p_k^{t_k}$~--- разложение на простые множители.
Положим $L(m) = \text{НОК}(\varphi(p_1^{t_1}), \ldots, \varphi(p_k^{t_k}))$.
Тогда, если $(a, m)=1$, то $a^{L(m)} - 1$ делится на $m$.

\item
Простое число $p$ больше пяти.
Докажите, что число из $p - 1$ единицы делится на $p$.

\item
Докажите, что для каждого $n$ существует натуральное число, состоящее только из нулей и
единиц, которое делится на $n$.

\item
Дано число $2^{2014}$.
Докажите, что можно приписать к нему слева несколько цифр так, чтобы получилась степень
двойки.

\item
Докажите, что $\varphi (a^n - 1)$ делится на $n$.

\item
Число $1 / n$, где $n$ взаимно просто с $10$, представили в виде периодической
десятичной дроби.
Докажите, что ее период является делителем $\varphi(n)$.

%\item
%Между двумя единицами пишется двойка, затем между любыми двумя соседними числами
%пишется их сумма и т.\,д.
%Докажите, что число $n$ в итоге будет выписано ровно $\varphi(n)$ раз.

\end{problems}

