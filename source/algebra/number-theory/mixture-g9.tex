% $date: 2014-06-24

% $timetable:
%   g9:
%     2014-06-24:
%     - 2

\section*{Теория чисел}

% $authors:
% - Михаил Харитонов

\begin{problems}

\item
Найдите свободный член многочлена $P(x)$ с целыми коэффициентами, если
известно, что его модуль меньше $700$ и $P(200) = P(7) = 2007$.
    
\item
Существуют ли $14$ натуральных чисел таких, что при увеличении каждого из них
на один произведение увеличится в $2008$ раз?

\item
Назовём число $n$ \emph{хорошим,} если каждое из чисел $n$, $n+1$, $n+2$, $n+3$
делится на сумму своих цифр (например, $60398$).
Если хорошее число заканчивается на цифру $8$, будет ли предпоследняя цифра
равна $9$?

\item
Пусть $p > 5$~--- простое число.
Докажите, что число, состоящее из $(p - 1)$-й единицы~--- $11 \ldots 1$~---
делится на $p$.

\item
Докажите бесконечность количества хороших чисел.

\item
Пусть число $N$~--- \emph{самодельное}, если оно делится на сумму любых подряд
идущих своих цифр.
Докажите, что число самодельных чисел~--- конечно.

\end{problems}

