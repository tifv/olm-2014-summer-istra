% $date: 2014-06-17

% $timetable:
%   g1011:
%     2014-06-17:
%     - 1

\section*{Теория чисел}

% $authors:
% - Олег Орлов

\begin{problems}

\item
Каждое из натуральных чисел $a$, $b$, $c$ и $d$ делится на натуральное число
$a b - c d$.
Докажите, что $a b - c d = 1$.

\item
Рассматриваются $2014$ чисел: $11$, $101$, $1001$, $\ldots$
Докажите, что среди этих чисел не менее $99\%$ составных.

\item
Натуральные числа $a$, $b$, $c$, $d$ таковы, что $a b = c d$.
Докажите, что число $a + b + c + d$~--- составное.

\item\emph{Китайская теорема об остатках.}
Натуральные числа $m_1$, $\ldots$, $m_n$ попарно взаимно просты.
\\
\sbp
Найдите в явном виде какое-нибудь целое число $x$, удовлетворяющее системе
\[ \left\{ \begin{aligned} &
    x \equiv 1 \pmod{m_1}
\\ &
    x \equiv 0 \pmod{m_2}
\\ &
    \ldots
\\ &
    x \equiv 0 \pmod{m_n}
\end{aligned} \right. \]
\sbp
Для любых целых $a_1$, $a_2$, $\ldots$, $a_n$ найдите все целые $x$,
удовлетворяющие системе
\[ \left\{ \begin{aligned} &
    x \equiv a_1 \pmod{m_1}
\\ &
    x \equiv a_2 \pmod{m_2}
\\ &
    \ldots
\\ &
    x \equiv a_n \pmod{m_n}
\end{aligned} \right. \]

\item
Существуют ли $100$ подряд идущих чисел таких, что ровно $10$ из них простые?

\item
Докажите, что среди любых $10$ натуральных чисел можно выбрать несколько, сумма
которых делится на $10$.

\item
Даны натуральные числа $a$, $b$, $c$, взаимно простые в совокупности.
Верно ли, что обязательно существует такое натуральное $n$, что число
$a^k + b^k + c^k$ не делится на $2^n$ ни при одном натуральном $k$?

\item
Докажите, что уравнение
\[
    x^2 + y^2 + z^2 + t^2 = 2 x y z t
\]
не имеет решений в натуральных числах.

\end{problems}

