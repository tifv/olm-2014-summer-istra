% $date: 2014-06-16

% $timetable:
%   g78r2:
%     2014-06-16:
%     - 3

\section*{Серия 1, сравнительно-простая}

% $authors:
% - Фёдор Бахарев

\begin{problems}

\item
Число $3 a + 5 b$ делится на~$11$.
Докажите, что число $5 a + b$ тоже делится на~$11$.

\end{problems}

\definition
Будем говорить что натуральные числа $a$ и~$b$ сравнимы по~модулю $m$ и~писать
\[
    a \equiv b \pmod{m}
,\]
если $(a - b)$ делится на~$m$.
Сама запись $a \equiv b \pmod{m}$ называется \emph{сравнением}.

\begin{problems}

\item
Докажите, пользуясь только определением, следующие свойства сравнений:
\sbp $a \equiv a$ (рефлексивность);
\\
\sbp если $a \equiv b$, то~$b \equiv a$ (симметричность);
\\
\sbp если $a \equiv b$ и~$b \equiv c$, то~$a \equiv c$ (транзитивность);
\\
\sbp если $a \equiv b$ и~$c \equiv d$, то~$a + c \equiv b + d$;
\\
\sbp если $a \equiv b$ то~$ac \equiv bc$;
\\
\sbp если $a \equiv b$ и~$c \equiv d$, то~$ac \equiv bd$;
\\
\sbp если $a \equiv b$ то~$a^n \equiv b^n$.
\\
(Все сравнения взяты по~одному и~тому~же модулю $m$.)

\item
Решите в~целых числах сравнение $3 x \equiv 239 \pmod 6$.

\item
$A = \overline{a_n a_{n-1} \ldots a_2 a_1 a_0}$~--- натуральное число.
Докажите, что
\\
\sbp $A \equiv \overline{a_1 a_0} \pmod 4$;
\\
\sbp $A \equiv \overline{a_2 a_1 a_0} \pmod 8$;
\\
\sbp $A \equiv a_0 + a_1 + \ldots + a_n \pmod 9$;
\\
\sbp $A \equiv a_0 - a_1 + a_2 - \ldots + (-1)^n a_n \pmod {11}$.

\item
Докажите, что
\sbp при натуральных $a$ и~$n$ \quad $a^n-1$ делится на~$a-1$;
\\
\sbp при натуральных $a$ и~$b$ и~нечетном $n$ число $a^n + b^n$ делится
на~$a + b$.

\item
$A = (16 a + 17 b) (17 a + 16 b)$ делится на~$11$.
Докажите, что $A$ делится на~$121$.

\item
Докажите, что $30^{99} + 61^{100}$ делится на~$31$.

\item
Докажите, что простое число $p$ входит в~разложение $n!$ на~простые множители
с~показателем
\[
    \left[  \frac{n}{p}  \right] +
    \left[ \frac{n}{p^2} \right] +
    \left[ \frac{n}{p^3} \right] + \ldots
\]

\end{problems}

