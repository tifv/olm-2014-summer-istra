% $date: 2014-06-16

% $timetable:
%   g78r1:
%     2014-06-16:
%     - 1

\section*{Серия 1, сравнительно-простая}

% $authors:
% - Фёдор Бахарев

\begin{problems}

\item
$A = \overline{a_n a_{n-1} \ldots a_2 a_1 a_0}$~--- натуральное число.
Докажите, что
\\
\sbp $A \equiv a_0 + a_1 + \ldots + a_n \pmod{9}$;
\\
\sbp $A \equiv a_0 - a_1 + a_2 - \ldots + (-1)^n a_n \pmod{11}$;
\\
\sbp
\(
    A
\equiv
    \overline{a_2 a_1 a_0} - \overline{a_5 a_4 a_3} +
    \overline{a_8 a_7 a_6} - \ldots
    \pmod{7}
\);
\\
\sbp
\(
    A
\equiv
    \overline{a_2 a_1 a_0} + \overline{a_5 a_4 a_3} +
    \overline{a_8 a_7 a_6} + \ldots
    \pmod{37}
\).

\item
$A = (16 a + 17 b) (17 a + 16 b)$ делится на~$11$.
Докажите, что $A$ делится на~$121$.

\item
Докажите, что $30^{99} + 61^{100}$ делится на~$31$.

\item
Пусть $a \equiv b \pmod m$.
Докажите, что $(a, m) = (b, m)$.

\item
Пусть $p$~--- простое число.
\\
\sbp Докажите, что если $x^2 \equiv 1 \pmod{p}$, то~$x \equiv \pm 1 \pmod{p}$.
\\
\sbp\emph{Теорема Вильсона.}
Докажите, что $(p - 1)! \equiv -1 \pmod{p}$.
\\
\sbp
Докажите, что если для $m > 1$ имеет место сравнение
$(m - 1)! \equiv -1 \pmod{m}$, то~$m$~--- простое.
\medskip

\item
Докажите, что простое число $p$ входит в~разложение $n!$ на~простые множители
с~показателем
\[
    \left[  \frac{n}{p}  \right] +
    \left[ \frac{n}{p^2} \right] +
    \left[ \frac{n}{p^3} \right] + \ldots
\]

\item
Докажите, что при всех натуральных $n$ имеет место равенство
$[1, 2, 3, \dots, 2n] = [n+1, n+2, \dots, 2n]$.

\item
Для любого натурального $n$ строится бесконечная последовательность
$n$, $d(n)$, $d(d(n))$, $\ldots$, где $d(k)$~--- число натуральных делителей
числа $k$.
Найдите все такие $n$, для которых в~соответствующей последовательности нет
полных квадратов.

\end{problems}

