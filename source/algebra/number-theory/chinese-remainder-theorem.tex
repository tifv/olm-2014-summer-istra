% $date: 2014-06-15

% $timetable:
%   g10:
%     2014-06-15:
%     - 3

\section*{Китайская теорема об остатках}

% $authors:
% - Алексей Устинов

\claim{Китайская теорема об остатках}
Пусть целые числа $m_1$, \ldots, $m_n$ попарно взаимно просты, то есть
$(m_i, m_j) = 1$ при $i \neq j$, $m = m_1 \ldots m_n$, и
$a_1$, \ldots, $a_n$~--- произвольные целые числа.
Тогда существует ровно одно целое число $x$ такое, что $0 \leq x < m$ и
\[
    \left\{ \begin{array}{l}
    x \equiv a_1 \pmod{m_1}
\\ \ldots \\
    x \equiv a_n \pmod{m_n}
    \end{array} \right.
\]

\begin{problems}

\item
Найдите наименьшее натуральное число, дающее при делении на $2$, $3$, $5$, $7$
остатки $1$, $2$, $4$, $6$ соответственно.

\item
При каких целых $n$ число $a_n = n^2 + 3 n + 1$ делится на $55$?

\item
Найдите остатки от деления:
\\
\sbp $19^{10}$ на $66$;
\quad
\sbp $19^{14}$ на $70$;
\quad
\sbp $17^9$ на $48$;
\quad
\sbp $14^{14^{14}}$ на $100$.

\item
Пусть натуральные числа $m_1$, $m_2$, \ldots, $m_n$ попарно взаимно просты.
Докажите, что если числа $x_1$, $x_2$, \ldots, $x_n$ пробегают полные системы
вычетов по модулям $m_1$, $m_2$,\ldots, $m_n$ соответственно, то число
\[
    x
=
    x_1 m_2 \ldots m_n
    +
    m_1 x_2 m_3 \ldots m_n
    + \ldots +
    m_1 m_2 \ldots m_{n-1} x_n
\]
пробегает полную систему вычетов по модулю $m_1 m_2 \ldots m_n$.
Выведите отсюда китайскую теорему об остатках.

\item
Найдите наименьшее натуральное число, половина которого~--- квадрат,
треть~--- куб, а пятая часть~--- пятая степень.

\item
Пусть $p \in \mathbb{Z}[x]$ и для любого $n$ значение $p(n)$ делится либо на
$2$, либо на $3$.
Докажите, что либо все числа $p(n)$ делятся на $2$, либо все они делятся на
$3$.

\item
Докажите, что существует сколь угодно длинный отрезок из натуральных чисел,
каждое из которых делится на квадрат некоторого простого числа.

\item
Точку $(x, y) \in \mathbb{Z}^2$ назовем \emph{невидимой} (из начала координат),
если $d = (x, y) > 1$ (её загораживает точка $(x/d, y/d)$, находящаяся на том
же луче).
Докажите, что для любого $k > 0$ найдется такая точка
$(a, b) \in \mathbb{Z}^2$, что каждая из точек $(a + x, b + y)$
($0 \leq x, y < k$) является невидимой.

\item
Докажите, что для любого натурального $k$ существует сколь угодно длинный
отрезок из натуральных чисел, каждое из которых делится по крайне мере на $k$
различных простых чисел.

\item
Докажите, что в наборе $1 \cdot 2$, $2 \cdot 3$, \ldots, $n (n + 1)$
существует $2^{\omega(n)}$ чисел, делящихся на $n$
($\omega(n)$~--- количество простых делителей $n$).

\itemx{*}
Бесконечная арифметическая прогрессия, состоящая из натуральных чисел,
содержит полный квадрат и полный куб.
Докажите, что она содержит шестую степень целого числа.

\end{problems}

