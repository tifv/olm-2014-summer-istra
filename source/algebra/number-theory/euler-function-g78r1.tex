% $date: 2014-06-23

% $timetable:
%   g78r1:
%     2014-06-23:
%     - 2
%     2014-06-24:
%     - 2

\section*{Функция Эйлера}

% $authors:
% - Алексей Устинов

\definition
Функция Эйлера $\varphi(n)$ определяется как количество чисел от~$1$ до~$n$,
взаимно простых с~$n$.

\begin{problems}

\item
Найдите
\\
\sbp $\varphi(17)$;
\qquad
\sbp $\varphi(p)$;
\qquad
\sbp $\varphi(p^2)$;
\qquad
\sbp $\varphi(p^{\alpha})$.

\item
Решите уравнения
\\
\sbp $\varphi(5^x) = 100$;
\qquad
\sbp $\varphi(7^x) = 294$;
\qquad
\sbp $\varphi(3^x \cdot 5^y) = 600$.

\end{problems}

\definition
Функция $f(n)$, определенная на~множестве натуральных чисел называется
\emph{мультипликативной,} если она удовлетворяет двум условиям:
\\
\textbf{(1)} $f(1) = 1$;
\qquad
\textbf{(2)} $f(m \cdot n) = f(m) \cdot f(n)$\enspace при~$(m, n) = 1$.

Если $f(1) = 1$ и~равенство $f(m \cdot n) = f(m) \cdot f(n)$ выполнено для всех
пар натуральных чисел $m$ и~$n$, то~функция $f(n)$ называется
\emph{вполне мультипликативной.}

\begin{problems}

\item
Основным свойством функции Эйлера $\varphi(n)$ является ее~мультипликативность.
Для взаимно простых $a$ и~$b$ рассмотрим таблицу
\[ \begin{array}{ccccc}
    1 & 2 & 3 & \cdots & b
\\
    b + 1 & b + 2 & b + 3 & \cdots & 2 b
\\
    \vdots
    &
    \vdots
    &
    \vdots
    &
    \ddots
    &
    \vdots
\\
    (a - 1) b + 1 & (a - 1) b + 2 \ & (a - 1) b + 3 & \cdots & ab.
\end{array} \]

В~каких столбцах этой таблицы находятся числа, взаимно простые с~числом~$b$?
Сколько в~каждом из~этих столбцов чисел, взаимно простых с~$a$?
Докажите мультипликативность функции Эйлера, ответив на~эти вопросы.

\item
Пусть $n = p_1^{\alpha_1} \ldots p_s^{\alpha_s}$.
Докажите равенство
\[
    \varphi(n)
=
    n \left(
        1 - \frac{1}{p_1}
    \right)
    \ldots
    \left(
        1 - \frac{1}{p_s}
    \right)
\]
\sbp пользуясь мультипликативностью функции Эйлера;
\\
\sbp пользуясь формулой включений и~исключений.

\item
Решите уравнения
\\
\sbp $\varphi(x) = 2$;
\qquad
\sbp $\varphi(x) = 8$;
\qquad
\sbp $\varphi(x) = 12$;
\qquad
\sbp $\varphi(x) = 14$.

\item
Для каких $n$ возможны равенства:
\\
\sbp $\varphi(n) = n - 1$
\qquad
\sbp $\varphi(2 n) = 2 \cdot \varphi(n)$;
\qquad
\sbp $\varphi(n^k) = n^{k-1} \cdot \varphi(n)$?
 
\item
Решите уравнения
\\
\sbp $\varphi(x) = x / 2$;
\qquad
\sbp $\varphi(x) = x / 3$;
\qquad
\sbp $\varphi(x) = x / 4$.

\definition
\emph{Приведенной системой вычетов} по~некоторому модулю $m$ называется система
чисел, взятых по~одному из~каждого класса, взаимно простого с~модулем.
(Говорят, что класс $\ov{a}$ взаимно прост с~модулем $m$, если само число $a$
взаимно просто с~$m$.)

\item
Пусть числа $x_1$, $x_2$, \ldots, $x_r$ образуют приведенную систему вычетов
по~модулю $m$.
Для каких $a$ и~$b$ числа $y_j = a x_j + b$ ($j = 1, \ldots, r$) также образуют
приведенную систему вычетов по~модулю $m$?

\item
По~какому модулю числа $1$ и~$5$ составляют приведенную систему вычетов?

\item
Известно, что $(m, n) > 1$.
Что больше, $\varphi(m \cdot n)$ или $\varphi(m) \cdot \varphi(n)$?

\end{problems}

