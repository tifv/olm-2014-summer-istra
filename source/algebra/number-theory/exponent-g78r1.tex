% $date: 2014-06-19

% $timetable:
%   g78r1:
%     2014-06-19:
%     - 1
%     2014-06-20:
%     - 1

\section*{Серия 3, про показатели}

% $authors:
% - Фёдор Бахарев

\begin{problems}

\item
Натуральное число~$n$ таково, что $2^n - 2$ делится на~$n$.
Докажите, что $2^{2^n-1} - 2$ делится на~$2^n - 1$.

\item
Пусть $p$, $q$~--- простые числа, $q > 5$.
Докажите, что если $q \mid 2^p + 3^p$, то~$q > p$.

\item
Найдите все такие простые числа $p$ и~$q$, что $2^p + 1$ делится на~$q$,
а~$2^q + 1$ делится на~$p$.

\item
Найдите все пары $(p, q)$ простых чисел такие, что число $2^p-1$ делится
на~$q$, и~среди простых делителей числа $q - 1$ имеются только числа
$2$, $3$, $5$ и~$7$.

\item
Найдите все нечетные числа~$n$, такие, что $3^n + 1$ делится на~$n$.

\item
Даны натуральные числа $a$ и~$n$.
Докажите, что $n \mid \varphi(a^n - 1)$.

\item
$a$, $b$, $p$~--- натуральные числа, такие, что $a \equiv b \pmod{p}$.
Докажите, что $a^p - b^p$ делится на~$p^2$.

\item
$p$, $q$, $r$~--- различные простые числа.
Известно, что $p q r$ делится на~$p + q + r$.
Докажите, что число $(p - 1) (q - 1) (r - 1) + 1$ является точным квадратом.

\item
Пусть $p$~--- простое вида $4 n + 1$.
Докажите, что $((2n)!)^2 + 1$ делится на~$p$.

\item
Решите в~натуральных числах уравнение $x^3 + y^3 - 6 x y + 27 = 0$.

\end{problems}

