% $date: 2014-06-18

% $timetable:
%   g78r2:
%     2014-06-18:
%     - 3
%     2014-06-19:
%     - 2

\section*{Серия 3, алгебраический разнобой}

% $authors:
% - Фёдор Бахарев

\begin{problems}

\item 
Вычислите:
\[
    \left(
        1 - \frac{1}{2^2}
    \right)
    \left(
        1 - \frac{1}{3^2}
    \right)
    \left(
        1 - \frac{1}{4^2}
    \right)
    \ldots
    \left(
        1 - \frac{1}{100^2}
    \right)
\;.\]

\item
Вычислите:
\\[0.7ex]
\sbp
\(
    \dfrac{1}{1 \cdot 2} + \dfrac{1}{2 \cdot 3}
    + \ldots +
    \dfrac{1}{99 \cdot 100}
\);
\qquad
\sbp
\(
    \dfrac{2 \cdot 1 + 1}{1^2 \cdot 2^2}
    +
    \dfrac{2 \cdot 2 + 1}{2^2 \cdot 3^2}
    + \ldots +
    \dfrac{2 \cdot 99 + 1}{99^2 \cdot 100^2}
\).

\item
\sbp
Докажите, что у $19^n$ сумма цифр хотя бы $10$.
\\
\sbp
Докажите, что сумма цифр числа $1981^n$ при любом натуральном $n$ не меньше $19$.

\item
$p$, $q$, $r$~--- различные простые числа.
Известно, что $pqr$ делится на $p + q + r$.
Докажите, что число $(p - 1) (q - 1) (r - 1) + 1$ является точным квадратом.

\item
Решите в простых числах:
\(
    \left[\dfrac{p}{2}\right]
    +
    \left[\dfrac{p}{3}\right]
    +
    \left[\dfrac{p}{6}\right]
=
    q
\).

\item
Докажите, что $3^{2^n} - 1$
\quad
\sbp делится на $2^{n+2}$;
\quad
\sbp не делится на $2^{n+3}$.

\end{problems}

