% $date: 2014-06-22

% $timetable:
%   g1011:
%     2014-06-22:
%     - 1

\section*{Квадратный трёхчлен}

% $authors:
% - Олег Орлов

\begin{problems}

\item
\sbp
Докажите, что любой рациональный корень уравнения $x^2 + p x + q = 0$ с~целыми
коэффициентами $p$ и~$q$ является целым числом.
\\
\sbp
Уравнения $x^2 + p_1 x + q_1 = 0$ и~$x^2 + p_2 x + q_2 = 0$ с~целыми
коэффициентами $p_1$, $q_1$, $p_2$, $q_2$ имеют общий нецелый корень.
Докажите, что $p_1 = p_2$, $q_1 = q_2$.

\item
Пусть коэффициенты уравнений $x^2 + p_1 x + q_1 = 0$ и~$x^2 + p_2 x + q_2 = 0$
связаны соотношением $p_1 p_2 = 2 (q_1 + q_2)$.
Докажите, что хотя~бы одно из~этих уравнений имеет решение.

\item
$f(x)$~--- квадратный трехчлен.
Для какого наибольшего количества натуральных значений $n$ может быть верно
$f(n + 2) = f(n + 1) + f(n)$.

\item
Один из~двух приведенных квадратных трехчленов имеет два корня меньших тысячи,
другой~--- два корня больших тысячи.
Может~ли сумма этих трехчленов иметь один корень меньший тысячи, а~другой~---
больший тысячи?

\item
Известно, что $f(x)$ и~$g(x)$~--- квадратные трехчлены.
Может~ли уравнение $f(g(x)) = 0$ иметь корни $2$, $3$, $5$, $7$?

\item
Приведенные квадратные трехчлены $f(x)$ и~$g(x)$ таковы, что оба уравнения
$f(g(x)) = 0$ и~$g(f(x)) = 0$ не~имеют вещественных корней.
Докажите, что хотя~бы одно из~уравнений $f(f(x)) = 0$ и~$g(g(x)) = 0$ тоже
не~имеет вещественных корней.

\end{problems}

