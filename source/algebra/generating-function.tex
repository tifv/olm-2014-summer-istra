% $date: 2014-06-25

% $timetable:
%   g10:
%     2014-06-25:
%     - 1
%     2014-06-26:
%     - 2

\section*{Производящие функции}

% $authors:
% - Алексей Устинов

\definition
Выражения вида
\begin{equation}\label{FSR}
    F(x) = a_0 + a_1 x + a_2 x^2 + \ldots + a_n x^n + \ldots
\end{equation}
называются \emph{формальными степенными рядами.}

Формальные степенные ряды  можно складывать, вычитать, умножать, делить,
дифференцировать и (с~некоторыми ограничениями) устраивать их~композицию,
не~беспокоясь о~сходимости.

\definition
Производной формального степенного ряда~(\ref{FSR}) называется ряд
\[
    F'(x) = a_1 + 2 a_2 x \ldots + n a_n x^{n-1} + \ldots
\]

\begin{problems}

\item
Найдите произведения следующих формальных степенных рядов:
\\[0.2ex]
\sbp $(1 + x + x^2 + x^3 + \ldots) (1 - x + x^2 - x^3 + \ldots)$.
\\[0.3ex]
\sbp $(1 + x + x^2 + x^3 + \ldots)^2$.

\item\textbf{Обращение степенного ряда.}
Докажите, что если $a_0 \neq 0$, то~для ряда $F(x)$ существует ряд
$F^{-1}(x) = b_0 + b_1 x + \ldots + b_n x^n + \ldots$ такой, что
$F(x) F^{-1}(x) = 1$.

\item
Вычислите ряды (найдите коэффициенты):
\\[0.2ex]
\sbp $(1 + x)^{-1}$;
\qquad
\sbp $(1 - x)^{-1}$;
\qquad
\sbp $(1 - x)^{-2}$.

\end{problems}

\definition
Пусть $\{a_n\} = a_0, a_1, \ldots$~--- произвольная числовая
последовательность.
Формальный степенной ряд
\(
    F(x) = a_0 + a_1 x + \ldots + a_n x^n + \ldots
\)
будем называть производящей функцией этой последовательности.

\begin{problems}

\item
Вычислите производящие функции следующих последовательностей
(выразите их~как <<конечные>> выражения, т.~е. рациональные функции от~$x$):
\\[0.2ex]
\sbp $a_n = n$;
\qquad
\sbp $a_n = n^2$;
\qquad
\sbp $a_n = C_m^n$.

\item
Вычислите суммы:
\\[0.2ex]
\sbp $C_n^1 + 2 C_n^2 + 3 C_n^3 + \ldots + n C_n^n$;
\\[0.3ex]
\sbp $C_n^1 + 2^2 C_n^2 + 3^2 C_n^3 + \ldots + n^2 C_n^n$.

\item\emph{Счастливые билеты.}
Предположим, что у~нас имеется $1\,000\,000$ автобусных билетов с~номерами
от~$000\,000$ до~$999\,999$.
Будем называть билет счастливым, если сумма первых трех цифр его номера равна
сумме трех последних.
Пусть $N$~--- количество счастливых билетов.
Докажите равенства:
\\[0.2ex]
\sbp{\small%XXX
\(
    (1 + x + \ldots + x^9)^3
    \cdot
    (1 + x^{-1} + \ldots + x^{-9})^3
=
    x^{27} + \ldots + a_1 x + N + a_1 x^{-1} + \ldots + x^{-27}
\);}
\sbp
\(
    (1 + x + \ldots + x^9)^6
=
    1 + \ldots + N x^{27} + \ldots + x^{54}
\).

\item
Найдите число счастливых билетов.

\item
Докажите, что производящая функция последовательности чисел Фибоначчи
$F(z) = F_0 + F_1 z + F_2 z^2 + \ldots$ может быть записана в~виде
\[
    F(z)
=
    \frac{z}{1 - z - z^2}
=
    \frac{1}{\sqrt{5}}
    \left(
        \frac{1}{1 - \phi z} - \frac{1}{1 - \widehat{\phi} z}
    \right)
,\]
где
$\phi = \frac{1 + \sqrt{5}}{2}$,
$\widehat{\phi} = \frac{1 - \sqrt{5}}{2}$.
\\
\emph{(Для знающих производную.)} Докажите формулу Бине.

\item
Пусть
\(
    C(z)
=
    \sum\limits_{n = 0}^{\infty}
        C_n z^n
\)~--- производящая функция последовательности чисел Каталана $\{C_n\}$.
Докажите, что она удовлетворяет равенству $C(z) = z C^2(z) + 1$, и~получите
явный вид функции $C(z)$.

\item
Выведите формулу для чисел Каталана,
воспользовавшись результатом предыдущей задачи и~равенством
\(
    (1 - z)^{1/2}
=
    \sum\limits_{n=0}^{\infty}
        C_{1/2}^n (-z)^n
,\)
где $C_{1/2}^n$~--- обобщённые биномиальные коэффициенты:
\[
    C_{1/2}^n
=
    \frac{
        (1/2) (1/2 - 1) \ldots (1/2 - n + 1 )
    }{n!}
\]

\end{problems}

