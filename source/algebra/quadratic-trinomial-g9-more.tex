% $date: 2014-06-18

% $timetable:
%   g9: 
%     2014-06-18:
%     - 2

\section*{Ещё квадратный трёхчлен}

% $authors:
% - Юлий Тихонов

% $build$matter[print]: [[.], [.]]
% $build$style[print]:
% - .[tiled4,-print]

\begin{problems}

\item
Пусть у трехчлена $a x^2 + b x + c$ с целыми $a$, $b$, $c$ есть два корня, один
из которых рациональный.
\\
\sbp
Верно ли, что второй корень тоже рациональный?
\\
\sbp
Пусть первый корень целый.
Верно ли, что второй корень целый?
\\
\sbp
Пусть первый корень целый и $a = -1$.
Верно ли, что второй корень целый?
\\
\sbp
Пусть $a = 1$, а про первый корень известна только рациональность.
Докажите, что всё равно оба корня целые.

\end{problems}

\observation
Квадратный трехчлен (с ненулевым коэффициентом при $x^2$) принимает каждое
своё значение не более двух раз.

\begin{problems}

\item
Пусть $f(x)$, $g(x)$ и $h(x)$~--- квадратные трехчлены.
Докажите, что у уравнения $f(g(h(x))) = 0$ не более восьми корней.

\item
Пусть $f(x)$~--- квадратный трехчлен, не являющийся константой, который в целых
$x$ принимает только значения вида $k^4$, где $k$ целое.
\\
\sbp
Докажите, что максимум $f(x)$ при $x \in [0; 2n]$ не меньше $n^4$.
\\
\sbp
Докажите, что при всех достаточно больших $n$ выполнено $f(2n) \geq n^4$.
\\
\sbp
Докажите, что таких трехчленов $f(x)$ не бывает.

\item
Пусть $f(x)$~--- квадратный трехчлен, не являющийся константой, который может
принимать в целых точках только значения $\pm 2^k 3^l 5^m$,
где $k$, $l$, $m$ натуральные.
Докажите, что таких трехчленов не бывает.

\end{problems}

