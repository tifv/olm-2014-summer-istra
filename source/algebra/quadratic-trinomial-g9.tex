% $date: 2014-06-16

% $timetable:
%   g9: 
%     2014-06-16:
%     - 1

\section*{Квадратный трёхчлен}

% $authors:
% - Юлий Тихонов

\theoremof{Виета}
Пусть $x_1$ и $x_2$~--- корни уравнения $x^2 + p x + q = 0$.
Тогда $x_1 \cdot x_2 = q$ и $x_1 + x_2 = -p$.

\begin{problems}

\itemy{0}
Сформулируйте и докажите обратное к теореме Виета утверждение.

\item
Даны вещественные числа $x_1$, $x_2$, $y_1$, $y_2$ такие, что
$x_1 + x_2 = y_1 + y_2$ и $x_1 x_2 = y_1 y_2$.
Докажите, что наборы $\{x_1, x_2\}$ и $\{y_1, y_2\}$ совпадают.

\item
При каком значении параметра $m$ сумма квадратов корней уравнения
\[
    x^2 - (m + 1) x + m - 1
=
    0
\]
является наименьшей?

\item
Для многочленов $f(x) = x^2 + a x + b$ и $g(y) = y^2 + p y + q$ с корнями
$x_1$, $x_2$ и $y_1$, $y_2$ соответственно, выразите через $a$, $b$, $p$, $q$
их \emph{результант}
\[
    R(f, g) = (x_1 - y_1) (x_1 - y_2) (x_2 - y_1) (x_2 - y_2)
.\]

\item
Пусть $x_1$ и $x_2$~--- корни квадратного уравнения $x^2 - 3 x - 5 = 0$.
Составьте квадратное уравнение, корнями которого являются числа:
\\
\sbp
$x_1 + \frac{1}{x_1}$ и~$x_2 + \frac{1}{x_2}$;
\qquad
\sbp
$x_1 + \frac{1}{x_2}$ и~$x_2 + \frac{1}{x_1}$.

\end{problems}


\subsection*{Ещё задачи}

\begin{problems}

\item
Один из двух приведенных квадратных трехчленов имеет два корня меньших тысячи,
другой — два корня больших тысячи.
Может ли сумма этих трехчленов иметь один корень меньший тысячи,
а другой — больший тысячи?

\item
Известно, что
$f(x)$, $g(x)$ и $h(x)$~--- квадратные трехчлены.
Может ли уравнение $f(g(h(x))) = 0$ иметь корни
$1$, $2$, $3$, $4$, $5$, $6$, $7$ и $8$?

\item
У квадратного уравнения $x^2 + p x + q = 0$ коэффициенты $p$ и $q$ увеличили на
единицу.
Эту операцию повторили девять раз.
Могло ли оказаться, что у каждого из десяти полученных уравнений корни~---
целые числа?

\item
Есть две параболы $y = x^2 + x - 41$ и $x = y^2 + y - 40$.
Докажите, что точки их пересечения лежат на одной окружности.

\item
Рассмотрим графики функций
$y = x^2 + p x + q$,
которые пересекают оси координат в трех различных точках.
Докажите, что все окружности, описанные около треугольников с вершинами в этих
точках, имеют общую точку.

\item
Квадратный трехчлен $f(x)$ разрешается заменить на один из трехчленов
\[
    x^2
    \cdot
    f\left(
        \frac{1}{x} + 1
    \right)
\text{\quadили\quad}
    (x - 1)^2
    \cdot
    f\left(
        \frac{1}{x - 1}
    \right)
.\]
Можно ли с помощью таких операций из квадратного трехчлена
$x^2 + 4 x + 3$
получить трехчлен
$x^2 + 10 x + 9$?

\item
Трёхчлен $a x^2 + b x + c$ при всех целых $x$ является точной четвертой
степенью.
Докажите, что тогда $a = b = 0$.

\end{problems}

