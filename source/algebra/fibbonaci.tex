% $date: 2014-06-21

% $timetable:
%   g78r1:
%     2014-06-21:
%     - 1

\section*{Числа Фиббоначи}

% $authors:
% - Алексей Устинов

\definition
Последовательность чисел Фибоначчи
\[
    \{F_0, \, F_1, \, F_2, \, \ldots\}
=
    \{
        0, \, 1, \, 1, \, 2, \, 3, \, 5, \, 8, \,
        13, \, 21, \, 34, \, 55, \, 89, \,
        144, \, 233, \, 377, \, 610, \, \ldots
    \}
\]
задается условиями $F_0 = 0$, $F_1 = 1$,
$F_{n+2} = F_{n+1} + F_n$ ($n \geq 0$).

% spell "1202\,г." -> "1202 год"

\emph{Эти числа были впервые описаны в <<Книге абака>> (1202\,г.)
итальянского математика Леонардо Пизанского (Фибоначчи).}

\begin{problems}

\item\emph{Задача Леонардо Пизанского.}
Некто приобрел пару кроликов и поместил их в огороженный со всех сторон загон.
Сколько кроликов будет через год, если считать, что каждый месяц пара дает в
качестве приплода новую пару кроликов, которые со второго месяца жизни также
начинают приносить приплод?

\item\label{kuz}\emph{О том, как прыгают кузнечики.}
Предположим, что имеется лента, разбитая на клетки и уходящая вправо до
бесконечности.
На первой клетке этой ленты сидит кузнечик.
Из любой клетки кузнечик может перепрыгнуть либо на одну, либо на две клетки
вправо.
Сколькими способами кузнечик может добраться до $n$-ой от начала ленты клетки?

\item
Некоторый алфавит состоит из $6$ букв,
которые для передачи по телеграфу кодированы так:
\[
    \cdot
\qquad
    -
\qquad
    {\cdot}\,{\cdot}
\qquad
    {-}\,{-}
\qquad
    {\cdot}\,{-}
\qquad
    {-}\,{\cdot}
\]
При передаче одного слова не сделали промежутков, отделяющих букву от буквы,
так что получилась сплошная цепочка из точек и тире, содержащая $12$ знаков.
Сколькими способами можно прочитать переданное слово?

\item
Чему равны числа Фибоначчи с отрицательными номерами
$F_{-1}$, $F_{-2}$, \ldots, $F_{-n}$, \ldots?

\item\emph{Делимость чисел Фибоначчи.}
Докажите справедливость следующих утверждений:
\\
\sbp $2 \mid F_n \quad\Leftrightarrow\quad 3 \mid n$;
\qquad
\sbp $3 \mid F_n \quad\Leftrightarrow\quad 4 \mid n$;
\\
\sbp $4 \mid F_n \quad\Leftrightarrow\quad 6 \mid n$;
\\
\sbp $F_m \mid F_n \quad\Leftrightarrow\quad m \mid n$ (при $F_m > 1$).

\item
Докажите, что для любого натурального $m$ существует число Фибоначчи
$F_n$ ($n \geq 1$), кратное $m$.

\item\emph{Тождество Кассини.}
Докажите равенство для $n > 0$:
\[
    F_{n+1} F_{n-1} - F_n^2
=
    (-1)^n
.\]
Будет ли тождество Кассини справедливо для всех целых $n$?

\item
Докажите следующие свойства чисел Фибоначчи:
\\
\sbp $F_1 + F_2 + \ldots + F_n = F_{n+2} - 1$;
\qquad
\sbp $F_1 + F_3 + \ldots + F_{2n-1} = F_{2n}$;
\\
\sbp $F_2 + F_4 + \ldots + F_{2n} = F_{2n+1} - 1$;
\qquad
\sbp $F_1^2 + F_2^2 + \ldots + F_n^2 = F_n F_{n+1}$.

\item
Докажите, что при $n \geq 1$ и $m \geq 0$ выполняется равенство
\[
    F_{n+m} = F_{n-1} F_m + F_n F_{m+1}
.\]
Попробуйте доказать его двумя способами: при помощи метода
математической индукции и при помощи
%кузнечика
интерпретации чисел Фибоначчи из задачи~\ref{kuz}.
Докажите также, что тождество Кассини является частным случаем этого равенства.

\item
Докажите равенства
\\
\sbp $F_{2n+1} = F_n^2 + F_{n+1}^2$;
\qquad
\sbp $F_{n+1} F_{n+2} - F_n F_{n+3} = (-1)^{n}$;
\\
\sbp $F_{3n} = F_n^3 + F_{n+1}^3 - F_{n-1}^3$.

\item
Вычислите $F_{n+2}^4 - F_n F_{n + 1} F_{n + 3} F_{n+4}$.

\item
Вычислите сумму
\[
    \frac{1}{1 \cdot 2} +
    \frac{2}{1 \cdot 3}
    + \ldots +
    \frac{F_{n}}{F_{n-1} \cdot F_{n+1}}
\;.\]

\end{problems}

