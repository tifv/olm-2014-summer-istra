% $date: 2014-06-24

% $timetable:
%   g1011:
%     2014-06-24:
%     - 1

\section*{Добавка по алгебре и теории чисел}

% $authors:
% - Олег Орлов

\begin{problems}

\item
Докажите, что ни~при каком целом~$k$ число $k^2 + k + 1$ не~делится на~$101$.

\item
Рассмотрим все натуральные числа, в~десятичной записи которых отсутствует ноль.
Докажите, что сумма обратных величин любого количества из~этих чисел
не~превосходит некоторого числа~$C$.

\item
Существуют~ли три попарно различных ненулевых целых числа, сумма которых равна
нулю, а~сумма тринадцатых степеней которых является квадратом некоторого
натурального числа?

\item
Два многочлена
$P(x) = x^4 + a x^3 + b x^2 + c x + d$ и $Q(x) = x^2 + p x + q$
принимают отрицательные значения на~некотором интервале $I$ длины более~$2$,
а~вне $I$~--- неотрицательны.
Докажите, что найдется такая точка $x_0$, что $P(x_0) < Q(x_0)$.

\item
Докажите неравенство для положительных $a_1$, $a_2$, $\ldots$, $a_n$:
\[
    \left(
        1 + \frac{1}{a_1 (1 + a_1)}
    \right)
    \cdot
    \left(
        1 + \frac{1}{a_2 (1 + a_2)}
    \right)
    \cdots
    \left(
        1 + \frac{1}{a_n (1 + a_n)}
    \right)
\geq
    \left(
        1 + \frac{1}{a (1 + a)}
    \right)^{\!n}
\]
где $a$~--- среднее геометрическое чисел $a_1$, $a_2$, $\ldots$, $a_n$.
    
\end{problems}

