% $date: 2014-06-24

% $timetable:
%   g11:
%     2014-06-24:
%     - 2
%     2014-06-25:
%     - 2

\section*{Добавка по алгебре и теории чисел}

% $authors:
% - Олег Орлов

\begin{problems}

\item
Два многочлена $P(x) = x^4 + a x^3 + b x^2 + c x + d$ и $Q(x) = x^2 + p x + q$
принимают отрицательные значения на некотором интервале $I$ длины более $2$,
а~вне $I$~--- неотрицательны.
Докажите, что найдется такая точка $x_0$, что $P(x_0) < Q(x_0)$.

\item
Последовательность натуральных чисел $c_1$, $c_2$, $c_3$, $\ldots$ такова,
что для любых натуральных $m$ и~$n$, удовлетворяющих условию
$1 \leq m \leq \sum_{i=1}^{n} c_i$,
найдутся такие натуральные числа $a_1$, $a_2$, $\ldots$, $a_n$, что
$m = \sum_{i=1}^{n} (c_i / a_i)$.
Для каждого $i$ найдите максимально возможное значение величины $c_i$.

\item
Для заданного натурального числа $k > 1$ через $Q(n)$, $n \in \mathbb{N}$,
обозначим наименьшее общее кратное чисел $n$, $n+1$, $\ldots$, $n+k$.
Докажите, что существует бесконечно много $n \in \mathbb{N}$ таких, что
$Q(n) > Q(n+1)$.

\item
Пусть натуральные числа $x$, $y$, $p$, $n$, $k$ таковы, что $x^n + y^n = p^k$.
Докажите, что если число $n$ ($n > 1$) нечетное, а~число $p$ нечетное простое,
то $n$ является степенью числа $p$ (с натуральным показателем).

\item
Для чисел $x$, $y$, $z$ из~отрезка $[1; 2]$ докажите неравенство
\[
    (x + y + z)
    \left(
        \frac{1}{x} + \frac{1}{y} + \frac{1}{z}
    \right)
\geq
    6
    \left(
        \frac{x}{y + z} + \frac{y}{x + z} + \frac{z}{x + y}
    \right)
\]

\end{problems}

