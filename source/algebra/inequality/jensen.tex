% $date: 2014-06-16

% $timetable:
%   g11:
%     2014-06-16:
%     - 2

\section*{Неравенство Йенсена}

% $authors:
% - Олег Орлов

\definition
Функция $f(x)$ называется \emph{выпуклой} на~отрезке $[a; b]$, если для любых
$x_1$, $x_2$ лежащих на~этом отрезке и~для любого $0 < \alpha < 1$ выполнено
неравенство
\[
    \alpha \cdot f(x_1) + (1 - \alpha) \cdot f(x_2)
\geq
    f(\alpha x_1 + (1 - \alpha) x_2)
\]

\theoremof{Лагранжа}
Если функция дифференцируема на~отрезке $[a; b]$, то~существует такая точка $c$
лежащая на~этом отрезке, что
\[
    f'(c) = \frac{f(b) - f(a)}{b - a}
\]

\begin{problems}

\item
Рассмотрим функцию $f(x)$, которая в~каждой точке на~отрезке $[a; b]$ имеет
вторую производную.
Докажите с~помощью теоремы Лагранжа, что если первая производная монотонно
не~убывает на~отрезке $[a; b]$ (т.~е. вторая производная неотрицательна),
то~функция выпукла на~этом отрезке.

\item\textbf{Неравенство Йенсена.}
Докажите, что если функция выпукла на~отрезке $[a; b]$, то~для любых
неотрицательных $\alpha_1$, $\alpha_2$, $\ldots$, $\alpha_n$, таких, что
$\alpha_1 + \alpha_2 + \ldots + \alpha_n = 1$ и~$x_1$, $x_2$, $\ldots$, $x_n$
лежащих на~$[a; b]$ выполняется неравенство:
\[
    \alpha_1 \cdot f(x_1) + \alpha_2 \cdot f(x_2)
    + \ldots +
    \alpha_n \cdot f(x_n)
\geq
    f(\alpha_1 \cdot x_1 + \alpha_2 \cdot x_2 + \ldots + \alpha_n \cdot x_n)
\]

\item
Пусть $x_1, x_2, \ldots, x_n \in [0;\pi]$.
Докажите, что
\\
\sbp
\(
    \sin x_1 + \sin x_2 + \ldots + \sin x_n
\leq
    n \cdot \sin \left(
        \frac{x_1 + x_2 + \ldots + x_n}{n}
    \right)
\)
\\
\sbp
\(
    \sin x_1 \cdot \sin x_2 \cdot \ldots \cdot \sin x_n
\leq
    \sin^n \left(
        \frac{x_1 + x_2 + \ldots + x_n}{n}
    \right)
\)

\item
Обозначим при положительных $x_1$, $x_2$, $\ldots$, $x_n$ и~$p \neq 0$
\[
    M_p
=
    \left(
        \frac{x_1^p + x_2^p + \ldots + x_n^p}{n}
    \right)^{\frac{1}{p}}
,\qquad
    M_0
=
    \sqrt[n]{x_1 x_2 \ldots x_n}
\;.\]
Докажите, что $M_p \geq M_q$, если
\\
\sbp $p > q > 0$ или $0 > p > q$;
\qquad
\sbp $p > q = 0$, $0 = p > q$.

% spell "Коши-Буняковского-Шварца" -> "Коши Буняковского Шварца"

\item
Выведите неравенство Коши-Буняковского-Шварца из~выпуклости функции $1/x$.

\item
Пусть функция $f(x)$ непрерывна на~отрезке $[a; b]$ и~для любых $x_1$, $x_2$
лежащих на~этом отрезке выполняется следующее неравенство:
\[
    \frac{f(x_1) + f(x_2)}{2}
\geq
    f\left(\frac{x_1 + x_2}{2}\right)
\;.\]
Докажите, что $f(x)$ выпукла на~$[a; b]$.

\item
Докажите неравенство для $x_1, x_2, \ldots x_n \geq 1$:
\[
    \frac{1}{1 + x_1} + \frac{1}{1 + x_2} + \ldots + \frac{1}{1 + x_n}
\geq
    \frac{n}{1 + \sqrt[n]{x_1 \cdot x_2 \cdot \ldots \cdot x_n}}
\;.\]

\end{problems}

