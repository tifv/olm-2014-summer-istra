% $date: 2014-06-19

% $timetable:
%   g11:
%     2014-06-19:
%     - 2

\section*{Неравенства}

% $authors:
% - Олег Орлов

\begin{problems}

\item
Пусть положительные числа $a$, $b$, $c$ таковы, что
$1/a + 1/b + 1/c \geq a + b + c$.
Докажите тогда, что $a + b + c \geq 3 a b c$.

\item
Для чисел $x$, $y$ из отрезка $[0; 1]$ докажите неравенство
\[
    \frac{1}{\sqrt{\mathstrut1 + x^2}}
    +
    \frac{1}{\sqrt{\mathstrut1 + y^2}}
\leq
    \frac{2}{\sqrt{\mathstrut1 + x y}}
\]
    
\item
\sbp
Для положительных $a_1$, $a_2$, $\ldots$, $a_n$, $b_1$, $b_2$, $\ldots$, $b_n$
докажите неравенство
\[
    \frac{a_1^2}{b_1} + \frac{a_2^2}{b_2} + \ldots + \frac{a_n^2}{b_n}
\geq
    \frac{(a_1 + a_2 + \ldots + a_n)^2}{b_1 + b_2 + \ldots + b_n}
\]
\sbp
Для положительных $a_1$, $a_2$, $\ldots$, $a_n$ докажите неравенство
\[
    \frac{a_1^3}{a_1 + a_2} + \frac{a_2^3}{a_2 + a_3}
    + \ldots +
    \frac{a_n^3}{a_n + a_1}
\geq
    \frac{1}{2}
    (a_1^2 + a_2^2 + \ldots + a_n^2)
\]

\item
Докажите неравенство для чисел $x$, $y$, $z$ из отрезка $[0; 1]$:
\[
    3 (x^2 y^2 + y^2 z^2 + z^2 x^2) - 2 x y z (x + y + z)
\leq
    3
\]

\item
Для положительных $a$, $b$, $c$ докажите неравенство
\[
    \frac{a^3}{a^2 + a b + b^2} +
    \frac{b^3}{b^2 + b c + c^2} +
    \frac{c^3}{c^2 + c a + a^2}
\geq
    \frac{1}{3}
    (a + b + c)
\]

%\item $a,b,c,d$ - положительные числа. Докажите, что $$\cfrac1a + \cfrac1b + \cfrac4c + \cfrac{16}d \geq \cfrac{64}{a+b+c+d}$$

\item
Положительные числа $a$, $b$ и $c$ таковы, что $a b c = 1$.
Докажите неравенство
\[
    \frac{1}{1 + a + b} + \frac{1}{1 + b + c} + \frac{1}{1 + a + c}
\leq
    1
\]
    %Московская математическая олимпиада. 1997г. 11класс. Задача №5.%
    %Турнир городов. Весна, 1997г. Основной вариант. 10-11класс. Задача №5 (№ 42 в книге "1000 задач турнира городов").%
    
%\item Докажите неравенство для положительных $a_1,a_2,\ldots,a_n$:$$\left(1+ \cfrac1{a_1(a_1+1)}\right) + \left(1+ \cfrac1{a_2(a_2+1)}\right) \dots \left(1+ \cfrac1{a_n(a_n+1)}\right) \geq \left(1+ \cfrac1{a(a+1)}\right)^n$$ ,где $a$ - среднее геометрическое чисел $a_1,a_2, \ldots, a_n$.

\end{problems}

