% $date: 2014-06-18

% $timetable:
%   g10:
%     2014-06-18:
%     - 1
%     2014-06-19:
%     - 2

\section*{Интерполяционный многочлен Лагранжа}

% $authors:
% - Алексей Устинов

\subsection*{Начало}

\begin{problems}

\item
Решите уравнение
\[
    c \frac{(x - a) (x - b)}{(c - a) (c - b)} +
    b \frac{(x - a) (x - c)}{(b - a) (b - c)} +
    a \frac{(x - b) (x - c)}{(a - b) (a - c)}
=
    x
.\]

\item
Докажите тождество
\[
    c^2 \frac{(x - a) (x - b)}{(c - a) (c - b)} +
    b^2 \frac{(x - a) (x - c)}{(b - a) (b - c)} +
    a^2 \frac{(x - b) (x - c)}{(a - b) (a - c)}
=
    x^2
.\]

\item
Пусть $x_1 < x_2 < \ldots < x_n$~--- действительные числа.
Постройте многочлены $f_1(x)$, $f_2(x)$, \ldots, $f_n(x)$ степени $n-1$,
которые удовлетворяют условиям $f_i(x_i) = 1$ и $f_i(x_j) = 0$ при
$i \neq j$ ($i, j = 1, 2, \ldots, n$).

\item
Опишите явный вид многочлена $f(x) = f_1(x) + f_2(x) + \ldots + f_n(x)$,
где $f_i(x)$~---  многочлены из предыдущей задачи.

\item
Пусть $x_1 < x_2 < \ldots < x_n$~--- действительные числа.
Докажите, что для любых $y_1$, $y_2$, \ldots, $y_n$ существует единственный
многочлен $f(x)$ степени не выше $n-1$ такой, что
$f(x_1) = y_1$, $\ldots$, $f(x_n) = y_n$.

\item
Какие остатки дает многочлен $f(x)$ из предыдущей задачи на многочлены вида
$(x - x_i)$?
Соотнесите этот факт с китайской теоремой об остатках для многочленов.

\end{problems}

\definition
Многочлен степени не выше $n - 1$, значения которого в данных точках
$x_1$, \ldots, $x_n$ \emph{(узлах интерполяции)} совпадают с заданными числами
$y_1$, \ldots, $y_n$, называется \emph{интерполяционным многочленом Лагранжа}.

\begin{problems}

\item
Пусть $A$, $B$ и $C$~--- остатки от деления многочлена $P(x)$ на
$x - a$, $x - b$ и $x - c$.
Найдите остаток от деления того же многочлена на произведение
$(x - a) (x - b) (x - c)$.

\item
Постройте многочлены $f(x)$ степени не выше 2, которые удовлетворяют условиям:
\\
\sbp $f(0) = 1$, $f(1) = 3$, $f(2) = 3$;
\\
\sbp
$f(-1) = -1$, $f(0) = 2$, $f(1) = 5$;
\\
\sbp
$f(-1) = 1$, $f(0) = 0$, $f(2) = 4$.

\item
Корабль с постоянной скоростью проплывает мимо небольшого острова.
Капитан каждый час измеряет расстояние до острова.
В $12$, $14$ и $15$ часов расстояния равнялись $7$,\ $5$ и $11$ километров
соответственно.
Каким было расстояние до острова в $13$ часов?
Чему оно будет равно в $16$ часов?

%\item Два корабля двигаются с постоянными скоростями.
%Расстояния между ними, измеренные в $12$, $14$ и $15$ часов
%равнялись $5$, $7$ и $2$ километра соответственно. Каким было
%расстояние между кораблями в $13$ часов?

\item
На плоскости расположено $100$ точек.
Известно, что через каждые четыре из них проходит график некоторого квадратного
трехчлена.
Докажите, что все $100$ точек лежат на графике одного квадратного трехчлена.

\item
Решите систему
\[ \left\{ \begin{array}{rcl}
    z + a y + a^2 x + a^3 & = & 0
\\
    z + b y + b^2 x + b^3 & = & 0
\\
    z + c y + c^2 x + c^3 & = & 0
\end{array} \right. \]

%\item Пусть $a$, $b$ и $c$~--- три различных числа.
%Докажите, что из равенств $$\left\{\begin{array}{rcl}
%x+ay+a^2z&=&0,\\x+by+b^2z&=&0,\\x+cy+c^2z&=&0,
%\end{array}\right.$$ следует, что $x=y=z=0$.

\item
Про многочлен $f(x) = x^{10} + a_9 x^9 + \ldots + a_0$ известно, что
\[
    f(1) = f(-1)
,\quad\ldots,\quad
    f(5) = f(-5)
.\]
Докажите, что $f(x) = f(-x)$ для любого действительного $x$.

\item
Пусть $P(x) = a_n x^n + \ldots + a_1 x + a_0$~--- многочлен с целыми
коэффициентами.
Докажите, что хотя бы одно из чисел
$|3^{n+1} - P(n + 1)|$, \ldots, $|3^{1} - P(1)|$, $|1 - P(0)|$
не меньше $1$.

\end{problems}


\subsection*{Продолжение}

\begin{problems}

\item
Докажите, что если $f(x)$ есть многочлен, степень которого меньше $n$, то дробь
\[
    \frac{f(x)}{(x - x_1) (x - x_2) \ldots (x - x_n)}
\]
($x_1$, $x_2$, \ldots, $x_n$~--- произвольные попарно различные числа)
может быть представлена в виде суммы $n$ простейших дробей:
\[
    \frac{A_1}{x - x_1} + \frac{A_2}{x - x_2} + \ldots + \frac{A_n}{x - x_n}
\;,\]
где $A_1$, $A_2$, \ldots, $A_n$ некоторые константы.

\item
Решите систему
\[ \left\{ \begin{aligned} &
    \frac{x_1}{a_1 - b_1} + \frac{x_2}{a_1 - b_2}
    + \ldots +
    \frac{x_n}{a_1 - b_n}
=
    1
\\ &
    \frac{x_1}{a_2 - b_1} + \frac{x_2}{a_2 - b_2}
    + \ldots +
    \frac{x_n}{a_2 - b_n}
=
    1
\\ & \ldots \\ &
    \frac{x_1}{a_n - b_1} + \frac{x_2}{a_n - b_2}
    + \ldots +
    \frac{x_n}{a_n - b_n}
=
    1
\end{aligned} \right. \]

\item
Пусть $F_1 = F_2 = 1$, $F_{n+2} = F_n + F_{n+1}$ и $f$~--- полином степени
$990$ такой, что $f(k) = F_k$ при $k \in \left\{992, \ldots, 1982\right\}$.
Докажите, что $f(1983) = F_{1983} - 1$.

\item
Пусть $f$~--- полином с целыми коэффициентами, и $p$~--- простое число такое,
что $f(0) = 0$, $f(1) = 1$, и $f(k)$ сравнимо либо с $0$, либо с $1$ по модулю
$p$ для всех целых $k$.
Докажите, что степень $f$ не меньше $p - 1$.

\item
Пусть $a_1$, $a_2$, \ldots, $a_n$~--- попарно различные целые числа.
Докажите, что для произвольного натурального $k$ число
\[
    \sum_{i=1}^{n}
        \cfrac{a_i^k}{
            \prod_{j \neq i}
                (a_i - a_j)
        }
\]
является целым.

\end{problems}

