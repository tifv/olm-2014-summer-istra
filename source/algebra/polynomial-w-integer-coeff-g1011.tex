% $date: 2014-06-26

% $timetable:
%   g1011:
%     2014-06-26:
%     - 2

\section*{Многочлены с целыми коэффициентами}

% $authors:
% - Виктор Трещёв

\begin{problems}

\item
Докажите, что если многочлен с~целыми коэффициентами при трех различных целых
значениях переменной принимает значение~1, то~он~не~имеет ни~одного целого
корня.

\item
Докажите, что многочлен степени $n$, принимающий в~$n + 1$ подряд идущих целых
точках только целые значения, принимает целые значения во~всех целых точках.

\item
Докажите, что для любого многочлена $P(x)$ с~целыми коэффициентами и~любых
различных целых чисел $a$ и~$b$ число $P(a) - P(b)$ делится на $a - b$.

\item
Докажите, что не~существует многочлена $P(x)$ с~целыми коэффициентами, для
которого $P(6) = 5$ и~$P(14) = 9$.

\item
Докажите, что для любого многочлена $P(x)$ степени $n$ с~натуральными
коэффициентами найдется такое целое число $k$, что числа
$P(k)$, $P(k + 1)$, \ldots, $P(k + 1996)$ будут составными.

\item
Дан многочлен двадцатой степени с~целыми коэффициентами.
На~плоскости отметили все точки с~целыми координатами, у~которых ординаты
не~меньше~0 и~не~больше~10.
Какое наибольшее число отмеченных точек может лежать на~графике этого
многочлена?

\end{problems}

