% $date: 2014-06-23

% $timetable:
%   g9:
%     2014-06-23:
%     - 1
%     - 2

\section*{Письменная олимпиада, 9 класс}

% $authors:
% - Михаил Харитонов

\begin{problems}

\item
Найдите наименьшее число~$x$, удовлетворяющее уравнению~$x^2 - [x]^2 = 2012$.
($[x]$ обозначает наибольшее целое число, не превосходящее~$x$.)

\item
На~столе лежит стопка из~300 карточек, в~которой содержится ровно по~100 белых,
черных и~красных карточек.
Для каждой белой карточки подсчитаем количество черных, лежащих ниже её,
для каждой черной~--- количество красных, лежащих ниже её,
для каждой красной~--- количество белых, лежащих ниже её.
Найдите наибольшее возможное значение суммы 300 получившихся чисел.

\item
На~сторонах $AB$ и~$AC$ треугольника $ABC$ внешним образом построены квадраты
$ABKL$ и~$ACMN$.
Докажите, что перпендикуляр к~отрезку~$LN$, проходящий через точку~$A$, делит
сторону~$BC$ пополам.

\item
Докажите, что найдется бесконечное число натуральных чисел, не~представимых
в~виде $p + n^2$, где $p$~--- простое, $n$~--- натуральное число.

\item
В~летней школе состоялся шахматный турнир в~один круг.
Исход партии называется \emph{неожиданным}, если в~этой партии игрок, набравший
в~турнире меньшее количество очков, обыграл игрока, набравшего большее
количество очков.
Может~ли доля партий с~\emph{неожиданным} исходом быть больше $75\%$?
Напомним, что в~шахматах в~случае победы игрок получает одно очко, в~случае
ничьи~--- пол-очка, в~случае поражения~--- ноль очков.

\end{problems}

