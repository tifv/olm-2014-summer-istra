% $date: 2014-06-23

% $timetable:
%   g10:
%     2014-06-23:
%     - 1
%     - 2

\section*{Письменная олимпиада, 10 класс}

% $authors:
% - Алексей Доледенок

\begin{problems}

\item
Решите уравнение $[x^3] = \{x^2\} + 11$
(где $[x]$~--- наибольшее целое число, не~превосходящее $x$,
а~$\{x\} = x - [x]$).

\item
За~какое минимальное число дней $n$~футбольных команд могут сыграть круговой
турнир?
Каждая команда должна сыграть с~каждой, в~один день одна команда может играть
не~более одного матча.

\item
На~каждой стороне треугольника взято по~две точки так, что все шесть отрезков,
соединяющих эти точки с~противолежащей вершиной, равны между собой.
Докажите, что середины этих отрезков лежат на~одной окружности. 

\item
В~Мексиканском заливе плавает нефтяное пятно.
Каждый день оно мгновенно меняется по~следующему закону: вокруг каждой точки
залива очерчивается круг радиуса 1~миля, считается площадь попавшей в~круг
части пятна, и, если эта площадь превышает половину площади круга,
то~тогда и~только тогда точка становится покрытой пятном.
Все точки меняются одновременно.
Докажите, что когда-нибудь пятно исчезнет.

\end{problems}

