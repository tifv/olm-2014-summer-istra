% $date: 2014-06-23

% $timetable:
%   g1011:
%     2014-06-23:
%     - 1
%     - 2

\section*{Письменная олимпиада, 10--11 классы}

% $authors: # in alphabetical order
% - Сергей Беляков
% - Антон Гусев
% - Андрей Кушнир
% - Олег Орлов

\begin{problems}

\item
Найдите минимальное положительное число~$x$, удовлетворяющее неравенству:
$x^2 - [x]^2 \geq 2014$.

\item
Пусть $P(m, n)$~--- количество делителей числа $m$, не~меньших чем~$n$.
Вычислите
\[
    \sum_{i=1}^{1000}
        P(1000 + i, i)
\]

\item
В~летней школе состоялся шахматный турнир в~один круг.
Исход партии называется \emph{неожиданным}, если в~этой партии игрок, набравший
в~турнире меньшее количество очков, обыграл игрока, набравшего большее
количество очков.
Может~ли доля партий с~\emph{неожиданным} исходом быть больше $75\%$?
Напомним, что в~шахматах в~случае победы игрок получает одно очко, в~случае
ничьи~--- пол-очка, и ноль очков в~случае поражения.

\item
В~остроугольном треугольнике $ABC$ на~сторонах $BC$, $CA$, $AB$ отмечены пары
точек $A_1$ и~$A_2$, $B_1$ и~$B_2$, $C_1$ и~$C_2$ соответственно таким образом,
что длины отрезков $A A_1$, $A A_2$, $B B_1$, $B B_2$, $C C_1$, $C C_2$ равны.
Докажите, что середины перечисленных выше равных отрезков лежат на~одной
окружности.

\end{problems}

