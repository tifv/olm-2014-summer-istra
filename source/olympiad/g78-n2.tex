% $date: 2014-06-27

% $timetable:
%   g78r2:
%     2014-06-27:
%     - 1
%     - 2
%   g78r1:
%     2014-06-27:
%     - 1
%     - 2

\section*{Заключительная устная олимпиада, 7--8 классы}

% $authors: # in alphabetical order
% - Фёдор Ивлев
% - Лев Шабанов

% $delegate$groups: false

% $matter[-g78r2,-g78r1,-guard,-integral]:
% - verbatim: \begingroup \def\jeolmgroupname{Кенгуру, Пингвины}
% - .[guard]
% - verbatim: \endgroup% \def\jeolmgroupname

% $matter[integral]:
% - - .[-integral]

\subsection*{Довывод}

\begin{problems}

\item
Разрежьте квадрат $4 \times 4$ на~9~прямоугольников так, чтобы равные
прямоугольники не~соприкасались ни~сторонами, ни~вершинами.

% spell "сникерсов" -> "шоколадных батончиков"
% spell "сникерса" -> "шоколадных батончика"
% spell "сникерс" -> "шоколадный батончик"

\item
Федя и~Лева нашли на~дороге по~пачке 13-рублевок.
В~буфете Федя выпил 8~пакетов сока, съел 4~сникерса и~9 бутербродов.
Лева выпил 3~пакета сока, съел 8~сникерсов и~5~бутербродов.
Пакет сока, сникерс и~бутерброд стоят по~целому числу рублей.
Оказалось, что Федя может расплатиться 13-рублевками без сдачи.
Покажите, что это может сделать и~Лева.

\item
На~шахматной доске $8 \times 8$ расставлено максимальное количество слонов так,
что никакие два не~угрожают друг другу.
Докажите, что количество способов сделать это является точным квадратом.

\end{problems}


\subsection*{Вывод}

\begin{problems}

\item
Найдите наименьшее натуральное число, которое не~делится на~11, но если
уменьшить или увеличить любую его цифру на~1, то~полученное число будет
делиться на~11. 
\emph{(Замечание: цифру~0 нельзя уменьшать, а~цифру~9~--- увеличивать).}

\item
Имеется 25~кусков сыра разного веса.
Всегда~ли можно один из~этих кусков разрезать на~две части и~разложить сыр
в~два пакета так, что части разрезанного куска окажутся в~разных пакетах,
веса пакетов будут одинаковы и число кусков в пакетах также будет одинаково?

\item
\emph{Высотой} пятиугольника назовем отрезок перпендикуляра, опущенного
из~вершины на~противоположную сторону,
а~\emph{медианой}~--- отрезок, соединяющий вершину с~серединой противоположной
стороны.
Известно, что в~некотором выпуклом пятиугольнике равны 10 длин~--- длины всех
высот и~всех медиан.
Докажите, что этот пятиугольник правильный
(то~есть со~всеми равными углами и~равными сторонами).

\end{problems}


\subsection*{Послевывод}

\begin{problems}

\item
Положительные числа $x$, $y$, $z$ таковы, что
$x + y = (y + z)^2$, $y + z = (x + z)^2$, $x + z = (x + y)^2$.
Найдите эти числа.

\item
В~однокруговом футбольном турнире участвуют $n$ команд
(любые две команды сыграли ровно один матч между собой).
При каких $n$ турнир мог закончиться так, что у~каждой команды количество
ничьих равно количеству поражений?

\item
Диагонали четырехугольника $ABCD$, вписанного в~окружность с~центром $O$,
пересекаются в~точке~$M$, $\angle AMB = 60^\circ$.
На~сторонах $AD$ и~$BC$ во~внешнюю сторону построены равносторонние
треугольники $ADK$ и~$BCL$.
Прямая $KL$ пересекает описанную около $ABCD$ окружность в~точках $P$ и~$Q$.
Докажите, что $OK = LO$.

\end{problems}


\subsection*{Бонус для умников}
  
\begin{problems}

% spell "Мишино число" -> "число"

\item
Миша задумал целое число, большее чем~100.
Аня называет целое число, большее чем~1.
Если Мишино число делится на~это число, Аня выиграла, иначе Миша вычитает
из~своего числа названное, и~Аня называет следующее число.
Ей запрещается повторять числа, названные ранее.
Если Мишино число станет отрицательным~--- Аня проигрывает.
Есть~ли у~нее выигрышная стратегия?

\end{problems}

