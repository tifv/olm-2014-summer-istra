% $date: 2014-06-15

% $timetable:
%   g78r2:
%     2014-06-15:
%     - 2
%     - 3
%   g78r1:
%     2014-06-15:
%     - 2
%     - 3

\section*{Отборочная устная олимпиада, 7--8 классы}

% $authors: # in alphabetical order
% - Фёдор Бахарев
% - Глеб Погудин

% $delegate$groups: false

% $matter[-g78r2,-g78r1,-guard,-integral]:
% - verbatim: \begingroup \def\jeolmgroupname{Кенгуру, Пингвины}
% - .[guard]
% - verbatim: \endgroup% \def\jeolmgroupname

% $matter[integral]:
% - - .[-integral]

\subsection*{Довывод}

\begin{problems}

\item
Мальчик Вася пошел к речке с двумя пустыми ведрами.
Объем одного равен пяти литрам, а объем второго Вася забыл, но помнит, что он
равен либо трем, либо четырем литрам.
Как выяснить объем этого ведра?
(Разрешается наполнять любой из сосудов до краев из речки, выливать всё
содержимое в речку и переливать из одного сосуда в другой, сколько получится,
то есть либо до краев, либо всё, что было.)

\item
Разрежьте треугольник на маленькие треугольники так, чтобы ни у каких двух
маленьких треугольников стороны не совпадали.

\item
Есть три человека: $A$, $B$ и $C$.
Один лжец (всегда лжет), один рыцарь (всегда говорит правду), один нормальный
(говорит то ложь, то правду --- как придется).
$A$ сказал <<$B$ правдивее $C$>>, $B$ сказал <<$C$ правдивее $A$>>.
Что ответит $C$ на вопрос <<$A$ правдивее $B$?>>?
Считается, что рыцарь правдивее нормального, а нормальный правдивее лжеца.

\item
Несколько клеток доски $10 \times 20$ покрашены в синий цвет.
Петя разрезал его на прямоугольники так, что в каждом оказалось по пять синих
клеток.
Вася разрезал на прямоугольники так, что в каждом оказалось по семь синих
клеток.
Докажите, что Дима не сможет разрезать так, чтобы в каждом оказалось по шесть
синих клеток.

\item
Можно ли разбить числа от $1$ до $21$ на несколько групп так, чтобы наибольшее
число в каждой группе было равно сумме остальных?

\end{problems}


\subsection*{Вывод}

\begin{problems}

\item
Найти все пары простых чисел $p$ и $q$ такие, что $p^q + q^p$ тоже простое.

\item
Докажите, что в любом выпуклом $2n$-угольнике найдется диагональ, не
параллельная ни одной из сторон.

\item
Клетки доски 100 на 100 красят в два цвета.
Требуется, чтобы в любом квадрате два на два было по две клетки каждого цвета.
Сколькими способами это можно сделать?

\end{problems}


\subsection*{Послевывод}

\begin{problems}

\item
В треугольнике $ABC$ проведены высоты $B B_1$ и $C C_1$.
Пусть они пересекаются в точке $H$.
Обозначим середину $AH$ через $M$, а середину $BC$~--- через $N$.
Докажите, что $B_1 C_1$ перпендикулярен $MN$.

\item
В связном графе $n$ вершин и $m$ ребер.
Сколькими способами можно выкинуть часть ребер так, чтобы в полученном графе
степени всех вершин были четны?

\end{problems}

