% $date: 2014-06-23

% $timetable:
%   g11:
%     2014-06-23:
%     - 1
%     - 2

\section*{Письменная олимпиада, 11 класс}

% $authors: # in alphabetical order
% - Сергей Беляков
% - Антон Гусев
% - Андрей Кушнир
% - Олег Орлов

\begin{problems}

\item
Пусть $f(x) = 1 - 1 / x$.
Найдите $\underbrace{f(f( \ldots f(}_{2014}2014)\ldots))$.

\item
В~остроугольном треугольнике $ABC$ отметили середины сторон $BC$, $CA$, $AB$
и~ортоцентр и~обозначили $A_0$, $B_0$, $C_0$, $H$ соответственно.
Окружность с~центром $A_0$, проходящая через $H$, пересекает сторону $BC$
в~точках $A_1$, $A_2$.
Аналогично определяются точки $B_1$, $B_2$, $C_1$, $C_2$.
Докажите, что шесть точек $A_1$, $A_2$, $B_1$, $B_2$, $C_1$, $C_2$ лежат
на~одной окружности.

\item
$P(x) = a_n x^n + \ldots + a_1 x + a_0$~--- многочлен с~целыми коэффициентами.
Обозначим $S(k)$ сумму цифр числа $k$.
Докажите, что в~последовательности $b_n = S\bigl(|P(n)|\bigr)$ какое-то число встретится
бесконечно много раз.

\item
В~колоде $999$ карт (попарно различных).
Фокусник и~ассистент показывают фокус.
Зритель вытягивает из~колоды две любые карты.
Ассистент по~своему усмотрению добавляет к~двум вытянутым картам еще третью.
После этого зритель из~трех карт забирает себе любую.
Затем в~зал входит фокусник, которому отдают две оставшиеся карты
(не~указывая ему, какая из~них первая, а~какая вторая).
Цель фокусника~--- угадать карту в~руках у~зрителя.
Могут~ли фокусник и~ассистент, предварительно договорившись, осуществить фокус?

\end{problems}

